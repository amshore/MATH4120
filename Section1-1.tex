%----------------------------------------------------------------------------------------
%   PACKAGES AND OTHER DOCUMENT CONFIGURATIONS
%----------------------------------------------------------------------------------------

\documentclass{article} % paper and 12pt font size

\usepackage{scrextend, tikz}
\usepackage{amsmath,amsfonts,amsthm} % Math packages
\setlength\parindent{0pt} % Removes all indentation from paragraphs - comment this line for an assignment with lots of text

%----------------------------------------------------------------------------------------
%   TITLE SECTION
%----------------------------------------------------------------------------------------

\newcommand{\horrule}[1]{\rule{\linewidth}{#1}} % Create horizontal rule command with 1 argument of height

\title{ 
\normalfont \normalsize 
\textsc{MATH 4120-001 --- Abstract Algebra} \\
\horrule{0.5pt} \\[0cm] % Thin top horizontal rule
\huge Section 1.1: 4, 5*, 8*, 11 \\ % The assignment title
\horrule{2pt} \\[0cm] % Thick bottom horizontal rule
}
\author{Andrew Shore} % Your name
\date{\normalsize\today} % Today's date or a custom date
\begin{document}

\maketitle % Print the title

%----------------------------------------------------------------------------------------
%   PROBLEM 1
%----------------------------------------------------------------------------------------
\section*{Problem 1}
\textbf{Problem statement}: 
For the following problems, find the quotient q and remainder r when a is divided by b. 
\\


%Part (a)
\begin{addmargin}[1em]{0em}
\textbf{(a):} $a$ = 17; $b$ = 4 \\
\underline{Solution}: 
\begin{addmargin}[1em]{0em}
$q$ = 4; $r$ = 1. \\
Does this follow the required properties of the division algorithm? \\
(1) $bq$ + $r$ = 4*(4) + 1 = 16 + 1 = 17 = $a$ \checkmark\\
(2) $r$ = 1 $\geq$ 0 and $r$ = 1 $<$ 4 = $b$ \checkmark
\end{addmargin} 
\hfill \break
%Part b
\textbf{(b):} $a$ = 0; $b$ = 19 \\
\underline{Solution}: 
\begin{addmargin}[1em]{0em}
$q$ = 0; $r$ = 0. \\
Does this follow the required properties of the division algorithm? \\
(1) $bq$ + $r$ = 19*(0) + 0 = 0 + 0 = 0 = $a$ \checkmark\\
(2) $r$ = 0 $\geq$ 0 and $r$ = 0 $<$ 19 = $b$ \checkmark
\end{addmargin} 
\hfill \break
%Part c
\textbf{(c):} $a$ = -17; $b$ = 4 \\
\underline{Solution}: 
\begin{addmargin}[1em]{0em}
$q$ = -5; $r$ = 3. \\
Does this follow the required properties of the division algorithm? \\
(1) $bq$ + $r$ = 4*(-5) + 3 = -20 + 3 = -17 = $a$ \checkmark\\
(2) $r$ = 3 $\geq$ 0 and $r$ = 3 $<$ 4 = $b$ \checkmark
\end{addmargin} 
\end{addmargin}  

%------------------------------------------------

\section*{Problem 5*}

%\lipsum[2] % Dummy text
\textbf{Problem statement}:
Let $a$ be any integer and let $b$ and $c$ be positive integers.  Suppose that when $a$ is divided by $b$, the quotient is $q$ and the remainder is $r$ so that 
\[a = bq + r     \quad \textrm{and} \quad      0 \leq r < b\]
If $ac$ is divided by $bc$, show that the quotient is $q$ and the remainder is $rc$.
\\


\underline{Solution}: 
\begin{addmargin}[1em]{0em}

\end{addmargin}


%----------------------------------------------------------------------------------------

\section*{Problem 8*}

%\lipsum[2] % Dummy text
\textbf{Problem statement}: 
Use the Division Algorithm to prove that every odd integer is either of the form $4k + 1$ or of the form $4k + 3$ for some integer $k$.
\\

\underline{Solution}: 
\begin{addmargin}[1em]{0em}

\end{addmargin}

%----------------------------------------------------------------------------------------

\section*{Problem 11}

%\lipsum[2] % Dummy text
\textbf{Problem statement}:
Prove the following version of the Division Algorithm, which holds for both positive and negative Divisors.
\\ \begin{addmargin}[1em]{0em}
\textit{Extended Division Algorithm: Let a and b be integers with b $\neq$ 0.  Then there exists unique integers q and r such that a = bq + 4 and 0 $\leq$ r $<$ $|b|$.}
\end{addmargin} \hfill \break
[\textit{Hint:} Apply Theorem 1.1 when $a$ is divided by $|b|$.  Then consider two cases ($b$ $>$ 0 and $b$ $<$ 0).]
\\


Solution: 
\begin{addmargin}[1em]{0em}

\end{addmargin}

%----------------------------------------------------------------------------------------

\end{document}