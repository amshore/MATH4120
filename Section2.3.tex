%----------------------------------------------------------------------------------------
%   PACKAGES AND OTHER DOCUMENT CONFIGURATIONS
%----------------------------------------------------------------------------------------

\documentclass{article} % paper and 12pt font size

\usepackage{scrextend, tikz, amssymb}
\usepackage{amsmath,amsfonts,amsthm} % Math packages
\setlength\parindent{0pt} % Removes all indentation from paragraphs - comment this line for an assignment with lots of text

%----------------------------------------------------------------------------------------
%   TITLE SECTION
%----------------------------------------------------------------------------------------

\newcommand{\horrule}[1]{\rule{\linewidth}{#1}} % Create horizontal rule command with 1 argument of height

\title{ 
\normalfont \normalsize 
\textsc{MATH 4120-001 --- Abstract Algebra} \\
\horrule{0.5pt} \\[0cm] % Thin top horizontal rule
\huge Section 2.3: 1d, 2c, 4*, 10*, 15, 17 \\ % The assignment title
\horrule{2pt} \\[0cm] % Thick bottom horizontal rule
}
\author{Andrew Shore} % Your name
\date{\normalsize\today} % Today's date or a custom date
\begin{document}

\maketitle % Print the title

%----------------------------------------------------------------------------------------
%   PROBLEM 1
%----------------------------------------------------------------------------------------
\section*{Problem 1d}


\textbf{Problem statement}: Find all of the units in $\mathbb{Z}_{10}$
\\

\underline{Solution}: 
\begin{addmargin}[1em]{0em}
$(0, 10) = 0 $\\
$(1, 10) = 1 $ so $[1]$ is a unit\\
$(2, 10) = 2 $\\
$(3, 10) = 1 $ so $[3]$ is a unit\\
$(4, 10) = 2 $\\
$(5, 10) = 5 $\\
$(6, 10) = 2 $\\
$(7, 10) = 1 $ so $[7]$ is a unit\\
$(8, 10) = 2 $\\
$(9, 10) = 1 $ so $[9]$ is a unit\\
\end{addmargin}    

\newpage
%----------------------------------------------------------------------------------------

\section*{Problem 2c}

\textbf{Problem statement}: Find all the zero divisors in $\mathbb{Z}_{9}$
\\

\underline{Solution}: 
\begin{addmargin}[1em]{0em}
$[0]$ is not a zero divisor by definition\\
$[1]$ is not a zero divisor \\
$[2]$ is not a zero divisor \\
$[3] * [3] = [0]$ so $[3]$ is a zero divisor \\
$[4]$ is not a zero divisor \\
$[5]$ is not a zero divisor \\
$[6] * [3] = [0]$ so $[6]$ is a zero divisor \\
$[7]$ is not a zero divisor \\
$[8]$ is not a zero divisor
\end{addmargin}

\newpage
%----------------------------------------------------------------------------------------

\section*{Problem 4*}


\textbf{Problem statement}: How many solutions does the equation $6x = 4$ have in:
\\

\begin{addmargin}[1em]{0em}
\textbf{(a)}:$\mathbb{Z}_{7}$
\begin{addmargin}[1em]{0em}
\underline{Solution}: \\
$[6] * [0] = [0]$ \\
$[6] * [1] = [6]$ \\
$[6] * [2] = [5]$ \\
$[6] * [3] = [4]$ \\
$[6] * [4] = [3]$ \\
$[6] * [5] = [2]$ \\
$[6] * [6] = [1]$ \\
Number of solutions: 1
\begin{addmargin}[1em]{0em}
\end{addmargin}
\end{addmargin}

\textbf{(b)}:$\mathbb{Z}_{8}$
\begin{addmargin}[1em]{0em}
\underline{Solution}: \\
$[6] * [0] = [0]$ \\
$[6] * [1] = [6]$ \\
$[6] * [2] = [4]$ \\
$[6] * [3] = [2]$ \\
$[6] * [4] = [0]$ \\
$[6] * [5] = [6]$ \\
$[6] * [6] = [4]$ \\
$[6] * [7] = [2]$ \\
Number of Solutions: 2
\begin{addmargin}[1em]{0em}
\end{addmargin}
\end{addmargin}

\textbf{(c)}:$\mathbb{Z}_{9}$
\begin{addmargin}[1em]{0em}
\underline{Solution}: \\
$[6] * [0] = [0]$ \\
$[6] * [1] = [6]$ \\
$[6] * [2] = [3]$ \\
$[6] * [3] = [0]$ \\
$[6] * [4] = [6]$ \\
$[6] * [5] = [3]$ \\
$[6] * [6] = [0]$ \\
$[6] * [7] = [6]$ \\
$[6] * [8] = [3]$ \\
Number of Solutions: 0
\begin{addmargin}[1em]{0em}
\end{addmargin}
\end{addmargin}

\newpage
\textbf{(d)}:$\mathbb{Z}_{10}$
\begin{addmargin}[1em]{0em}
\underline{Solution}: \\
$[6] * [0] = [0]$ \\
$[6] * [1] = [6]$ \\
$[6] * [2] = [2]$ \\
$[6] * [3] = [8]$ \\
$[6] * [4] = [4]$ \\
$[6] * [5] = [0]$ \\
$[6] * [6] = [6]$ \\
$[6] * [7] = [2]$ \\
$[6] * [8] = [8]$ \\
$[6] * [9] = [4]$ \\
Number of Solutions: 2
\begin{addmargin}[1em]{0em}
\end{addmargin}
\end{addmargin}
\end{addmargin}

\newpage
%----------------------------------------------------------------------------------------

\section*{Problem 10*}

\textbf{Problem statement}: Prove that every nonzero element of $\mathbb{Z}_n$ is either a unit or a zero divisor, but not both.
\\


\underline{Solution}: 
\begin{addmargin}[1em]{0em}
\begin{proof}
Let $a \in \mathbb{Z}_n$ be nonzero. \\
\underline{\textbf{Case 1:} $(a,n) = 1$}
\begin{addmargin}[1em]{0em}
By theorem 2.10, then $a$ is a unit of $\mathbb{Z}_n$.
\\Also, because $(a,n) = 1$, there do not exist any $0 \leq b < n$ such that $ab = nk$.
\\Thus, $n \not| ab \implies ab \not\equiv 0 ($mod $n)$.
\\So, $[a][b] \neq [0]$.
\\Because $b$ ranges over all values from $0$ to $n-1$, this statement is valid for all $[b]$ in $\mathbb{Z}_n$.
\\Thus, $a$ is not a zero divisor of $\mathbb{Z}_n$.
\\Therefore $a$ is a unit but not a zero divisor of $\mathbb{Z}_n$.
\end{addmargin}
\underline{\textbf{Case 2:} $(a,n) = d > 1$}
\begin{addmargin}[1em]{0em}
By theorem 2.10, $a$ is not a unit of $\mathbb{Z}_n$.
\\Because $(a,n) = d$, there exist $b,c \in \mathbb{Z}_n$
\\So $a = dc$ and $n = bd$ 
\\Thus, $ab = bcd \implies ab = cn$
\\However, in $\mathbb{Z}_n$, $cn = 0$, so $ab = 0$.
\\Thus, $a$ is a zero divisor of $\mathbb{Z}_n$.
\\Therefore $a$ is a zero divisor but not a unit of $\mathbb{Z}_n$.
\end{addmargin}
Therefore, because all greatest common divisors are positive integers, we have tested for all non-zero possible values of $a$ in $\mathbb{Z}_n$.
\\Thus, all non-zero $a$ in $\mathbb{Z}_n$ are either units or zero divisors but not both.
\end{proof}
\end{addmargin}

\newpage
%----------------------------------------------------------------------------------------


\section*{Problem 15}

\textbf{Problem statement}: Let $a, a_1, b, b_1, n, n_1, u, v$ be integers with $n > 1$, $d = (a,n)$, $d|b$, $au + nv = d$, $a = da_1$, $b = db_1$, and $n = dn_1$.  Then $[ub_1], [ub_1 + n_1], [ub_1 + 2n_1], [ub_1 + 3n_1], ..., [ub_1 + (d-1)n_1]$ are solutions to $[a]x = [b]$
\\ \hfill \break
Use this to solve the following equations:
\\

\begin{addmargin}[1em]{0em}
\textbf{(a)}:$16x = 9$ in $\mathbb{Z}_{18}$
\begin{addmargin}[1em]{0em}
\underline{Solution}: \\
$a = 16$ \\
$b = 9$ \\
$n = 18$ \\
$d = (a,n) = 2$ \\
$2 \not| 9$, so there is no solution $x \in \mathbb{Z}_{18}$
\begin{addmargin}[1em]{0em}
\end{addmargin}
\end{addmargin}

\textbf{(b)}:$25x = 10$ in $\mathbb{Z}_{8}$
\begin{addmargin}[1em]{0em}
\underline{Solution}: 
$a = 25$ \\
$b = 10$ \\
$n = 8$ \\
$d = (a,n) = 1$ \\
$1|10$ \checkmark \\
$u = 1$ \\
$v = -3$ \\
$a_1 = 25$ \\
$b_1 = 10$ \\
$n_1 = 8$\\
Thus, $x$ has solution: $[1 * 10] = [2]$
\begin{addmargin}[1em]{0em}
\end{addmargin}
\end{addmargin}
\end{addmargin}

\newpage
%----------------------------------------------------------------------------------------

\section*{Problem 17}

\textbf{Problem statement}: Prove that the product of two units in $\mathbb{Z}_n$ is also a unit.
\\


\underline{Solution}: 
\begin{addmargin}[1em]{0em}
\begin{proof}
Suppose $a,b$ are units in $\mathbb{Z}_n$.
\\Then $an = 1$ and $bn = 1$
\\Suppose we have $ab = x$.
\\Then $(an)(bn) = xn^2$
\\However, $xn^2 = x$ because we are in $\mathbb{Z}_n$.
\\So, $(1)(1) = x \implies x = 1$.
\\Thus, $ab = 1$, so $ab$ is a unit of $\mathbb{Z}_n$.
\end{proof}
\end{addmargin}

\newpage
%----------------------------------------------------------------------------------------

\end{document}