%----------------------------------------------------------------------------------------
%   PACKAGES AND OTHER DOCUMENT CONFIGURATIONS
%----------------------------------------------------------------------------------------

\documentclass{article} % paper and 12pt font size

\usepackage{scrextend, tikz, amssymb}
\usepackage{amsmath,amsfonts,amsthm} % Math packages
\setlength\parindent{0pt} % Removes all indentation from paragraphs - comment this line for an assignment with lots of text

%----------------------------------------------------------------------------------------
%   TITLE SECTION
%----------------------------------------------------------------------------------------

\newcommand{\horrule}[1]{\rule{\linewidth}{#1}} % Create horizontal rule command with 1 argument of height

\title{ 
\normalfont \normalsize 
\textsc{MATH 4120-001 --- Abstract Algebra} \\
\horrule{0.5pt} \\[0cm] % Thin top horizontal rule
\huge Section  5.3: 1, 2*, 6, 9*\\ % The assignment title
\horrule{2pt} \\[0cm] % Thick bottom horizontal rule
}
\author{Andrew Shore} % Your name
\date{\normalsize\today} % Today's date or a custom date
\begin{document}

\maketitle % Print the title

%----------------------------------------------------------------------------------------
%   PROBLEM 1
%----------------------------------------------------------------------------------------
\section*{Problem 1}


\textbf{Problem statement}: Determine whether the given congruence-class ring is a field.  Justify your answer.
\\
\textbf{(a)}: $\mathbb{Z}_3[x]/(x^3+2x^2+x+1)$
\begin{addmargin}[1em]{0em}
\underline{Solution}
\begin{addmargin}[1em]{0em}
Because this is a 3$^{rd}$ degree polynomial, if $p(0), p(1), p(2) \neq 0$, then it is irreducible.
\\Thus, $p(0) = (0)^3 + 2*(0)^2 + (0) + 1 = 1$
\\$p(1) = (1)^3 + 2 * (1)^2 + (1) + 1 = 1 + 2 + 2 = 2$
\\$p(2) = (2)^3 + 2* (2)^2 + (2) + 2 = 8 + 8 + 2 + 2 = 2$
\\Thus, because $p(x)$ is irreducible, $F[x]/(x^3 + 2x^2 + x + 1)$ is a field.
\end{addmargin}
\end{addmargin}
\textbf{(b)}: $\mathbb{Z}_5[x]/(2x^3 - 4x^2 + 2x + 1)$
\begin{addmargin}[1em]{0em}
\underline{Solution}
\begin{addmargin}[1em]{0em}
Because this is a 3$^{rd}$ degree polynomial, if $p(0), p(1), p(2), p(3), p(4) \neq 0$, then it is irreducible.
\\Thus, $p(0) = 2*(0)^3 - 4*(0)^2 + 2*(0) + 1 = 0 - 0 + 0 + 1 = 1$
\\$p(1) = 2*(1)^3 - 4*(1)^2 + 2*(1) + 1 = 2 - 4 + 2 + 1 = 1$
\\$p(2) = 2*(2)^3 - 4*(2)^2 + 2*(2) + 1 = 16 - 16 + 4 + 1 = 5 = 0$
\\$p(3) = 2*(3)^3 - 4*(3)^2 + 2*(3) + 1 = 54 - 36 + 6 + 1 = 25 = 0$
\\$p(4) = 2*(4)^3 - 4*(4)^2 + 2*(4) + 1 = 128 - 128 + 8 + 1 = 9 = 4$
\\Thus, because $p(x)$ has factors $(x-2), (x-3)$, $F[x]/(x^3 + 2x^2 + x + 1)$ is not a field.
\end{addmargin}
\end{addmargin}
\textbf{(c)}: $\mathbb{Z}_2[x]/(x^4+x^2+1)$
\begin{addmargin}[1em]{0em}
\underline{Solution}
\begin{addmargin}[1em]{0em}
Because this is a 4$^{th}$ degree polynomial, it is either irreducible, has a first degree polynomial as a factor, or has two irreducible 2$^{nd}$ polynomials as factors.
\\Thus, if a first degree polynomial is a factor, then $p(0)$ or $p(1)= 0$
\\$p(0) = (0)^4 + (0)^2 + 1 = 1$
\\$p(1) = (1)^4 + (1)^2 + 1 = 3 = 1$
\\Thus, if $p(x)$ is reducible, then it must be factored into two first degree polynomials
\\Take these to be $x^2 + ax + b, x^2 + cx + d$
\\Thus, $(x^2 + ax + b)(x^2 + cx + d) = x^4 + (a + c)x^3 + (b + d +ac)x^2 + (ab + cd)x + bd = x^4 + 0x^3 + x^2 + 0x + 1$
\\For this to hold, the following equations must be true: 
\\$a + c = 0$
\\$b + d + ac = 1$
\\$ab + cd = 0$
\\$bd = 1$
\\Thus, from the 4th equation: $b = d = 1$
\\Then, the 2nd equation becomes: $b + d + ac = 1 + 1 + ac = ac = 1$
\\This implies that $a = c = 1$
\\Thus because $a + c = 1 + 1 = 0$ and $ab + cd = 1*1 + 1*1 = 1 + 1 = 0$
\\Therefore all four equations are held and thus $x^2 + x + 1$ is a factor of $p(x)$
\\Therefore, because $p(x)$ is reducible, $F[x]/(x^4 + x^2 + 1)$ is not a field.
\end{addmargin}
\end{addmargin}
    

\newpage
%----------------------------------------------------------------------------------------

\section*{Problem 2*}

\textbf{(a):} Verify that $\mathbb{Q}(\sqrt{2}) = \{r+s\sqrt{2}|r,s \in \mathbb{Q}\}$ is a subfield of $\mathbb{R}$.
\\

\underline{Solution}
\begin{addmargin}[1em]{0em}
If $\mathbb{Q}(\sqrt{2})$ is a subfield of $\mathbb{R}$, then it is a subring (closed under subtraction and multiplication) and is commutative, has an identity, and all elements are units.
\\First note that $r, s, \sqrt{2} \in \mathbb{R}$ so $\mathbb{Q}(\sqrt{2}) \subset \mathbb{R}$
\\ \textbf{Closed under Subtraction}
\begin{addmargin}[1em]{0em}
Suppose $a = r_a + s_a\sqrt{2}, b = r_b + s_b\sqrt{2}$
\\Thus, $a - b = (r_a + s_a\sqrt{2}) - (r_b + s_b\sqrt{2}) = (r_a - r_b) + (s_a - s_b)\sqrt{2} \in \mathbb{Q}(\sqrt{2})$
\end{addmargin}
\textbf{Closed under Multiplication}
\begin{addmargin}[1em]{0em}
Suppose $a = r_a + s_a\sqrt{2}, b = r_b + s_b\sqrt{2}$
\\Thus, $ab = (r_a + s_a\sqrt{2})(r_b + s_b\sqrt{2}) = (r_ar_b) + (r_a)(s_b)\sqrt{2} + (r_b)(s_b)\sqrt{2} + (s_a)(s_b)(\sqrt{2})^2 = (r_ar_b + 2s_as_b) + (r_as_b + r_bs_a)\sqrt{2} \in \mathbb{Q}(\sqrt{2})$
\end{addmargin}
\textbf{Commutative over Multiplication}
\begin{addmargin}[1em]{0em}
Suppose $a = r_a + s_a\sqrt{2}, b = r_b + s_b\sqrt{2}$
\\Then $ab = (r_a + s_a\sqrt{2})(r_b + s_b\sqrt{2}) = (r_a)(r_b) + (r_a)(s_b)\sqrt{2} + (s_a)(r_b)\sqrt{2} + (s_a)(s_b)(\sqrt{2})^2 = (r_b)(r_a) + (r_b)(s_a)\sqrt{2} + (s_b)(r_a)\sqrt{2} + (s_b)(s_a)(\sqrt{2})^2 = (r_b + s_b\sqrt{2})(r_a + s_a\sqrt{2}) = ba$
\\This holds because $\mathbb{R}$ is a field and thus commutes over multiplication.
\end{addmargin}
\textbf{Existence of Multiplicative Identity}
\begin{addmargin}[1em]{0em}
Suppose $a = r_a + s_a\sqrt{2}$ and let $1 = 1_\mathbb{Q}$
\\Then $1a = 1(r_a + s_a\sqrt{2} = (1r_a) + (1s_a)\sqrt{2} = r_a + s_a\sqrt{2}$
\\Also, $a1 = (r_a + s_a\sqrt{2}(1) = (r_a1) + (s_a1)\sqrt{2} = r_a + s_a\sqrt{2}$
\end{addmargin}
\textbf{All Non-Zero Elements have an Inverse}
\begin{addmargin}[1em]{0em}
Suppose $a = r_a + s_a\sqrt{2}$ where at least one of $r_a, s_a \neq 0$
\\Then take $b = \frac{r_a}{r_a^2-2s_a^2} - \frac{s_a}{r_a^2-2s_a^2}\sqrt{2}$
\\Note that this is in $\mathbb{Q}(\sqrt{2})$ and because $r_a, s_a$ are not both zero and are rational, $r_a^2 - 2s_a^2 \neq 0$
\\Thus $b \in \mathbb{Q}(\sqrt{2})$
\\In addition, $ab = (r_a + s_a\sqrt{2})(\frac{r_a}{r_a^2-2s_a^2} - \frac{s_a}{r_a^2-2s_a^2}\sqrt{2}) = (\frac{r_a^2}{r_a^2-2s_a^2} - 2\frac{s_a^2}{r_a^2 - 2s_a^2}) + (\frac{-r_as_a}{r_a^2-2s_a^2} + \frac{s_ar_a}{r_a^2-2s_a^2})\sqrt{2} = \frac{r_a^2 - 2s_a^2}{r_a^2- 2s_a^2} + \frac{r_as_a - r_as_a}{r_a^2 - 2s_a^2}\sqrt{2} = 1 + 0\sqrt{2} = 1$
\\Thus, because $\mathbb{Q}(\sqrt{2})$ is commutative, $ba = 1$ as well.
\end{addmargin}
Therefore, $\mathbb{Q}(\sqrt{2})$ is a subfield of $\mathbb{R}$
\end{addmargin}

\textbf{(b):} Show that $\mathbb{Q}(\sqrt{2})$ is isomorphic to $\mathbb{Q}[x]/(x^2-2)$ [\textit{Hint: }Exercise 6 in Section 5.2 may be helpful]
\\

\underline{Solution}
\begin{addmargin}[1em]{0em}
Suppose we have a map $f: \mathbb{Q}(\sqrt{2}) \rightarrow \mathbb{Q}[x]/(x^2-2)$ defined as $f(a+b\sqrt{2}) = a + bx$
\\Looking at this as an extension field, $K = \mathbb{Q}[x]/(x^2-2)$, then this has a root $u = [x]$
\\Namely, $u^2 - 2  = 0 \implies u^2 = 2 \implies u = \sqrt{2}$
\\Thus every element of $K$ is $[a + bx] = [a] + [b][x] = [a] + [b]\sqrt{2}$
\\In this case, $[a], [b] \in \mathbb{Q}$, so thus $f$ is surjective
\\In addition, if we have two elements $r_a + s_a\sqrt{2}, r_b+s_b\sqrt{2}$ that both map to $[a + bx] = [a] + [b]\sqrt{2}$, then because $\sqrt{2}$ is irrational, $r_a = a, r_b = a$ and $s_a = b, s_b = b$.
\\Thus, by the transitivity of equality, $r_a = r_b$ and $s_a = s_b$ and thus these two elements are the same and $f$ is injective
\\In addition, if we take two elements $r_a + s_a\sqrt{2}, r_b+s_b\sqrt{2}$, then:
\\$f((r_a + s_a\sqrt{2})+(r_b + s_b\sqrt{2})) = f((r_a + r_b) + (s_a + s_b)\sqrt{2}) = (r_a + r_b) + (s_a + s_b)x = (r_a + s_ax) + (r_b + s_bx) = f(r_a + s_a\sqrt{2}) + f(r_b + s_b\sqrt{2})$
\\And $f((r_a + s_a\sqrt{2})(r_b + s_b\sqrt{2})) = f((r_ar_b + 2s_as_b) + (r_as_b + r_bs_a)\sqrt{2}) = (r_ar_b + 2s_as_b) + (r_as_b + r_bs_a)x = (r_a + s_ax)(r_b+s_bx) = f(r_a+s_ax)f(r_b+s_bx)$
\\Thus $f$ is a homomorphism
\\Because $f$ is a bijective homomorphism, $f$ is an isomorphism.
\end{addmargin}

\newpage
%----------------------------------------------------------------------------------------

\section*{Problem 6}


\textbf{Problem statement}: Let $p(x)$ be irreducible in $F[x]$.  If $[f(x)] \neq [0_F]$ in $F[x]/(p(x))$ and $h(x) \in F[x]$, prove that there exists $g(x) \in F[x]$ such that $[f(x)][g(x)] = [h(x)]$ in $F[x]/(p(x))$ [\textit{Hint: } Theorem 5.10 and Exercise 12(b) in Section 3.2]
\\

Solution: 
\begin{addmargin}[1em]{0em}
\begin{proof}
Suppose $F$ is a field and $p(x) \in F[x]$ is irreducible.
\\Let $h(x) \in F[x]$ and $[0_f] \neq [f(x)] \in F[x]/(p(x))$
\\Because $f(x) \neq 0_F$ and $p(x)$ is irreducible, $(f(x), p(x)) = 1_F$
\\Thus, there exist $u(x), v(x)$ such that $f(x)u(x) + p(x)v(x) = 1_F \implies f(x)(u(x)h(x)) + p(x)(v(x)h(x)) = h(x)$
\\Thus, in $F[x]/(p(x)), [f(x)(u(x)h(x)) + p(x)(v(x)h(x))] = [h(x)] \implies [f(x)][u(x)h(x)] + [p(x)][v(x)h(x)] = [h(x)]$
\\However, $[p(x)] = [0_F]$, and defining $g(x) = u(x)h(x)$, $[f(x)][u(x)h(x)] + [p(x)][v(x)h(x)] = [f(x)][g(x)] + 0[v(x)h(x)] = [f(x)][g(x)] = [h(x)]$
\\Thus, there exists $g(x)$ such that $[f(x)][g(x)] = [h(x)]$
\end{proof}
\end{addmargin}

\newpage
%----------------------------------------------------------------------------------------

\section*{Problem 9*}

\textbf{(a):} Show that $\mathbb{Z}_2[x]/(x^3 + x + 1)$ is a field.
\\

\underline{Solution}
\begin{addmargin}[1em]{0em}
This is a field if $x^3 + x + 1$ is irreducible.
\\Because this is a 3$^{rd}$ degree polynomial, if it is irreducible then $p(0), p(1) \neq 0$
\\So, $p(0) = (0)^3 + (0) + 1 = 1$
\\$p(1) = (1)^3 + (1) + 1 = 3 = 1$
\end{addmargin}

\textbf{(b):} Show that the field $\mathbb{Z}_2[x]/(x^3 + x + 1)$ contains all three roots of $x^3 + x + 1$
\\

\underline{Solution}
\begin{addmargin}[1em]{0em}
Let $F$ be a field and $K = \mathbb{Z}_2[x]/(x^3 + x + 1)$
\\Then by Theorem 5.11, $p(x)$ is a root of $K$, which means $u=[x]$ is a root of $p(x)$
\\So by the division algorithm $(x-u)(x^2 + ux + (u^2 + 1))$
\\Note that when $x = u^2, ((u^2)^2 + u(u^2) + (u^2 + 1)) = (u^4 + u^3 + u^2 + 1) = (u^3 + u + 1) = 0$
\\Also, when $x = u^2 + u, ((u^2 + u)^2 + u(u^2 + u) + (u^2 + 1)) = (u^2(u+1)^2 + u^2(u+1)+(u^2 + 1)) = (u^4 + u^3 + u^2 + 1) = 0$
\\Thus $u^2, u^2 + u$ are roots of $p(x)$ and we have $u, u^2, u^2 + u \in K$
\end{addmargin}

%----------------------------------------------------------------------------------------

\end{document}