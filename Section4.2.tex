%----------------------------------------------------------------------------------------
%   PACKAGES AND OTHER DOCUMENT CONFIGURATIONS
%----------------------------------------------------------------------------------------

\documentclass{article} % paper and 12pt font size

\usepackage{scrextend, tikz, amssymb}
\usepackage{amsmath,amsfonts,amsthm, polynom} % Math packages
\setlength\parindent{0pt} % Removes all indentation from paragraphs - comment this line for an assignment with lots of text

%----------------------------------------------------------------------------------------
%   TITLE SECTION
%----------------------------------------------------------------------------------------

\newcommand{\horrule}[1]{\rule{\linewidth}{#1}} % Create horizontal rule command with 1 argument of height

\title{ 
\normalfont \normalsize 
\textsc{MATH 4120-001 --- Abstract Algebra} \\
\horrule{0.5pt} \\[0cm] % Thin top horizontal rule
\huge Section 4.2: 1, 3, 5bdf, 6b*, 11*, 13 \\ % The assignment title
\horrule{2pt} \\[0cm] % Thick bottom horizontal rule
}
\author{Andrew Shore} % Your name
\date{\normalsize\today} % Today's date or a custom date
\begin{document}

\maketitle % Print the title

%----------------------------------------------------------------------------------------
%   PROBLEM 1
%----------------------------------------------------------------------------------------
\section*{Problem 1}
\textbf{Problem statement}: If $f(x) \in F[x]$, show that every nonzero constant polynomial divides $f(x)$
\\

\underline{Solution}: 
\begin{addmargin}[1em]{0em}
\begin{proof}
Suppose $F$ is a field and $f(x) \in F[x]$.
\\Let $g(x) \in F[x]$ be a nonzero constant polynomial.
\\Then we can rewrite $g(x) = g \in F$
\\In addition, if $c \in F$ is the leading coefficient of $f(x)$, then there is $h(x) \in F[x]$ such that $f(x) = ch(x)$
\\Thus, by the division algorithm, there exists $q(x), r(x) \in F[x]$ such that $f(x) = q(x)g(x) + r(x)$
\\So $ch(x) = q(x)g + r(x) \implies h(x) = q(x)c^{-1}g + c^{-1}r(x)$
\\If $q(x) = h(x)g^{-1}c$, then $h(x) = h(x)g^{-1}cc^{-1}g + c^{-1}r(x)$
\\Because $h(x) = h(x) + c^{-1}r(x) \implies r(x) = 0$
\\Thus $f(x) = q(x)g \implies g|f(x)$
\end{proof}
\end{addmargin}
\newpage
%----------------------------------------------------------------------------------------
\section*{Problem 3}
\textbf{Problem statement}: If $a,b \in F$ and $a \neq b$, show that $x + a$ and $x + b$ are relatively prime in $F[x]$
\\

\underline{Solution}: 
\begin{addmargin}[1em]{0em}
\begin{proof}
I will present this proof by proving the contrapostive statement: If $a, b \in F$ and $x + a$ and $x + b$ are not relatively prime, then $a = b$.
\\Let $F$ be a field and $a, b \in F$ with $x + a, x + b \in F[x]$ such that $x +a$ and $x+b$ are not relatively prime
\\Then $(x + a, x + b) \neq 1_F$
\\However, because $1_F$ is the only polynomial of degree $0$, and the degree of gcd of $x + a$ and $x + b$ must be less or equal to $1$, then for some $c \in F$, $(x + a, x + b) = x + c$
\\This implies for some $n,m \in F$, $x + a = n(x + c)$ and $x + b = m(x + c)$
\\Because $x + a$ and $x + b$ are monic, $n = m = 1_F$
\\Thus $x + a = x + c$ and $x + b = x + c$
\\Therefore, $x + a = x + b \implies a = b$
\\Thus, the contrapositive has been proven and the original statement is proven as well.
\end{proof}
\end{addmargin}
\newpage
%----------------------------------------------------------------------------------------
\section*{Problem 5bdf}
\textbf{Problem statement}:Use the Euclidean Algorithm to find the gcd (which is monic) of the given polynomials:
\\

\textbf{(b): }$x^5 + x^4 + 2x^3 - x^2 -x - 2$ and $x^4 + 2x^3 + 5x^2 + 4x + 4$ in $\mathbb{Q}[x]$
\begin{addmargin}[1em]{0em}
\underline{Solution}: 
\begin{addmargin}[1em]{0em}
$x^5 + x^4 + 2x^3 - x^2 - x - 2 = (x-1)(x^4 + 2x^3 + 5x^2 + 4x + 4) + (-x^3 - x + 2)$
\\$x^4 + 2x^3 + 5x^2 + 4x + 4 = (x + 2)(x^3 + x - 2) + (4x^2 + 4x + 8)$
\\$x^3 + x - 2 = (x - 1)(x^2 + x + 2) + (0)$
\\Thus, $(x^5 + x^4 + 2x^3 - x^2 -x - 2, x^4 + 2x^3 + 5x^2 + 4x + 4) = x^2 + x + 2$
\end{addmargin}
\end{addmargin}


\textbf{(d): }$4x^4 + 2x^3 + 6x^2 + 4x + 5$ and $3x^3 + 5x^2 + 6x$ in $\mathbb{Z}_7[x]$
\begin{addmargin}[1em]{0em}
\underline{Solution}: 
\begin{addmargin}[1em]{0em}
$4x^4 + 2x^3 + 6x^2 + 4x + 6 = (-x)(3x^3 + 5x^2 + 6x) + (5x^2 + 4x + 5)$
\\$3x^3 + 5x^2 + 6x = (-5x - 2)(5x^2 + 4x + 5) + (4x + 3)$
\\$5x^2 + 4x + 5 = (-4x - 3)(4x + 3) + (0)$
\\Also, because we are in $\mathbb{Z}_7, 4x + 3 = 4x - 4 = 4(x-1)$
\\Thus, $(4x^4 + 2x^3 + 6x^2 + 4x + 5, 3x^3 + 5x^2 + 6x) = x-1$
\end{addmargin}
\end{addmargin}
\textbf{(f): }$x^4 + x + 1$ and $x^2 + x + 1$ in $\mathbb{Z}_2[x]$
\begin{addmargin}[1em]{0em}
\underline{Solution}: 
\begin{addmargin}[1em]{0em}
$x^4 + x + 1 = (x)(x^3 + x + 1) + (x^2 + 1)$
\\$x^3 + x + 1 = (x)(x^2 + 1) + (1)$
\\$x^2 + 1 = (x^2 + 1)(1) + (0)$
\\Thus, $(x^4 + x + 1, x^2 + x + 1) = 1$ and the two polynomials are relatively prime.
\end{addmargin}
\end{addmargin}
\newpage
%----------------------------------------------------------------------------------------
\section*{Problem 6b*}
\textbf{Problem statement}: Express each of the gcd's for Problem 5b as a linear combination of the two given polynomials
\\

\underline{Solution}: 
\begin{addmargin}[1em]{0em}
For simplicity define $a = x^5 + x^4 + 2x^3 -x^2 -x - 2, b = x^4 + 2x^3 + 5x^2 + 4x + 4$
\\$a = (x-1)(b) + (-1)(x^3 + x - 2)$
\\$(-1)a + (x-1)b = (x^3 + x - 2)$
\\$b = (x + 2)(x^3 + x - 2) + 4(x^2 + x + 2)$
\\$b = (x + 2)((-1)a + (x-1)b) + (4)(x^2 + x + 2)$
\\$(x+2)a - ((x+2)(x-1) - 1)b = (4)(x^2 + x + 2)$
\\$(\frac{1}{4}x + \frac{1}{2})a - \frac{1}{4}(x^2 + x - 3)b = x^2 + x + 2$
\\$(\frac{1}{4}x + \frac{1}{2})a - (\frac{1}{4}x^2 + \frac{1}{4}x - \frac{3}{4})b = x^2 + x + 2$
\end{addmargin}
\newpage
%----------------------------------------------------------------------------------------
\section*{Problem 11*}
\textbf{Problem statement}: Prove the following statement: Let $F$ be a field and $a(x),b(x) \in F[x]$, not both zero.  Then there is a unique greatest common divisor $d(x)$ of $a(x)$ and $b(x)$.  Furthermore, there are (not necessarily unique) polynomials $u(x)$ and $v(x)$ such that $d(x) = a(x)u(x) + b(x)v(x)$
\\

\underline{Solution}: 
\begin{addmargin}[1em]{0em}
\begin{proof}
Let $F$ be a field and $a(x),b(x) \in F[x]$ not both zero.
\\Let $S = \{a(x)m(x) + b(x)n(x) | m(x),n(x) \in F[x]\}$
\\Note that $a(x), b(x) \in S$, so $S$ contains at least one nonzero polynomial.
\\So, the set $R = \{deg(a(x)m(x) + b(x)n(x) | m(x),n(x) \in F[x], a(x)m(x) + b(x)n(x) \neq 0_F)\}$ is nonempty and thus has a smallest element $r \in R$ by the Well Ordering Principle.
\\In addition, there is a corresponding element $s \in S$ such that $deg(s) = r$
\\Also, to make $s$ a monic polynomial, suppose for some monic polynomial $t \in S$, with $d \in F$ the leading coefficient of $s$, $s = dt$
\\Because $F$ is a field, $d$ is nonzero and thus has an inverse, so $t = d^{-1}s$
\\Thus, by the definition of $S$, for some $u, v \in F[x]$, $t = au + bv$
\\Now, for $t$ to be the gcd of $a$ and $b$, it must divide $a$ and $b$ and any polynomial that also divides $a$ and $b$ must be of lesser or equal degree
\\ \textbf{Divides $a$ and $b$}
\begin{addmargin}[1em]{0em}
By the division algorithm, there are polynomials $q, r, u, v \in F[x]$ such that $a = tq + r$ with $0 \geq deg(r) < deg(t)$ or $r = 0_F$
\\So $r = a - tq = a - (au + bv)q = a - aqu - bvq = a(1-qu) + b(-vq)$
\\Thus $r \in S$
\\However, by the definition of $t$, the degree of $t$ is the smallest among the elements of $S$ (corresponding to the least element in $R$)
\\Thus, because $deg(r) < deg(t)$ is impossible, this implies $r = 0_F$
\\Therefore, $a = tq$ which implies that $t|a$.
\\In addition there are (possibly different) polynomials $q, r, u , v \in F[x]$ such that $b = tq + r$ with $0 \geq deg(r) < deg(t)$ or $r = 0_F$
\\So $r = b - tq = b - (au + bv)q = b - aqu - bvq = a(-qu) + b(1-vq)$
\\Thus $r \in S$
\\However, by the definition of $t$, the degree of $t$ is the smallest among the elements of $S$ (corresponding to the least element in $R$)
\\Thus, because $deg(r) < deg(t)$ is impossible, this implies $r = 0_F$
\\Therefore, $b = tq$ which implies that $t|b$.
\\Thus, $t|a$ and $t|b$
\end{addmargin}
\textbf{Maximum degree monic polynomial}
\begin{addmargin}[1em]{0em}
Let $c \in F[x]$ be a common divisor of $a$ and $b$
\\Suppose $u, v \in F[x]$ such that $au + bv = t$
\\Then for some polynomials $n, m \in F[x]$, $a = cn$ and $b = cm$.
\\Thus $t = au + bv = cnu + cmv = c(nu + mv)$
\\Therefore, $deg(t) \geq deg(c) + deg(nu + mv)$ by Theorem 4.7
\\And thus, $deg(t) \geq deg(c)$
\end{addmargin}
Therefore $t$ is the greatest common divisor of $a$ and $b$
\\Finally, to show that $t$ is the unique greatest common divisor, suppose $d$ is another greatest common divisor of $a$ and $b$
\\Thus $d$ is a common divisor and there exist $n,m \in F[x]$ such that $a = nd$ and $b = md$
\\Thus, $t = au + bv = (nd)u + (md)v = d(nu + mv)$
\\Therefore $deg(t) \geq deg(d) + deg(nu + mv)$
\\However, because $t$ and $d$ are gcd's, $deg(t) = deg(d)$
\\This implies that $0 \geq deg(nu + mv) \implies deg(nu + mv) = 0$
\\Therefore, $nu + mv = c$ where $c \in F$
\\Thus, $t = cd$ and because $t$ and $d$ are monic, the leading coefficients imply $1_F = c$
\\Thus $t = d$ and $t$ is the unique gcd of $a$ and $b$.
\end{proof}
\end{addmargin}
\newpage
%----------------------------------------------------------------------------------------
\section*{Problem 13}
\textbf{Problem statement}: Prove the following statement: Let $F$ be a field and $a(x),b(x),c(x) \in F[x]$.  If $a(x)|b(x)c(x)$ and $a(x)$ and $b(x)$ are relatively prime, then $a(x)|c(x)$.
\\

\underline{Solution}: 
\begin{addmargin}[1em]{0em}
\begin{proof}
Let $F$ be a field and $a(x), b(x), c(x) \in F[x]$
\\Let $a(x)|b(x)c(x)$
\\Then for some $k(x) \in F[x]$, $b(x)c(x) = a(x)k(x)$
\\Suppose $a(x)$ and $b(x)$ are relatively prime.
\\Then for some $u(x),v(x) \in F[x]$, $1_F = a(x)u(x) + b(x)v(x)$
\\Thus, $c(x) = c(x)a(x)u(x) + c(x)b(x)v(x) = c(x)a(x)u(x) + a(x)k(x)v(x) = a(x)(c(x)u(x) + k(x)v(x)) \implies a(x)|c(x)$
\end{proof}
\end{addmargin}

\end{document}
