%----------------------------------------------------------------------------------------
%   PACKAGES AND OTHER DOCUMENT CONFIGURATIONS
%----------------------------------------------------------------------------------------

\documentclass{article} % paper and 12pt font size

\usepackage{scrextend, tikz, amssymb}
\usepackage{amsmath,amsfonts,amsthm} % Math packages
\setlength\parindent{0pt} % Removes all indentation from paragraphs - comment this line for an assignment with lots of text

%----------------------------------------------------------------------------------------
%   TITLE SECTION
%----------------------------------------------------------------------------------------

\newcommand{\horrule}[1]{\rule{\linewidth}{#1}} % Create horizontal rule command with 1 argument of height

\title{ 
\normalfont \normalsize 
\textsc{MATH 4120-001 --- Abstract Algebra} \\
\horrule{0.5pt} \\[0cm] % Thin top horizontal rule
\huge Section 1.2: 1c, 2, 5, 13, 15c, 15j*, 23*, 28  \\ % The assignment title
\horrule{2pt} \\[0cm] % Thick bottom horizontal rule
}
\author{Andrew Shore} % Your name
\date{\normalsize\today} % Today's date or a custom date
\begin{document}

\maketitle % Print the title


%----------------------------------------------------------------------------------------
\section*{Problem 1c}


\textbf{Problem statement}: Find the great common divisor of (112, 57).
\\

\underline{Solution}: 
\begin{addmargin}[1em]{0em}

\end{addmargin}    

\newpage
%----------------------------------------------------------------------------------------

\section*{Problem 2}

\textbf{Problem statement}: Prove that if $b|a$ if and only if $(-b)|a$
\\

\underline{Solution}: 
\begin{addmargin}[1em]{0em}
\textbf{(a)}:
\begin{addmargin}[1em]{0em}
\end{addmargin}
\end{addmargin}

\newpage
%----------------------------------------------------------------------------------------

\section*{Problem 5}


\textbf{Problem statement}: If $a$ and $b$ are nonzero integers such that $a|b$ and $b|a$, prove that $a = \pm b$.
\\

\underline{Solution}: 
\begin{addmargin}[1em]{0em}
\begin{proof}

\end{proof}
\end{addmargin}

\newpage
%----------------------------------------------------------------------------------------

\section*{Problem 13}


\textbf{Problem statement}: Suppose that $a, b, q, r$ are integers such that $a = bq + r$.  Prove each of the following statements.
\\

%part a
\begin{addmargin}[1em]{0em}
\textbf{(a)}: Every common divisor $c$ of $a$ and $b$ is also a common divisor of $b$ and $r$.
\\ \hfill \break
\textit{Hint:} For some integers $s$ and $t$, we have $a = cs$ and $b = ct$.  Substitute these results into $a = bq + r$ and show that $c|r$.
\\ \hfill \break
\underline{Soution}:
\begin{addmargin}[1em]{0em}
words
\end{addmargin}
\end{addmargin}
\hfill \break

%part b
\begin{addmargin}[1em]{0em}
\textbf{(b)}: Every common divisor of $b$ and $r$ is also a common divisor of $a$ and $b$.
\\ \hfill \break
\underline{Soution}:
\begin{addmargin}[1em]{0em}
words
\end{addmargin}
\end{addmargin}
\hfill \break

%part c
\begin{addmargin}[1em]{0em}
\textbf{(c)}: $(a,b)=(b,r)$
\\ \hfill \break
\underline{Soution}:
\begin{addmargin}[1em]{0em}
words
\end{addmargin}
\end{addmargin}
\hfill \break

\newpage
%----------------------------------------------------------------------------------------

\section*{Problem 15c}


\textbf{Problem statement}: Use the Euclidean Algorithm to find (1003, 456).
\\

\underline{Solution}: 
\begin{addmargin}[1em]{0em}
\begin{proof}

\end{proof}
\end{addmargin}

\newpage
%----------------------------------------------------------------------------------------

\section*{Problem 15j*}


\textbf{Problem statement}: The $gcd$ can be viewed as a linear combination of the arguments.  Using the remainder form each step of the Euclidean Algorightm, express the gcd of (1003, 456) as a linear combination of 1003 and 456.
\\

\underline{Solution}: 
\begin{addmargin}[1em]{0em}
\begin{proof}

\end{proof}
\end{addmargin}

\newpage
%----------------------------------------------------------------------------------------

\section*{Problem 23*}


\textbf{Problem statement}: Use induction to show that if $(a,b) = 1$, then $(a,b^n) = 1$ for all $n \geq 1$.
\\

\underline{Solution}: 
\begin{addmargin}[1em]{0em}
\begin{proof}

\end{proof}
\end{addmargin}

\newpage
%----------------------------------------------------------------------------------------

\section*{Problem 28}


\textbf{Problem statement}: Prove that a positive integer is divisible by 9 if and only in the sum of its digits is divisible by 3 [\textit{Hint}: $10^3 = 999 + 1$ and similarly for other powers of 10].
\\

\underline{Solution}: 
\begin{addmargin}[1em]{0em}
\begin{proof}

\end{proof}
\end{addmargin}

%----------------------------------------------------------------------------------------

\end{document}