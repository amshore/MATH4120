%----------------------------------------------------------------------------------------
%   PACKAGES AND OTHER DOCUMENT CONFIGURATIONS
%----------------------------------------------------------------------------------------

\documentclass{article} % paper and 12pt font size

\usepackage{scrextend, tikz, amssymb}
\usepackage{amsmath,amsfonts,amsthm} % Math packages
\setlength\parindent{0pt} % Removes all indentation from paragraphs - comment this line for an assignment with lots of text

\reversemarginpar

%----------------------------------------------------------------------------------------
%   TITLE SECTION
%----------------------------------------------------------------------------------------

\newcommand{\horrule}[1]{\rule{\linewidth}{#1}} % Create horizontal rule command with 1 argument of height

\title{ 
\normalfont \normalsize 
\textsc{MATH 4120-001 --- Abstract Algebra} \\
\horrule{0.5pt} \\[0cm] % Thin top horizontal rule
\huge Section 1.2: 1c, 2, 5, 13, 15c, 15j*, 23*, 28  \\ % The assignment title
\horrule{2pt} \\[0cm] % Thick bottom horizontal rule
}
\author{Andrew Shore} % Your name
\date{\normalsize\today} % Today's date or a custom date
\begin{document}

\maketitle % Print the title


%----------------------------------------------------------------------------------------
\section*{Problem 1c}


\textbf{Problem statement}: Find the great common divisor of (112, 57).
\\

\underline{Solution}: 
\begin{addmargin}[1em]{0em}
To solve this, I will use the division algorithm:
\\(1): $112 = (1)(57) + 55$
\\(2): $57 = (1)(55) + 2$
\\(3): $55 = (27)(2) + 1$
\\(4): $27 = (27)(1) + 0$
\\Thus $(112, 57) = 1$ and they are coprime.
\end{addmargin}    

\newpage
%----------------------------------------------------------------------------------------

\section*{Problem 2}

\textbf{Problem statement}: Prove that if $b|a$ if and only if $(-b)|a$
\\

\underline{Solution}: 
\begin{addmargin}[1em]{0em}
\begin{proof}
Let $a,b \in \mathbb{Z}$.
\\ \marginpar{$\Rightarrow$}
Suppose that $b|a$.
\\Then there exists $k \in \mathbb{Z}$ such that $a = bk$.
\\Furthermore, $a = bk = (-b)(-k) = (-b)t$ where $t = -k$, which is an integer.
\\Thus by definition, $(-b)|a$.
\\ \marginpar{$\Leftarrow$}
Suppose that $(-b)|a$
\\Then there exists $k \in \mathbb{Z}$ such that $a = (-b)k$.
\\Furthermore, $a = (-b)k = (-(-b))(-k) = b(-k) = bt$ where $t = -k$, which is an integer.
\\Thus by definition, $b|a$.
\end{proof}
\end{addmargin}

\newpage
%----------------------------------------------------------------------------------------

\section*{Problem 5}


\textbf{Problem statement}: If $a$ and $b$ are nonzero integers such that $a|b$ and $b|a$, prove that $a = \pm b$.
\\

\underline{Solution}: 
\begin{addmargin}[1em]{0em}
\begin{proof}
Let $a, b$ be nonzero integers such that $a|b$ and $b|a$
\\Then by definition, $b = am$ and $a = bn$ where $m,n \in \mathbb{Z}$.
\\So, solving for $a$, $a = bn = (am)(n) = a(mn) \implies mn = 1$.
\\However, because $mn \in \mathbb{Z}$, the only solutions are $m = 1, n = 1$ and $m = -1, n = -1$ as $1$ and $-1$ are the only nonzero integers to have multiplicative inverses.
\\Using these solutions in the original equations, they resolve to $a = b, a = -b, b = a, b = -a$ or put simply, $a = \pm b$.
\end{proof}
\end{addmargin}

\newpage
%----------------------------------------------------------------------------------------

\section*{Problem 13}


\textbf{Problem statement}: Suppose that $a, b, q, r$ are integers such that $a = bq + r$.  Prove each of the following statements.
\\

%part a
\begin{addmargin}[1em]{0em}
\textbf{(a)}: Every common divisor $c$ of $a$ and $b$ is also a common divisor of $b$ and $r$.
\\ \hfill \break
\textit{Hint:} For some integers $s$ and $t$, we have $a = cs$ and $b = ct$.  Substitute these results into $a = bq + r$ and show that $c|r$.
\\ \hfill \break
\underline{Solution}:
\begin{addmargin}[1em]{0em}
\begin{proof}
Let $a, b, q, r, c \in \mathbb{Z}$ be such that $a = bq + r$ and $c|a$ and $c|b$.
\\Then there exists $m, n \in \mathbb{Z}$ such that $a = cm$ and $b = cn$.
\\By substituting, $cm = cnq + r \implies c(m-nq) = r$.
\\But $m,n,q \in \mathbb{Z}$, so this implies, $c|r$.
\end{proof}
\end{addmargin}
\end{addmargin}
\hfill \break

%part b
\begin{addmargin}[1em]{0em}
\textbf{(b)}: Every common divisor of $b$ and $r$ is also a common divisor of $a$ and $b$.
\\ \hfill \break
\underline{Solution}:
\begin{addmargin}[1em]{0em}
\begin{proof}
Let $a, b, q, r, c \in \mathbb{Z}$ be such that $a = bq + r$ and $c|b$ and $c|r$.
\\Thus, c is defined as a common divisor of $b$ and $r$.
\\In this proof, because $c$ is arbitrary, if this holds for $c$, it holds for all common divisors of $b$ and $r$.
\\So, there exist $m,n \in \mathbb{Z}$ such that $b = cm$ and $r = cn$.
\\By substitution, $a = cmq + cn = c(mq + n)$.
\\Because $m,q,n \in \mathbb{Z}$, this implies $c|a$.
\\Therefore, $c$ is a common divisor of $a$ and $b$.
\end{proof}
\end{addmargin}
\end{addmargin}
\hfill \break

%part c
\begin{addmargin}[1em]{0em}
\textbf{(c)}: $(a,b)=(b,r)$
\\ \hfill \break
\underline{Soution}:
\begin{addmargin}[1em]{0em}
\begin{proof}
Let $a, b, q, r \in \mathbb{Z}$ be such that $a = bq + r$.
\\Suppose that $g$ is the greatest common divisor of $a$ and $b$.
\\Because $g$ is a common divisor of $a$ and $b$, by part (a), $g$ is a divisor of $r$.
\\Thus, $g$ is a common divisor of $b$ and $r$.
\\Now, suppose $h$ is the greatest common divisor of $b$ and $r$.
\\Because $g$ is a divisor of $b$ and $r$ and $h$ is the greatest common divisor of $b$ and $r$, $h \geq g$.
\\However, because $h$ is a divisor of $b$ and $r$, by part (b), $h$ is a divisor of $a$ and $b$.
\\Thus, because $h$ is a divisor of $a$ and $b$ and $g$ is the greatest common divisor of $a$ and $b$, $g \geq h$.
\\Therefore, $g \geq h \geq g \implies g = h$, or in equivalent notation, $(a,b) = (b,r)$.
\end{proof}
\end{addmargin}
\end{addmargin}

\newpage
%----------------------------------------------------------------------------------------

\section*{Problem 15c}


\textbf{Problem statement}: Use the Euclidean Algorithm to find (1003, 456).
\\

\underline{Solution}: 
\begin{addmargin}[1em]{0em}
(1): $1003 = (2)(456) + 91$
\\(2): $456 = (5)(91) + 1$
\\(3): $91 = (1)(91) + 0$
\\Thus, by the Euclidean Algorithm, $(1003, 456) = 1$
\end{addmargin}

\newpage
%----------------------------------------------------------------------------------------

\section*{Problem 15j*}


\textbf{Problem statement}: The $gcd$ can be viewed as a linear combination of the arguments.  Using the remainder from each step of the Euclidean Algorightm, express the gcd of (1003, 456) as a linear combination of 1003 and 456.
\\

\underline{Solution}: 
\begin{addmargin}[1em]{0em}
First solve for the remainder of each step
\\(1): $91 = (2)(456) - 1003$
\\(2): $1 = (5)(91) - 456$
\\Substituting (1) into (2):
\\$1 = (5)((2)(456) - 1003) - 456 = (9)(456) - (5)(1003)$
\\Thus, $1 = (9)*456 + (-5)*1003$
\end{addmargin}

\newpage
%----------------------------------------------------------------------------------------

\section*{Problem 23*}


\textbf{Problem statement}: Use induction to show that if $(a,b) = 1$, then $(a,b^n) = 1$ for all $n \geq 1$.
\\

\underline{Solution}: 
\begin{addmargin}[1em]{0em}
\begin{proof}
Suppose $a,b,n \in \mathbb{Z}$ such that $(a,b) = 1$ and $n \geq 1$.
\\ \marginpar{Base Case}
Note that $(a, b^1) = 1$ by definition.
\\ \marginpar{Induct. Hypo.}
Suppose that $(a, b^n) = 1$.
\\ \marginpar{Induction Step}
Note that $(a, b^{n+1}) = 1 \Leftrightarrow am + b^{n+1}k = 1$ for $m, k \in \mathbb{Z}$
\\This is equivalent to $am + b^n(bk) = 1$.
\\But $bk$ is an integer and is equivalent to $(a,b^n) = 1$.
\\Thus because all of these statements are equivalent, $(a,b^n) = 1 \implies (a,b^{n+1}) = 1$.
\end{proof}
\end{addmargin}

\newpage
%----------------------------------------------------------------------------------------

\section*{Problem 28}


\textbf{Problem statement}: Prove that a positive integer is divisible by 9 if and only in the sum of its digits is divisible by 3 [\textit{Hint}: $10^3 = 999 + 1$ and similarly for other powers of 10].
\\

\underline{Solution}: 
\begin{addmargin}[1em]{0em}
\begin{proof}
Let $a \in \mathbb{Z}$ be defined as its digits in base $10$.
\\So, $a = \sum_{i = 0}{a_i10^i} = \sum_{i=0}{a_i(9\sum_{j=0}^{i}{10^j} + 1)} = \sum_{i=0}{a_i9\sum_{j=0}^{i}{10^j}} + \sum_{i=0}{a_i} = 9\sum_{i=0}{a_i\sum_{j=0}^{i}{10^j}} + \sum_{i=0}{a_i}$
\\To simplify this, note that $x = \sum_{i=0}{a_i\sum_{j=0}^{i}{10^j}}$ and $y = \sum_{i=0}{a_i}$ are integers and that $y$ is the sum of the digits of $a$.
\\ \marginpar{$\Rightarrow$}
Suppose $9|a$
\\Then $9|(9x + y)$
\\Or equivalently, there is a $k \in \mathbb{Z}$, $9x + y = 9k \implies y = 9(k-x) \implies y = 3(3k - 3x)$ 
\\Thus, because $(3k - 3x) \in \mathbb{Z}$, $3|y$.
\\ \marginpar{$\Leftarrow$}
For the sake of contradiction suppose that $3|y$ and $9 \nmid a$
\\Then there does not exist an integer $k \in \mathbb{Z}$ such that $a = 9k$.
\\Equivalently, for all $k$, $9x + y \neq 9k$.
\\However, this implies $y \neq 3(3k - 3x)$, which contradicts the statement that $3|y$.
\\Thus, the assumption must be wrong, and so $9|a$.
\end{proof}
\end{addmargin}

%----------------------------------------------------------------------------------------

\end{document}