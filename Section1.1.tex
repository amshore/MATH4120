%----------------------------------------------------------------------------------------
%   PACKAGES AND OTHER DOCUMENT CONFIGURATIONS
%----------------------------------------------------------------------------------------

\documentclass{article} % paper and 12pt font size

\usepackage{scrextend, tikz, amssymb}
\usepackage{amsmath,amsfonts,amsthm} % Math packages
\setlength\parindent{0pt} % Removes all indentation from paragraphs - comment this line for an assignment with lots of text

%----------------------------------------------------------------------------------------
%   TITLE SECTION
%----------------------------------------------------------------------------------------

\newcommand{\horrule}[1]{\rule{\linewidth}{#1}} % Create horizontal rule command with 1 argument of height

\title{ 
\normalfont \normalsize 
\textsc{MATH 4120-001 --- Abstract Algebra} \\
\horrule{0.5pt} \\[0cm] % Thin top horizontal rule
\huge Section 1.1: 4, 5*, 8*, 11 \\ % The assignment title
\horrule{2pt} \\[0cm] % Thick bottom horizontal rule
}
\author{Andrew Shore} % Your name
\date{\normalsize\today} % Today's date or a custom date
\begin{document}

\maketitle % Print the title

%----------------------------------------------------------------------------------------
%   PROBLEM 1
%----------------------------------------------------------------------------------------
\section*{Problem 1}
\textbf{Problem statement}: 
For the following problems, find the quotient q and remainder r when a is divided by b. 
\\


%Part (a)
\begin{addmargin}[1em]{0em}
\textbf{(a):} $a$ = 8,126,493; $b$ = 541 \\
\underline{Solution}: 
\begin{addmargin}[1em]{0em}
$q$ = 15,021; $r$ = 132. \\
Does this follow the required properties of the division algorithm? \\
(1) $bq$ + $r$ = 541*(15,021) + 132 = 8,126,361 + 132 = 8,126,493 = $a$ \checkmark\\
(2) $r$ = 132 $\geq$ 0 and $r$ = 132 $<$ 541 = $b$ \checkmark
\end{addmargin} 
\hfill \break

%Part b
\textbf{(b):} $a$ = -9,217,654; $b$ = 617 \\
\underline{Solution}: 
\begin{addmargin}[1em]{0em}
$q$ = -14,940; $r$ = 326. \\
Does this follow the required properties of the division algorithm? \\
(1) $bq$ + $r$ = 617*(-14,940) + 326 = -9,217,980 + 326 = -9,217,654 = $a$ \checkmark\\
(2) $r$ = 326 $\geq$ 0 and $r$ = 326 $<$ 617 = $b$ \checkmark
\end{addmargin} 
\hfill \break

%Part c
\textbf{(c):} $a$ = 171,819,920; $b$ = 4,321 \\
\underline{Solution}: 
\begin{addmargin}[1em]{0em}
$q$ = 39,763; $r$ = 3997. \\
Does this follow the required properties of the division algorithm? \\
(1) $bq$ + $r$ = 4321*(39763) + 3997 = 171,815,923 + 3997 = 171,819,920 = $a$ \checkmark\\
(2) $r$ = 3997 $\geq$ 0 and $r$ = 3 $<$ 4321 = $b$ \checkmark
\end{addmargin} 
\end{addmargin}  

%------------------------------------------------
\newpage
\section*{Problem 5*}

\textbf{Problem statement}:
Let $a$ be any integer and let $b$ and $c$ be positive integers.  Suppose that when $a$ is divided by $b$, the quotient is $q$ and the remainder is $r$ so that 
\[a = bq + r     \quad \textrm{and} \quad      0 \leq r < b\]
If $ac$ is divided by $bc$, show that the quotient is $q$ and the remainder is $rc$.
\\


\underline{Solution}: 
\begin{addmargin}[1em]{0em}
Suppose we have $a,b,c, q, r$ defined as in the problem.
\\We now wish to look at $bc|ac$ 
\\By Theorem 1.1, there exist a unique quotient, $q_1$, and remainder, $r_1$, such that $ac = (bc)q_1 + r_1$ 
\\However, $a = bq + r \implies (bq + r)c = bcq_1 + r_1 \implies bcq + rc = bcq_1 + r_1$
\\Matching terms on both sides, we get $q_1 = q$ and $r_1 = rc$
\end{addmargin}


%----------------------------------------------------------------------------------------
\newpage
\section*{Problem 8*}

\textbf{Problem statement}: 
Use the Division Algorithm to prove that every odd integer is either of the form $4k + 1$ or of the form $4k + 3$ for some integer $k$.
\\

\underline{Solution}: 
\begin{addmargin}[1em]{0em}
\begin{proof} \hfill \break
Let $x \in \mathbb{Z}$ be odd.
\\By the definition of oddness, $2 \nmid x$
\\Therefore, for any integer c, $x \neq 2c$
\\By the Division Algorithm, $x$ can be written as $4k + r$ for some $k, r \in \mathbb{Z}, 0 \leq r < 4 $
\\Thus $r = 0,1,2,3$
\\If $r = 0$, $x = 4k + 0 = 4k = 2(2k)$
\\This contradicts the statement that $x \neq 2c$, so $ r \neq 0$
\\If $r = 1$, $x = 4k + 1 = 2(2k) + 1$
\\By the Division Algorithm, this remainder is unique, so it is not possible to write this as $x = 2c$, thus $r = 1$ is possible.
\\If $r = 2$, $x = 4k + 2 = 2(2k + 1)$
\\This contradicts the statement that $x \neq 2c$, so $ r \neq 2$
\\If $r = 3$, $x = 4k + 3 = 2(2k + 1) + 1$
\\By the Division Algorithm, this remainder is unique, so it is not possible to write this as $x = 2c$, thus $r = 3$ is possible.
\\Therefore for odd $x$, $x = 4k + 1$ or $x = 4k + 3$.
\end{proof}
\end{addmargin}

%----------------------------------------------------------------------------------------
\newpage
\section*{Problem 11}

%\lipsum[2] % Dummy text
\textbf{Problem statement}:
Prove the following version of the Division Algorithm, which holds for both positive and negative Divisors.
\\ \begin{addmargin}[1em]{0em}
\textit{Extended Division Algorithm: Let a and b be integers with b $\neq$ 0.  Then there exists unique integers q and r such that a = bq + r and 0 $\leq$ r $<$ $|b|$.}
\end{addmargin} \hfill \break
[\textit{Hint:} Apply Theorem 1.1 when $a$ is divided by $|b|$.  Then consider two cases ($b$ $>$ 0 and $b$ $<$ 0).]
\\


Solution: 
\begin{addmargin}[1em]{0em}
\begin{proof} \hfill \break
Let $a$ and $b$ be integers with $b \neq 0$.
\\If $b > 0$, then the Division Algorithm applies.
\\Thus there exist a unique $r$, $q$ such that $a = bq + r$ and $0 \leq r < b$
\\Because $b > 0$, $b = |b|$.
\\Therefore, $a = |b|q + r$ and $0 \leq r < |b|$
\\Then $a = (-|b|)(-q) + r$ and $0 \leq r < |b|$
\\However, $-|b| = -b$ when $b>0$ and thus $-b < 0$
\\Redefine $b$ as $-b$ by looking at negative values of $b$.
\\Thus, $a = b(-q) + r$ and $0 \leq r < |b|$.
\\Noting that $-q$ is still an integer, it can be relabeled as $q$.
\\Thus for negative values of $b$, $a = bq + r$ and $0 \leq r < |b|$.
\\However, noting that $q$ and $r$ are unique as defined in the Division Algorithm, so must $-q$ and $r$ be unique and thus for when $b < 0$, the chosen $b$ and $r$ must still be unique.
\\Therefore, for $b>0$ or $b<0$ and thus for all $b \neq 0$, $a = bq + r$ and $0 \leq r < |b|$ with a unique $q$ and $r$.
\end{proof}
\end{addmargin}

%----------------------------------------------------------------------------------------

\end{document}
