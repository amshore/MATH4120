%----------------------------------------------------------------------------------------
%   PACKAGES AND OTHER DOCUMENT CONFIGURATIONS
%----------------------------------------------------------------------------------------

\documentclass{article} % paper and 12pt font size

\usepackage{scrextend, tikz, amssymb}
\usepackage{amsmath,amsfonts,amsthm} % Math packages
\setlength\parindent{0pt} % Removes all indentation from paragraphs - comment this line for an assignment with lots of text

%----------------------------------------------------------------------------------------
%   TITLE SECTION
%----------------------------------------------------------------------------------------

\newcommand{\horrule}[1]{\rule{\linewidth}{#1}} % Create horizontal rule command with 1 argument of height

\title{ 
\normalfont \normalsize 
\textsc{MATH 4120-001 --- Abstract Algebra} \\
\horrule{0.5pt} \\[0cm] % Thin top horizontal rule
\huge Section 6.2:  4, 10a*, 12, 20*, 29, 30\\ % The assignment title
\horrule{2pt} \\[0cm] % Thick bottom horizontal rule
}
\author{Andrew Shore} % Your name
\date{\normalsize\today} % Today's date or a custom date
\begin{document}

\maketitle % Print the title

%----------------------------------------------------------------------------------------
%   PROBLEM 1
%----------------------------------------------------------------------------------------
\section*{Problem 4}


\textbf{Problem statement}: Let $[a_n]$ denote the congruence class of the integer $a$ modulo $n$.
\\
\textbf{(a):} Show that the map $f: \mathbb{Z}_{12} \rightarrow \mathbb{Z}_4$ that sends $[a]_{12}$ to $[a]_4$ is a well-defined, surjective homomorphism.
\\
\underline{Solution}: 
\begin{addmargin}[1em]{0em}
\begin{proof}
Suppose $f: \mathbb{Z}_{12} \rightarrow \mathbb{Z}_4$ is defined as $f([a]_{12}) = [a]_4$
\\ \textbf{Well-Defined}
\begin{addmargin}[1em]{0em}
Suppose for $[a]_{12}, [b]_{12} \in \mathbb{Z}_{12}$ where $[a]_{12} = [b]_{12}$ that we have $f([a]_{12}) = [a]_{4}, f([b]_{12}) = [b]_4$
\\Then $[a]_{12} - [b]_{12} = 0 \implies [a - b]_{12} = 0 \implies a - b = 12k \implies a - b = 4(3k) \implies 4|(a-b) \implies [a]_4 = [b]_4$
\\Therefore, $f$ is well defined
\end{addmargin}
\textbf{Surjective}
\begin{addmargin}[1em]{0em}
Suppose $[a]_4 \in \mathbb{Z}_4$
\\Then $a = 4k + r$ with $r = 0, 1, 2, 3$
\\Note that $a - r = 4k \implies [a]_4 = [r]_4$
\\So take $f([r]_12) = [r]_4 = [a]_4$ because $r = 0, 1, 2, 3$
\\Thus, $f$ is surjective.
\end{addmargin}
\textbf{Homomorphism}
\begin{addmargin}[1em]{0em}
Suppose $[a]_{12}, [b]_{12} \in \mathbb{Z}_{12}$
\\Then $f([a]_{12}[b]_{12}) = f([ab]_{12}) = [ab]_{4} = [a]_4[b]_4 = f([a]_{12})f([b]_{12})$
\\And $f([a]_{12} + [b]_{12}) = f([a + b]_{12}) = [a + b]_4 = [a]_4 + [b]_4 = f([a]_{12}) + f([b]_{12})$
\\Thus, $f$ is a homomorphism.
\end{addmargin}
Therefore, $f$ is a well-defined surjective homomorphism.
\end{proof}
\end{addmargin}


\textbf{(b):} Find the kernel of $f$.
\\
\underline{Solution}: 
\begin{addmargin}[1em]{0em}
For the kernel, we want $[a]_{12}$ such that $[a]_4 = 0 \implies a = 4k \implies k = 0, 1, 2$
\\Therefore $Ker(f) = \{[0], [4], [8]\}$
\end{addmargin}    

\newpage
%----------------------------------------------------------------------------------------

\section*{Problem 10a*}

\textbf{Problem statement}: Let $f: R \rightarrow S$ be a surjective homomorphism of rings and let $I$ be an ideal in $R$.  Prove that $f(I)$ is an ideal in $S$, where $f(I) = \{s \in S|s = f(a)$ for some $a \in I\}$
\\


\underline{Solution}: 
\begin{addmargin}[1em]{0em}
\textbf{(a)}:
\begin{addmargin}[1em]{0em}
\begin{proof}
Suppose $f: R \rightarrow S$ is a surjective homomorphism and $I \unlhd R$
\\Take $a, b \in I$ and $c \in R$
\\Then $a-b \in I$ and $f(a - b) = f(a) - f(b) \in S$
\\In addition, $ac, cb \in I$ and $f(ac) = f(a)f(c) \in S$ and $f(cb) = f(c)f(b) \in S$
\\Therefore, $f(I) \unlhd S$
\end{proof}
\end{addmargin}
\end{addmargin}

\newpage
%----------------------------------------------------------------------------------------

\section*{Problem 12}


\textbf{Problem statement}: Let $I$ be an ideal in a noncommutative ring $R$ such that $ab - ba \in I$ for all $a, b \in R$.  Prove that $R/I$ is commutative.
\\

Solution: 
\begin{addmargin}[1em]{0em}
\begin{proof}
Suppose that $R$ is a noncommutative ring and $I \unlhd R$ such that $I = \{ab - ba|a,b \in R\}$
\\ Suppose $x+I,y+I \in R/I$
\\ Then $(x+I)(y+I) - (y+I)(x+I) = (xy - yx) + I$
\\ However, because $x,y \in R, xy - yx \in I$
\\ Thus, $(xy - yx) + I = 0 \implies (x+I)(y+I) - (y+I)(x+I) = 0 \implies (x+I)(y+I) = (y+I)(x+I)$
\\ Thus, because $x,y$ are general, $R/I$ is commutative.
\end{proof}
\end{addmargin}

\newpage
%----------------------------------------------------------------------------------------

\section*{Problem 20*}


\textbf{Problem statement}: Let $f: R \rightarrow S$ be a homomorphism of rings with kernel $K$.  Let $I$ be an ideal in $R$ such that $I \subseteq K$.  Show that $\bar{f}: R/I \rightarrow S$ given by $\bar{f}(r + I) = f(r)$ is a well defined homomorphism.
\\

Solution: 
\begin{addmargin}[1em]{0em}
\begin{proof}
Suppose $f:R \rightarrow S$ is a homomorphism with kernel $K$.
\\Let $I \unlhd R$ with $I \subseteq K$
\\Take $\bar{f}:R/I \rightarrow S$ be defined as $\bar{f}(r + I) = f(r)$
\\ \textbf{Well-Defined}
\begin{addmargin}[1em]{0em}
Suppose we have $a + I, b + I \in R/I$ so that $\bar{f}(a+I) = f(a)$ and $\bar{f}(b + I) = f(b)$ and $a + I = b + I$
\\Then $f(a) - f(b) = f(a - b) = \bar{f}((a-b) + I) = \bar{f}((a+I) - (b + I)) = \bar{f}(0) = \bar{f}(0 + I) = f(0)$
\\Because $f$ is a homomorphism, $f(0_R) = 0_S$, so then $f(a) - f(b) = 0 \implies f(a) = f(b)$
\\Therefore, $f$ is well defined.        
\end{addmargin}
\textbf{Homomorphism}
\begin{addmargin}[1em]{0em}
Suppose $a + I, b + I \in R/I$
\\Then $\bar{f}((a+I)(b+I)) = \bar{f}((ab)+I) = f(ab) = f(a)  f(b) = \bar{f}(a+I)\bar{f}(b+I)$
\\And $\bar{f}((a+I) + (b+I)) = \bar{f}((a + b) + I) = f(a + b) = f(a) + f(b) = \bar{f}(a+I) + \bar{f}(b+I)$
\\Therefore $f$ is a homomorphism.
\end{addmargin}
\end{proof}
\end{addmargin}

\newpage
%----------------------------------------------------------------------------------------

\section*{Problem 29}


\textbf{Problem statement}: Let $S$ and $I$ be defined by $S = \left( \begin{matrix} a & b \\ 0 & c \end{matrix} \right)$ and $I = \left( \begin{matrix} 0 & b \\ 0 & 0 \end{matrix} \right)$.  Prove that $S/I \cong \mathbb{R} \times \mathbb{R}$
\\

Solution: 
\begin{addmargin}[1em]{0em}
\begin{proof}
Suppose $S = \left( \begin{matrix} a & b \\ 0 & c \end{matrix} \right) \subseteq \mathbb{M}_2(\mathbb{R})$ and $I = \left( \begin{matrix} 0 & b \\ 0 & 0 \end{matrix} \right)$
\\Let $f: S \rightarrow \mathbb{R} \times \mathbb{R}$ such that $f( \left( \begin{matrix} a & b \\ 0 & c \end{matrix} \right) ) = (a,c)$
\\ \textbf{Well-Defined}
\begin{addmargin}[1em]{0em}
Note if $\left( \begin{matrix} a & b \\ 0 & c \end{matrix} \right) = \left( \begin{matrix} e & d \\ 0 & g \end{matrix} \right)$, then $(a,c) - (e,g) = (a-e,c-g) = f( \left( \begin{matrix} a-e & b - d \\ 0 & e - g \end{matrix} \right) ) = f(\left( \begin{matrix} a & b \\ 0 & c \end{matrix} \right) - \left( \begin{matrix} d & e \\ 0 & g \end{matrix}\right)) = f(0) = 0 \implies (a,c) = (e,g)$
\\Thus $f$ is well-defined.
\end{addmargin}
\textbf{Surjective}
\begin{addmargin}[1em]{0em}
Suppose we have $(a,c) \in \mathbb{R} \times \mathbb{R}$
\\Then $f( \left( \begin{matrix} a & b \\ 0 & c \end{matrix} \right) ) = (a,c)$
\\Therefore $f$ is surjective
\end{addmargin}
\textbf{Homomorphism}
\begin{addmargin}[1em]{0em}
Suppose $\left( \begin{matrix} a & b \\ 0 & c \end{matrix} \right), \left( \begin{matrix} e & d \\ 0 & g \end{matrix} \right) \in S$
\\Then $f(\left(\begin{matrix} a & b \\ 0 & c \end{matrix} \right) + \left( \begin{matrix} e & d \\ 0 & g\end{matrix} \right)) = f( \left( \begin{matrix} a + e & b + d \\ 0 & c + g \end{matrix} \right) ) = (a + e, c + g) = (a,c) + (e,g) = f( \left( \begin{matrix} a & b \\ 0 & c \end{matrix} \right) ) + f(\left(  \begin{matrix}  e & d \\ 0 & g\end{matrix} \right) )$
\\And $f(\left(\begin{matrix} a & b \\ 0 & c \end{matrix} \right) \left( \begin{matrix} e & d \\ 0 & g\end{matrix} \right)) = f(\left( \begin{matrix} ae & ad + bg \\ 0 & cg \end{matrix} \right) ) = (ae, cg) = (a,c)(e,g) = f( \left( \begin{matrix} a & b \\ 0 & c \end{matrix} \right) )  f(\left(  \begin{matrix}  e & d \\ 0 & g\end{matrix} \right) )$
\\Therefore, $f$ is a homomorphism.
\end{addmargin}
\textbf{Kernel}
\begin{addmargin}[1em]{0em}
Note that $Ker(f)$ is the set of all matricies that map to $(0,0)$
\\Thus because $f( \left( \begin{matrix} 0 & b \\ 0 & 0\end{matrix}  \right) ) = (0,0)$ then $Ker(f) = I$
\end{addmargin}
Thus $S/I \cong \mathbb{R} \times \mathbb{R}$
\end{proof}
\end{addmargin}

\newpage
%----------------------------------------------------------------------------------------

\section*{Problem 30}


\textbf{Problem statement}: \textbf{(The Second Isomorphism Theorem)} Let $I$ and $J$ be ideals of a ring $R$.  Then $I \cap J$ is an ideal in $I$, and $J$ is an ideal in $I + J$.  Prove that $\frac{I}{I \cap J} \cong \frac{I + J}{J}$ [\textit{Hint: } Show that $f: I \rightarrow (I + J)/J$ given by $f(a) = a + J$ is a surjectrive homomorphism with kernel $I \cap J$.]
\\

Solution: 
\begin{addmargin}[1em]{0em}
\begin{proof}
Suppose $R$ is a ring and $I,J \unlhd R$
\\Then $I \cap J \unlhd I$ and $J \unlhd I + J$
\\Suppose $f: I \rightarrow (I + J)/J$ such that $f(a) = a + J$
\\ \textbf{Surjective}
\begin{addmargin}[1em]{0em}
Suppose $i \in I, j \in J$
\\Then $f(i) = i + J = i + (j + J) = (i + j) + J \in (I + J)/J$
\\Thus, because $i, j$ are chosen arbitrarily, $f$ is surjective
\end{addmargin}
\textbf{Homomorphism}
\begin{addmargin}[1em]{0em}
Suppose $a,b \in I$
\\Then $f(a + b) = (a + b) + J = (a + J) + (b + J) = f(a) + f(b)$
\\And $f(ab) = (ab) + J = (a + J)(b + J) = f(a)f(b)$
\\These steps are valid because $I$ is an ideal and thus has closure.
\\Therefore $f$ is a homomorphism.
\end{addmargin}
\textbf{Kernel}
\begin{addmargin}[1em]{0em}
Suppose $x \in Ker (f)$
\\Then $J = f(x) = x + J \implies x \in J$
\\Now suppose $x \in I \cap J$
\\Then $f(x) = x + J = J$ because $x \in J$
\\Thus, $x \in Ker(f)$
\\Therefore $Ker(f) = I \cap J$
\end{addmargin}
Thus, by the first isomorphism theorem, $I/(I \cap J) \cong (I + J)/J$
\end{proof}
\end{addmargin}

\newpage
%----------------------------------------------------------------------------------------

\end{document}