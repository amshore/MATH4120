%----------------------------------------------------------------------------------------
%   PACKAGES AND OTHER DOCUMENT CONFIGURATIONS
%----------------------------------------------------------------------------------------

\documentclass{article} % paper and 12pt font size

\usepackage{scrextend, tikz, amssymb}
\usepackage{amsmath,amsfonts,amsthm} % Math packages
\setlength\parindent{0pt} % Removes all indentation from paragraphs - comment this line for an assignment with lots of text

%----------------------------------------------------------------------------------------
%   TITLE SECTION
%----------------------------------------------------------------------------------------

\newcommand{\horrule}[1]{\rule{\linewidth}{#1}} % Create horizontal rule command with 1 argument of height

\title{ 
\normalfont \normalsize 
\textsc{MATH 4120-001 --- Abstract Algebra} \\
\horrule{0.5pt} \\[0cm] % Thin top horizontal rule
\huge Section  5.1: 1, 4*, 7, 10*, 12\\ % The assignment title
\horrule{2pt} \\[0cm] % Thick bottom horizontal rule
}
\author{Andrew Shore} % Your name
\date{\normalsize\today} % Today's date or a custom date
\begin{document}

\maketitle % Print the title

%----------------------------------------------------------------------------------------
%   PROBLEM 1
%----------------------------------------------------------------------------------------
\section*{Problem 1}


\textbf{Problem statement}: Let $f(x), g(x), p(x) \in F[x]$, with $p(x)$ nonzero.  Determine whether $f(x) \equiv g(x) ($mod$ p(x))$.  Show your work.
\\

\textbf{(a): } $f(x) = x^5 - 2x^4 + 4x^3 + x + 1; \quad  g(x) = 3x^4 + 2x^3 - 5x^2 - 9; \quad p(x) = x^2 + 1; \quad F = \mathbb{Q}$
\begin{addmargin}[1em]{0em}
\underline{Solution}: 
\begin{addmargin}[1em]{0em}
$(x^2 + 1) | ((x^5 - 2x^4 + 4x^3 + x + 1) - (3x^4 + 2x^3 - 5x^2 - 9))$
\\$(x^2 + 1) | (x^5 - 5x^4 + 2x^3 + 5x^2 + x + 10)$
\\Thus for $u(x) \in F[x]$, $x^5 - 5x^4 + 2x^3 + 5x^2 + x + 10 = u(x)(x^2 + 1)$
\\ $x^5 - 5x^4 + 2x^3 + 5x^2 + x + 10 =(x^3 - 5x^2 + x + 10)(x^2 + 1)$
\\Thus $f(x)$ and $g(x)$ are congruent modulo $p(x)$
\end{addmargin}
\end{addmargin}    

\textbf{(b): } $f(x) = x^4 + x^2 + x + 1; \quad  g(x) = x^4 + x^3 + x^2 + 1; \quad p(x) = x^2 + x; \quad F = \mathbb{Z}_2$
\begin{addmargin}[1em]{0em}
\underline{Solution}: 
\begin{addmargin}[1em]{0em}
$(x^2 + x)|((x^4 + x^2 + x + 1) - (x^4 + x^3 + x^2 + 1))$
\\$(x^2 + x)|(-x^3 + x)$
\\$(x^2 + x)|(x^3 + x)$
\\Thus for $u(x) \in F[x]$, $x^3 + x = u(x)(x^2 + x)$
\\$x^3 + x = (x + 1)(x^2 + x)$
\\Thus $f(x)$ and $g(x)$ are congruent modulo $p(x)$
\end{addmargin}
\end{addmargin}    

\textbf{(c): } $f(x) = 3x^5 + 4x^4 + 5x^3 - 6x^2 + 5x - 7; \quad  g(x) = 2x^5 + 6x^4 + x^3 + 2x^2 + 2x -5; \quad p(x) =x^3 - x^2 + x - 1; \quad F = \mathbb{R}$
\begin{addmargin}[1em]{0em}
\underline{Solution}: 
\begin{addmargin}[1em]{0em}
$(x^3 - x^2 + x - 1) | ((3x^5 + 4x^4 + 5x^3 - 6x^2 + 5x - 7) - (2x^5 + 6x^4 + x^3 + 2x^2 + 2x -5))$
\\$(x^3 - x^2 + x - 1) | (x^5 - 2x^4 + 4x^3 - 8x^2 + 3x - 2)$
\\Thus for $u(x) \in F[x]$, $x^5 - 2x^4 + 4x^3 - 8x^2 + 3x - 2 = u(x)(x^3 - x^2 + x - 1)$
\\However, $x^5 - 2x^4 + 4x^3 - 8x^2 + 3x - 2 = (x^2 - x + 2)(x^3 - x^2 + x - 1) + (-4x^2)$
\\Thus, $f(x)$ and $g(x)$ are not congruent modulo $p(x)$
\end{addmargin}
\end{addmargin}    

\newpage
%----------------------------------------------------------------------------------------

\section*{Problem 4*}

\textbf{Problem statement}: Show that, under congruence modulo $x^3 + 2x + 1$ in $\mathbb{Z}_3[x]$, there are exactly 27 distinct congruence classes
\\


\underline{Solution}: 
\begin{addmargin}[1em]{0em}
Suppose $f(x) \in \mathbb{Z}_3[x]$ and that $p(x) = x^3 + 2x + 1$
\\Note that by corollary 5.4, all congruence classes are either disjoint or identical and that by Corollary 5.5, if $q(x) \in \mathbb{Z}_3[x]$ is such that $f(x) = q(x)p(x) + r(x)$ by the division algorithm, then $[f(x)] = [r(x)]$
\\Thus, because there are exactly $3^3 = 27$ possible different values for $[r(x)]$ such that $deg(r(x)) < 3$ or $r(x) = 0$ there are exactly $27$ congruence classes.
\\$1: 0$
\\$2: 1$
\\$3: 2$
\\$4: x$
\\$5: x+1$
\\$6: x+2$
\\$7: 2x$
\\$8: 2x + 1$
\\$9: 2x + 2$
\\$10: x^2$
\\$11: x^2 + 1$
\\$12: x^2 + 2$
\\$13: x^2 + x$
\\$14: x^2 + x+1$
\\$15: x^2 + x+2$
\\$16: x^2 + 2x$
\\$17: x^2 + 2x + 1$
\\$18: x^2 + 2x + 2$
\\$19: 2x^2$
\\$20: 2x^2 + 1$
\\$21: 2x^2 + 2$
\\$22: 2x^2 + x$
\\$23: 2x^2 + x+1$
\\$24: 2x^2 + x+2$
\\$25: 2x^2 + 2x$
\\$26: 2x^2 + 2x + 1$
\\$27: 2x^2 + 2x + 2$
\end{addmargin}

\newpage
%----------------------------------------------------------------------------------------

\section*{Problem 7}


\textbf{Problem statement}: Describe the congruence classes in $F[x]$ modulo the polynomial $x$.
\\

Solution: 
\begin{addmargin}[1em]{0em}
Each congruence class is congruent to an element $a \in F$
\\There will be as many distinct congruence classes as there are elements in $F$
\\All polynomials with the same 0$^{th}$ degree element will be in the same congruence class
\end{addmargin}

\newpage
%----------------------------------------------------------------------------------------

\section*{Problem 10*}


\textbf{Problem statement}: Prove or disprove: If $p(x)$ is irreducible in $F[x]$ and $f(x)g(x) \equiv 0_F($mod $p(x))$, then $f(x) \equiv 0_F($mod $p(x))$ or $g(x) \equiv 0_F($mod $p(x))$
\\

Solution: 
\begin{addmargin}[1em]{0em}
\begin{proof}
Suppose $F$ is a field and $p(x), f(x), g(x) \in F[x]$ where $p(x)$ is irreducible and $f(x)g(x) \equiv 0_F($mod $p(x))$
\\Then $p(x)|(f(x)g(x)-0_F) \implies p(x)|f(x)g(x)$
\\Thus, because $p(x)$ is irreducible $p(x)|f(x)$ or $p(x)|g(x)$
\\This implies $f(x) \equiv 0_F($mod $p(x))$ or $g(x) \equiv 0_F($mod $p(x))$
\end{proof}
\end{addmargin}

\newpage
%----------------------------------------------------------------------------------------

\section*{Problem 12}


\textbf{Problem statement}: If $f(x)$ is relatively prive to $p(x)$, prove that there is a polynomial $g(x) \in F[x]$ such that $f(x)g(x) \equiv 1_F($mod $p(x))$.
\\

Solution: 
\begin{addmargin}[1em]{0em}
\begin{proof}
Let $F$ be a field and $f(x), g(x), h(x),p(x) \in F[x]$ with $f(x)$ relatively prime to $p(x)$
\\Then $1_F = f(x)g(x) + h(x)p(x) \implies h(x)p(x) = 1_F - f(x)g(x) \implies p(x)|(1_F - f(x)g(x)) \implies f(x)g(x) \equiv 1_F($mod $p(x))$
\end{proof}
\end{addmargin}

%----------------------------------------------------------------------------------------

\end{document}