%----------------------------------------------------------------------------------------
%   PACKAGES AND OTHER DOCUMENT CONFIGURATIONS
%----------------------------------------------------------------------------------------

\documentclass{article} % paper and 12pt font size

\usepackage{scrextend, tikz, amssymb}
\usepackage{amsmath,amsfonts,amsthm} % Math packages
\setlength\parindent{0pt} % Removes all indentation from paragraphs - comment this line for an assignment with lots of text

%----------------------------------------------------------------------------------------
%   TITLE SECTION
%----------------------------------------------------------------------------------------

\newcommand{\horrule}[1]{\rule{\linewidth}{#1}} % Create horizontal rule command with 1 argument of height

\title{ 
\normalfont \normalsize 
\textsc{MATH 4120-001 --- Abstract Algebra} \\
\horrule{0.5pt} \\[0cm] % Thin top horizontal rule
\huge Section 6.3: 1, 5, 9, 13*, 14*, 20 \\ % The assignment title
\horrule{2pt} \\[0cm] % Thick bottom horizontal rule
}
\author{Andrew Shore} % Your name
\date{\normalsize\today} % Today's date or a custom date
\begin{document}

\maketitle % Print the title

%----------------------------------------------------------------------------------------
%   PROBLEM 1
%----------------------------------------------------------------------------------------
\section*{Problem 1}


\textbf{Problem statement}: If $n$ is a composite integer, prove that $(n)$ is not a prime ideal in $\mathbb{Z}$
\\

\underline{Solution}: 
\begin{addmargin}[1em]{0em}
\begin{proof}
Suppose $n$ is a composite integer and that $(n) \unlhd \mathbb{Z}$
\\Because $n$ is a composite integer, $n \geq 4$, so $(n) \subset \mathbb{Z}$
\\Also, for some $1 \neq a, b \in \mathbb{Z}, n = ab$
\\Thus note that $a \in (a)$ and $b \in (b)$
\\However, because $a, b < n, a, b \not\in (n)$
\\Also, $ab = n \in (n)$
\\Therefore, because $ab \in (n)$ and $a \not\in (n)$ and $b \not\in (n), (n)$ is not a prime ideal.
\end{proof}
\end{addmargin}    

\newpage
%----------------------------------------------------------------------------------------

\section*{Problem 5}

\textbf{Problem statement}: List all maximal ideals in $\mathbb{Z}_6$.  Do the same in $\mathbb{Z}_{12}$
\\

\underline{Solution}: 
\begin{addmargin}[1em]{0em}
\textbf{$\mathbb{Z}_{6}$}
\begin{addmargin}[1em]{0em}
$(2),(3)$
\end{addmargin}
\textbf{$\mathbb{Z}_{12}$}
\begin{addmargin}[1em]{0em}
$(2),(3)$
\end{addmargin}
\end{addmargin}

\newpage
%----------------------------------------------------------------------------------------

\section*{Problem 9}


\textbf{Problem statement}: Let $R$ be an integral domain in which every ideal is principal.  If $(p)$ is a nonzero prime ideal in $R$, prove that $p$ has this property: Whenever $p$ factors $p = cd$, then $c$ or $d$ is a unit in $R$.
\\

Solution: 
\begin{addmargin}[1em]{0em}
\begin{proof}
Suppose $R$ is an integral domain in which all ideals are principal.
\\Suppose $(p)$ is a nonzero prime ideal in $R$ in which $p = cd$
\\Then because $cd = p \in (p), c \in (p)$ or $d \in (p)$
\\If $c \in (p)$ then $c = pk \implies c = cdk \implies dk = 1 \implies d$ is a unit.
\\Otherwise, if $d \in (p)$ then $d = pk = cdk \implies 1 = ck \implies c$ is a unit.
\\Thus in either case, either $c$ or $d$ is a unit.
\end{proof}
\end{addmargin}

\newpage
%----------------------------------------------------------------------------------------

\section*{Problem 13*}


\textbf{Problem statement}: If $I$ is an ideal in a ring $R$, then $I \times I$ is an ideal in $R \times R$.  Prove that $(R \times R)/(I \times I)$ is isomorphic to $R/I \times R/I$. [\textit{Hint:} Show that the function $f: R \times R \rightarrow R/I \times R/I$ given by $f((a,b)) = (a + I, b + I)$ is a surjective homomorphism of rings with kernel $I \times I$]
\\

Solution: 
\begin{addmargin}[1em]{0em}
\begin{proof}
Suppose $R$ is a ring and $I \unlhd R$.
\\Note that $I \times I \unlhd R \times R$
\\In order to show that $(R \times R)/(I \times I) \cong R/I \times R/I$, I need to show that $f:R\times R \rightarrow R/I \times R/I$ defined as $f((a,b)) = (a + I, b+I)$ is a surjective homomorphism with $Ker(f) = I \times I$
\\ \textbf{Surjective}
\begin{addmargin}[1em]{0em}
Suppose we have $(a + I, b +I) \in R/I \times R/I$
\\Then by definition $(a,b) \in R \times R$ is such that $f((a,b)) = (a+I,b+I)$
\\Therefore, $f$ is surjective.
\end{addmargin}
\textbf{Homomorphism}
\begin{addmargin}[1em]{0em}
Suppose $(a,b), (c,d) \in R \times R$
\\Then $f((a,b)(c,d))= f((ac,bd)) = (ac + I, bd +I) = (a+I,b+I)(c+I,d+I)=f((a,b))f((c,d))$
\\Also, $f((a,b)+(c,d))=f((a+c,b+d)) = (a+c + I, b+d+I) = (a+I,b+I) + (c+I,d+I) = f((a,b)) + f((c,d))$
\\Therefore $f$ is a homomorphism
\end{addmargin}
\underline{$Ker(f) = I \times I$}
\begin{addmargin}[1em]{0em}
First let $(a+I, b+I) \in Ker(f)$
\\Then $(a + I, b + I) = (0,0) \implies a \in I, b \in I$
\\Thus $(a+I,b+I) \in I \times I \implies Ker(f) \subseteq I \times I$
Secondly, let $(a+I, b+I) \in I \times I$
\\Then $a + I \in I, b + I \in I \implies a \in I, b \in I$
\\Thus, $(a + I, b + I) = (0, 0) \in Ker(f)$
\\Therefore, $I \times I \subseteq Ker(f)$
\\Thus, $Ker(f) = I \times I$
\end{addmargin}
Therefore, $f$ is an isomorphism from $(R \times R)/(I \times I)$ to $R/I \times R/I$
\end{proof}
\end{addmargin}

\newpage
%----------------------------------------------------------------------------------------

\section*{Problem 14*}


\textbf{Problem statement}: If $P$ is a prime ideal in a commutative ring $R$, is the ideal $P \times P$ a prime ideal in $R \times R$?
\\

Solution: 
\begin{addmargin}[1em]{0em}
\begin{proof}
Suppose $R$ is a commutative ring and that $P$ is a prime ideal in $R$
\\Then first note that for $(a_1,a_2),(b_1,b_2) \in P \times P, (r_1,r_2) \in R \times R$:
\\$(a_1,a_2)-(b_1,b_2) = (a_1 - b_1, a_2-b_2) \in P \times P$ because $a_1 - b_1 \in P, a_2 - b_2 \in P$ because $P$ is an ideal.
\\And: $(a_1,a_2)(r_1,r_2) = (a_1r_1, a_2r_2) \in P \times P$ because $a_1r_1 \in P, a_2r_2 \in P$ because $P$ is an ideal.
\\Thus, $I \times I \unlhd R \times R$
\\Suppose $(a_1,a_2)(b_1,b_2) \in P \times P$
\\Then $(a_1b_1, a_2b_2) \in P \times P$
\\Because $a_1b_1 \in P$, then $a_1 \in P$ or $b_1 \in P$ and because $a_2b_2 \in P$, then $a_2 \in P$ or $b_2 \in P$.
\\If $a_1 \in P$ and $b_2 \in P$ but $a_2 \not\in P$ and $b_1 \not\in P$, then $(a_1,a_2),(b_1,b_2) \not\in P \times P$
\\Therefore, $P \times P$ is not a prime ideal in $R \times R$
\end{proof}
\end{addmargin}

\newpage
%----------------------------------------------------------------------------------------

\section*{Problem 20}


\textbf{Problem statement}: Find an ideal in $\mathbb{Z} \times \mathbb{Z}$ that is prime but not maximal.
\\

Solution: 
\begin{addmargin}[1em]{0em}
Let $I = \{(n,0)|n \in \mathbb{Z}$ and is prime$\}$
\\Thus, $I$ is a prime ideal in $\mathbb{Z} \times \mathbb{Z}$
\\Suppose $f: (\mathbb{Z} \times \mathbb{Z})/I \rightarrow \mathbb{Z}$ is defined such that $f((m,n)) = n$
\\This is obviously a surjective homomorphism with kernel $I$ because:
\\ $\forall n \in \mathbb{Z}, f((m,n)) = n$
\\ $f((n,a) + (m,b)) = f((n + m,a+b)) = a+b = f((n,a)) + f((m,b))$
\\ $f((n,a)(m,b)) = f((nm,ab)) = ab = f((n,a))f((m,b))$
\\ $f(m,0) = 0 \implies Ker(f) = I$
\\Thus, because $\mathbb{Z} \times \mathbb{Z}$ is a commutative ring with identity, but is isomorphic to an integral domain, but not a field.
\\Thus, $I$ is prime, but not maximal.
\end{addmargin}

\newpage
%----------------------------------------------------------------------------------------

\end{document}