%----------------------------------------------------------------------------------------
%   PACKAGES AND OTHER DOCUMENT CONFIGURATIONS
%----------------------------------------------------------------------------------------

\documentclass{article} % paper and 12pt font size

\usepackage{scrextend, tikz, amssymb}
\usepackage{amsmath,amsfonts,amsthm} % Math packages
\setlength\parindent{0pt} % Removes all indentation from paragraphs - comment this line for an assignment with lots of text

%----------------------------------------------------------------------------------------
%   TITLE SECTION
%----------------------------------------------------------------------------------------

\newcommand{\horrule}[1]{\rule{\linewidth}{#1}} % Create horizontal rule command with 1 argument of height

\title{ 
\normalfont \normalsize 
\textsc{MATH 4120-001 --- Abstract Algebra} \\
\horrule{0.5pt} \\[0cm] % Thin top horizontal rule
\huge Section 6.3: 1, 5, 9, 13*, 14*, 20 \\ % The assignment title
\horrule{2pt} \\[0cm] % Thick bottom horizontal rule
}
\author{Andrew Shore} % Your name
\date{\normalsize\today} % Today's date or a custom date
\begin{document}

\maketitle % Print the title

%----------------------------------------------------------------------------------------
%   PROBLEM 1
%----------------------------------------------------------------------------------------
\section*{Problem 1}


\textbf{Problem statement}: If $n$ is a composite integer, prove that $(n)$ is not a prime ideal in $\mathbb{Z}$
\\

\underline{Solution}: 
\begin{addmargin}[1em]{0em}

\end{addmargin}    

\newpage
%----------------------------------------------------------------------------------------

\section*{Problem 5}

\textbf{Problem statement}: List all maximal ideals in $\mathbb{Z}_6$.  Do the same in $\mathbb{Z}_{12}$
\\

\underline{Solution}: 
\begin{addmargin}[1em]{0em}
\textbf{(a)}:
\begin{addmargin}[1em]{0em}
\end{addmargin}
\end{addmargin}

\newpage
%----------------------------------------------------------------------------------------

\section*{Problem 9}


\textbf{Problem statement}: Let $R$ be an integral domain in which every ideal is principal.  If $(p)$ is a nonzero prime ideal in $R$, prove that $p$ has this property: Whenever $p$ factors $p = cd$, then $c$ or $d$ is a unit in $R$.
\\

Solution: 
\begin{addmargin}[1em]{0em}
\begin{proof}

\end{proof}
\end{addmargin}

\newpage
%----------------------------------------------------------------------------------------

\section*{Problem 13*}


\textbf{Problem statement}: If $I$ is an ideal in a ring $R$, then $I \times I$ is an ideal in $R \times R$.  Prove that $(R \times R)/(I \times I)$ is isomorphic to $R/I \times R/I$. [\textit{Hint:} Show that the function $f: R \times R \rightarrow R/I \times R/I$ given by $f((a,b)) = (a + I, b + I)$ is a surjective homomorphism of rings with kernel $I \times I$]
\\

Solution: 
\begin{addmargin}[1em]{0em}
\begin{proof}

\end{proof}
\end{addmargin}

\newpage
%----------------------------------------------------------------------------------------

\section*{Problem 14*}


\textbf{Problem statement}: If $P$ is a prime ideal in a commutative ring $R$, is the ideal $P \times P$ a prime ideal in $R \times R$?
\\

Solution: 
\begin{addmargin}[1em]{0em}
\begin{proof}

\end{proof}
\end{addmargin}

\newpage
%----------------------------------------------------------------------------------------

\section*{Problem 20}


\textbf{Problem statement}: Find an ideal in $\mathbb{Z} \times \mathbb{Z}$ that is prime but not maximal.
\\

Solution: 
\begin{addmargin}[1em]{0em}
\begin{proof}

\end{proof}
\end{addmargin}

\newpage
%----------------------------------------------------------------------------------------

\end{document}