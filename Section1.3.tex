%----------------------------------------------------------------------------------------
%   PACKAGES AND OTHER DOCUMENT CONFIGURATIONS
%----------------------------------------------------------------------------------------

\documentclass{article} % paper and 12pt font size

\usepackage{scrextend, tikz, amssymb}
\usepackage{amsmath,amsfonts,amsthm} % Math packages
\setlength\parindent{0pt} % Removes all indentation from paragraphs - comment this line for an assignment with lots of text
\reversemarginpar

%----------------------------------------------------------------------------------------
%   TITLE SECTION
%----------------------------------------------------------------------------------------

\newcommand{\horrule}[1]{\rule{\linewidth}{#1}} % Create horizontal rule command with 1 argument of height

\title{ 
\normalfont \normalsize 
\textsc{MATH 4120-001 --- Abstract Algebra} \\
\horrule{0.5pt} \\[0cm] % Thin top horizontal rule
\huge Section 1.3: 1d, 3, 5a, 10*, 15*, 19, 25, 30*  \\ % The assignment title
\horrule{2pt} \\[0cm] % Thick bottom horizontal rule
}
\author{Andrew Shore} % Your name
\date{\normalsize\today} % Today's date or a custom date
\begin{document}

\maketitle % Print the title

%----------------------------------------------------------------------------------------
%   PROBLEM 1
%----------------------------------------------------------------------------------------
\section*{Problem 1d}


\textbf{Problem statement}: Express 2,042,040 as a product of primes.
\\

\underline{Solution}: 
\begin{addmargin}[1em]{0em}
Divisible by 2: 2,042,040 = 2 * 1,021,020
\\Divisible by 2: 1,021,020 = 2 * 510,510
\\Divisible by 2: 501,510 = 2 * 255,255
\\Divisible by 3: 255,255 = 3 * 85,085
\\Divisible by 5: 85,085 = 5 * 17,017
\\Divisible by 7: 17,017 = 7 * 2,431
\\Divisible by 11: 2,431 = 11 * 221
\\Divisible by 13: 221 = 13 * 17
\\Divisible by 1: 17 = 1 * 17 (Prime)
\\ \hfill \break
2,042,040 = $2^3*3*5*7*11*13*17$
\end{addmargin}    

\newpage
%----------------------------------------------------------------------------------------

\section*{Problem 3}

\textbf{Problem statement}: Which of the following numbers are prime
\\

\begin{addmargin}[1em]{0em}

%part a
\textbf{(a)}: 701
\\ \hfill \break
\underline{Solution}: 
\begin{addmargin}[1em]{0em}
701 is a prime number
\end{addmargin}
\hfill \break

%part b
\textbf{(b)}: 1009
\\ \hfill \break
\underline{Solution}: 
\begin{addmargin}[1em]{0em}
1009 is a prime number
\end{addmargin}
\hfill \break

%part c
\textbf{(c)}: 1949
\\ \hfill \break
\underline{Solution}: 
\begin{addmargin}[1em]{0em}
1949 is a prime number
\end{addmargin}
\hfill \break

%part d
\textbf{(d)}: 1951
\\ \hfill \break
\underline{Solution}: 
\begin{addmargin}[1em]{0em}
1951 is a prime number
\end{addmargin}
\end{addmargin}

\newpage
%----------------------------------------------------------------------------------------

\section*{Problem 5a}


\textbf{Problem statement}: List all the positive integer divisors of $3^s5^t$, where $s, t \in \mathbb{Z}$ and $s, t > 0$.
\\

\underline{Solution}: 
\begin{addmargin}[1em]{0em}
Let S be the set of all positive integer divisors of $3^s5^t$ for $s, t \in \mathbb{Z}$ and $s, t > 0$
\\Say we call the postive integer $r$ and we have an integer $a|r$.
\\Then $a|3^s5^t \implies a|3^s$ or $a|5^t$ because 3 and 5 are prime.
\\Thus for $i, j \in \mathbb{Z}$ where $0 \leq i \leq s, 0 \leq j \leq t$, $a = 3^{i}$ or $a = 5^{j}$
\\Therefore in general, $a = 3^i5^j$ where $i, j \in \mathbb{Z}$ and $0 \leq i \leq s, 0 \leq j \leq t$
\\Thus, $S = \{3^i5^j | s, t, i, j \in \mathbb{Z}, 0 \leq i \leq s > 0, 0 \leq j \leq t > 0\}$
\end{addmargin}

\newpage
%----------------------------------------------------------------------------------------

\section*{Problem 10*}


\textbf{Problem statement}: Let $p$ be an integer other than $0, \pm 1$.  Prove that $p$ is a prime if and only if for each $a \in \mathbb{Z}$ either $(a,p) = 1$ or $p|a$.
\\

\underline{Solution}: 
\begin{addmargin}[1em]{0em}
\begin{proof}
Let $p \in \mathbb{Z}$ and $ p \notin \{0, \pm 1\}$
\\ \marginpar{$\Rightarrow$}
Let $p$ be a prime number and $a \in \mathbb{Z}$.
\\Then by the definition of a prime number, the only divisors of $p$ are $\pm 1$ and $\pm p$.
\\Thus if $a \neq \pm pk$ for $k \in \mathbb{Z}$, then $a$ and $p$ do not share any factors and thus $(a,p) = 1$.
\\In the other case, if $a = \pm pk$ for $k \in \mathbb{Z}$, then by definition of divisibility, $p|a$.
\\ \marginpar{$\Leftarrow$}
For the sake of contradiction, let $a \in \mathbb{Z}$ such that $(a,p) = 1$ or $p|a$ and let $p$ not be prime.
\\Since all values of $a$ must be valid, let $a$ be a positive divisor of $p$ (the negative divisors of $p$ are equal to the negative of the positive divisors) which is not in the set $\{\pm 1, \pm p\}$.
\\Then because $a$ is a divisor of $p$, $(a, p) \neq 1$.
\\In addition, because $a$ is a divisor of $p$, $a < p$, which contradicts the idea that $p|a$ as this implies that $a \geq p$.
\\Thus this is a contradiction and the premise must be false, which implies that $p$ is a prime number.
\end{proof}
\end{addmargin}

\newpage
%----------------------------------------------------------------------------------------

\section*{Problem 15*}


\textbf{Problem statement}: If $p$ is prime and $p|a^n$, is it true that $p^n|a^n$?  Justify your answer. [\textit{Hint:} Corollary 1.6]
\\

\underline{Solution}: 
\begin{addmargin}[1em]{0em}
Yes, this is true.
\\First observe that $a$ can be written as a product of prime factors $q_1q_2 ... q_k$.
\\Then $a^n = q_1^nq_2^n ... q_k^n$.
\\Thus, because $p|a^n$, $p|q_1^nq_2^n ... q_k^n$.
\\So by Corollary 1.6, $p|q_i^n$ which then implies $p|q_i$.
\\Redefine the $q_i$ to be $q_1$ and reorder the terms.
\\Note that this implies that $q_1 = pk$ for $k \in \mathbb{Z}$
\\Raising both sides to the n$^{th}$ power, we get $q_1^n = p^nk^n \implies p^n|q_1^n$
\\Finally using the fact that $x|y \implies x|yz$, we can rewrite this as $p^n|q_1^nq_2^n ... q_k^n$
\end{addmargin}

\newpage
%----------------------------------------------------------------------------------------

\section*{Problem 19}


\textbf{Problem statement}: Suppose that $a = p_1^{r_1}p_2^{r_2} \ldots p_k^{r_k}$ and $b = p_1^{s_1}p_2^{s_2} \ldots p_k^{s_k}$, where $p_1,p_2,\ldots,p_k$ are distinct positive primes and each $r_i,s_i \geq 0$.  Prove that $a|b$ if and only if $r_i \leq s_i$ for every $i$.
\\

\underline{Solution}: 
\begin{addmargin}[1em]{0em}
\begin{proof}
Let $a = p_1^{r_1}p_2^{r_2} \ldots p_k^{r_k}$ and $b = p_1^{s_1}p_2^{s_2} \ldots p_k^{s_k}$, where $p_1,p_2,\ldots,p_k$ are distinct positive primes and each $r_i,s_i \geq 0$.
\\ \marginpar{$\Rightarrow$}
Suppose that $a|b$.
\\Then $b = ta$, where $t \in \mathbb{Z}$
\\Thus, $p_1^{r_1}p_2^{r_2} \ldots p_k^{r_k} = tp_1^{s_1}p_2^{s_2} \ldots p_k^{s_k} \implies 1 = tp_1^{s_1 - r_1}p_2^{s_2-r_2} \ldots p_k^{s_k-r_k}$
\\Note that because for $p_i, p_j$ do not share any divisors other than $\pm 1$ if $i \neq j$
\\Thus, unless $i = j$, $\frac{p_i}{p_j}$ will not be an integer as there will be remaining integers in the denominator.
\\This implies, that if for any of the $p_i^{s_i - r_i}$, if $s_i - r_i < 0$, then the resulting value will not be an integer.
\\Thus, all of the $s_i - r_i \geq 0 \implies s_i \geq r_i$ (or equivalents, $r_i \leq s_i$.
\\ \marginpar{$\Leftarrow$}
Suppose that for every $i$, $r_i \leq s_i$.
\\Then, $b = p_1^{s_1}p_2^{s_2} \ldots p_k^{s_k} = p_1^{s_1 - r_1}p_2^{s_2 - r_2} \ldots p_k^{s_k - r_k} (p_1^{r_1}p_2^{r_2} \ldots p_k^{r_k})$
\\Because $s_i \geq r_i$ for all $i$, $s_i - r_i \geq 0 \implies p_i^{s_i - r_i} \in \mathbb{Z}$.
\\Thus $b = ka$ because $p_1^{s_1 - r_1}p_2^{s_2 - r_2} \ldots p_k^{s_k - r_k}$ is an integer and $a = p_1^{r_1}p_2^{r_2} \ldots p_k^{r_k}$.
\\Therefore, by definition, $a|b$.
\end{proof}
\end{addmargin}

\newpage
%----------------------------------------------------------------------------------------

\section*{Problem 25}


\textbf{Problem statement}: Let $p$ be prime and $1 \leq k < p$.  Prove that $p$ divides the binomial coefficient $\binom{p}{k}$.  [Recall that $\binom{p}{k} = \frac{p!}{k!(p-k)!}$].
\\

\underline{Solution}: 
\begin{addmargin}[1em]{0em}
\begin{proof}
Let $p$ be prime and $1 \leq k < p$.
\\Note that $\binom{p}{k} = \frac{p!}{k!(p-k)!}$.
\\Then $p! = \binom{p}{k}k!(p-k)!$
\\On the left hand side, it is clear that $p|p!$ using Corollary 1.6 and that $p! = p(p-1) \ldots (1)$
\\Thus, $p|\binom{p}{k}k!(p-k)!$
\\However the smallest divisor of $p > 1$ is $p$ and because all positive divisors of a number are less than or equal to the number.
\\Thus $p \nmid k!$ and because $(p-k) < p$ because $1 \leq k < p$, $p \nmid (p-k)!$
\\Therefore, the only term that $p$ can divide is $\binom{p}{k}$
\\So by Corollary 1.6, $p|\binom{p}{k}$
\end{proof}
\end{addmargin}

\newpage
%----------------------------------------------------------------------------------------

\section*{Problem 30*}


\textbf{Problem statement (a)}:  Prove that there are no nonzero integers $a,b$ such that $a^2 = 2b^2$.  [\textit{Hint:} Use the Fundamental Theorem of Arithmetic].
\\

\underline{Solution}: 
\begin{addmargin}[1em]{0em}
\begin{proof}
For the sake of contradiction, let $a, b$ be nonzero integers such that $a^2 = 2b^2$.
\\By the fundamental theorem of arithmetic, then $a = p_1^{r_1}p_2^{r_2} \ldots p_k^{r_k} and b = q_1^{s_1}q_2^{s_2} \ldots q_l^{s_l}$ where $p_i, q_i$ are primes and $r_i, s_i \geq 0$ for all $i$.
\\By definition, $p_1^{2r_1}p_2^{2r_2} \ldots p_k^{2r_k} = 2q_1^{2s_1}q_2^{2s_2} \ldots q_l^{2s_l}$.
\\Because of the uniqueness of the Fundamental Theorem of Arithmetic, all of the elements of the left hand side must equal the elements of the right hand side up to reordering.
\\To remedy this, assert that if $i < j$, then $p_i,q_i < p_j, q_j$ and move the net negative sign to $p_1, q_1$ and reorder the terms on each side as necessary.
\\Because 2 is the smallest positive prime number, all of the $p_i, q_i$ for $i > 1$ must be greater than 2.
\\Thus because primes share no divisors, all of the $p_i^{r_i} = q_i^{s_i}$ for $i > 1$ and the equation reduces to $2^{2r_1} = 2*2^{2s_1}$ because the $p_1$ and $q_1$ would have been reduced also unless they equal 2.
\\Therefore, $2^{2r_1} = 2^{2s_1 + 1}$
\\So, for some integers $r_1, s_1$, $2r_1 = 2s_1 + 1$.
\\However, this implies $2(r_1 - s_1) = 1 \implies 2k = 1$ for some $k \in \mathbb{Z}$.
\\This is a contradiction because the left hand side is an even number, but 1 is an odd number.
\\Therefore, there must not exist $a, b$ such that $a^2 = 2b^2$.
\end{proof}
\end{addmargin}

\hfill \break 

\textbf{Problem statement (b)}:  Prove that $\sqrt{2}$ is irrational.  [\textit{Hint:} Use proof by contradiction.  Assume that $\sqrt{2} = \frac{a}{b}$ (with $a,b \in \mathbb{Z}$) and use part {a} to reach a contradiction].

\underline{Solution}: 
\begin{addmargin}[1em]{0em}
\begin{proof}
For the sake of contradiction, assume that $\sqrt{2}$ is a rational number.
\\Then by definition, $\sqrt{2} = \frac{a}{b}$ where $a,b \in \mathbb{Z}$.
\\Thus $b = \sqrt{2}a \implies b^2 = 2a^2$.
\\However, as proved in part (a), there are no integers $a, b$ for which this is true.
\\Thus this is a contradiction and $\sqrt{2}$ is not a rational number (aka irrational).
\end{proof}
\end{addmargin}

\newpage
%----------------------------------------------------------------------------------------

\end{document}