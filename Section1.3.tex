%----------------------------------------------------------------------------------------
%   PACKAGES AND OTHER DOCUMENT CONFIGURATIONS
%----------------------------------------------------------------------------------------

\documentclass{article} % paper and 12pt font size

\usepackage{scrextend, tikz, amssymb}
\usepackage{amsmath,amsfonts,amsthm} % Math packages
\setlength\parindent{0pt} % Removes all indentation from paragraphs - comment this line for an assignment with lots of text
\reversemarginpar

%----------------------------------------------------------------------------------------
%   TITLE SECTION
%----------------------------------------------------------------------------------------

\newcommand{\horrule}[1]{\rule{\linewidth}{#1}} % Create horizontal rule command with 1 argument of height

\title{ 
\normalfont \normalsize 
\textsc{MATH 4120-001 --- Abstract Algebra} \\
\horrule{0.5pt} \\[0cm] % Thin top horizontal rule
\huge Section 1.3: 1d, 3, 5a, 10*, 15*, 19, 25, 30*  \\ % The assignment title
\horrule{2pt} \\[0cm] % Thick bottom horizontal rule
}
\author{Andrew Shore} % Your name
\date{\normalsize\today} % Today's date or a custom date
\begin{document}

\maketitle % Print the title

%----------------------------------------------------------------------------------------
%   PROBLEM 1
%----------------------------------------------------------------------------------------
\section*{Problem 1d}


\textbf{Problem statement}: Express 2,042,040 as a product of primes.
\\

\underline{Solution}: 
\begin{addmargin}[1em]{0em}
Divisible by 2: 2,042,040 = 2 * 1,021,020
\\Divisible by 2: 1,021,020 = 2 * 510,510
\\Divisible by 2: 501,510 = 2 * 255,255
\\Divisible by 3: 255,255 = 3 * 85,085
\\Divisible by 5: 85,085 = 5 * 17,017
\\Divisible by 7: 17,017 = 7 * 2,431
\\Divisible by 11: 2,431 = 11 * 221
\\Divisible by 13: 221 = 13 * 17
\\Divisible by 1: 17 = 1 * 17 (Prime)
\\ \hfill \break
2,042,040 = $2^3*3*5*7*11*13*17$
\end{addmargin}    

\newpage
%----------------------------------------------------------------------------------------

\section*{Problem 3}

\textbf{Problem statement}: Which of the following numbers are prime
\\

\begin{addmargin}[1em]{0em}

%part a
\textbf{(a)}: 701
\\ \hfill \break
\underline{Solution}: 
\begin{addmargin}[1em]{0em}
701 is a prime number
\end{addmargin}
\hfill \break

%part b
\textbf{(b)}: 1009
\\ \hfill \break
\underline{Solution}: 
\begin{addmargin}[1em]{0em}
1009 is a prime number
\end{addmargin}
\hfill \break

%part c
\textbf{(c)}: 1949
\\ \hfill \break
\underline{Solution}: 
\begin{addmargin}[1em]{0em}
1949 is a prime number
\end{addmargin}
\hfill \break

%part d
\textbf{(d)}: 1951
\\ \hfill \break
\underline{Solution}: 
\begin{addmargin}[1em]{0em}
1951 is a prime number
\end{addmargin}
\end{addmargin}

\newpage
%----------------------------------------------------------------------------------------

\section*{Problem 5a}


\textbf{Problem statement}: List all the positive integer divisors of $3^s5^t$, where $s, t \in \mathbb{Z}$ and $s, t > 0$.
\\

\underline{Solution}: 
\begin{addmargin}[1em]{0em}
Let S be the set of all positive integer divisors of $3^s5^t$ for $s, t \in \mathbb{Z}$ and $s, t > 0$
\\Say we call the postive integer $r$ and we have an integer $a|r$.
\\Then $a|3^s5^t \implies a|3^s$ or $a|5^t$ because 3 and 5 are prime.
\\Thus for $i, j \in \mathbb{Z}$ where $0 \leq i \leq s, 0 \leq j \leq t$, $a = 3^{i}$ or $a = 5^{j}$
\\Therefore in general, $a = 3^i5^j$ where $i, j \in \mathbb{Z}$ and $0 \leq i \leq s, 0 \leq j \leq t$
\\Thus, $S = \{3^i5^j | s, t, i, j \in \mathbb{Z}, 0 \leq i \leq s > 0, 0 \leq j \leq t > 0\}$
\end{addmargin}

\newpage
%----------------------------------------------------------------------------------------

\section*{Problem 10*}


\textbf{Problem statement}: Let $p$ be an integer other than $0, \pm 1$.  Prove that $p$ is a prime if and only if for each $a \in \mathbb{Z}$ either $(a,p) = 1$ or $p|a$.
\\

\underline{Solution}: 
\begin{addmargin}[1em]{0em}
\begin{proof}
Let $p \in \mathbb{Z}$ and $ p \notin \{0, \pm 1\}$
\\ \marginpar{$\Rightarrow$}
Let $p$ be a prime number and $a \in \mathbb{Z}$.
\\Then by the definition of a prime number, the only divisors of $p$ are $\pm 1$ and $\pm p$.
\\Thus if $a \neq \pm pk$ for $k \in \mathbb{Z}$, then $a$ and $p$ do not share any factors and thus $(a,p) = 1$.
\\In the other case, if $a = \pm pk$ for $k \in \mathbb{Z}$, then by definition of divisibility, $p|a$.
\\ \marginpar{$\Leftarrow$}
For the sake of contradiction, let $a \in \mathbb{Z}$ such that $(a,p) = 1$ or $p|a$ and let $p$ not be prime.
\\Since all values of $a$ must be valid, let $a$ be a positive divisor of $p$ (the negative divisors of $p$ are equal to the negative of the positive divisors) which is not in the set $\{\pm 1, \pm p\}$.
\\Then because $a$ is a divisor of $p$, $(a, p) \neq 1$.
\\In addition, because $a$ is a divisor of $p$, $a < p$, which contradicts the idea that $p|a$ as this implies that $a \geq p$.
\\Thus this is a contradiction and the premise must be false, which implies that $p$ is a prime number.
\end{proof}
\end{addmargin}

\newpage
%----------------------------------------------------------------------------------------

\section*{Problem 15*}


\textbf{Problem statement}: If $p$ is prime and $p|a^n$, is it true that $p^n|a^n$?  Justify your answer. [\textit{Hint:} Corollary 1.6]
\\

\underline{Solution}: 
\begin{addmargin}[1em]{0em}
Yes, this is true.
\\First observe that $a$ can be written as a product of prime factors $q_1q_2 ... q_k$.
\\Then $a^n = q_1^nq_2^n ... q_k^n$.
\\Thus, because $p|a^n$, $p|q_1^nq_2^n ... q_k^n$.
\\So by Corollary 1.6, $p|q_i^n$ which then implies $p|q_i$.
\\Redefine the $q_i$ to be $q_1$ and reorder the terms.
\\Note that this implies that $q_1 = pk$ for $k \in \mathbb{Z}$
\\Raising both sides to the n$^{th}$ power, we get $q_1^n = p^nk^n \implies p^n|q_1^n$
\\Finally using the fact that $x|y \implies x|yz$, we can rewrite this as $p^n|q_1^nq_2^n ... q_k^n$
\end{addmargin}

\newpage
%----------------------------------------------------------------------------------------

\section*{Problem 19}


\textbf{Problem statement}: Suppose that $a = p_1^{r_1}p_2^{r_2} \ldots p_k^{r_k}$, where $p_1,p_2,\ldots,p_k$ are distinct positive primes and each $r_i,s_i \geq 0$.  Prove that $a|b$ if and only if $r_i \leq s_i$ for every $i$.
\\

Solution: 
\begin{addmargin}[1em]{0em}
\begin{proof}

\end{proof}
\end{addmargin}

\newpage
%----------------------------------------------------------------------------------------

\section*{Problem 25}


\textbf{Problem statement}: Let $p$ be prime and $1 \leq k < p$.  Prove that $p$ divides the binomial coefficient $\binom{p}{k}$.  [Recall that $\binom{p}{k} = \frac{p!}{k!(p-k)!}$].
\\

Solution: 
\begin{addmargin}[1em]{0em}
\begin{proof}

\end{proof}
\end{addmargin}

\newpage
%----------------------------------------------------------------------------------------

\section*{Problem 30*}


\textbf{Problem statement (a)}:  Prove that there are no nonzero integers $a,b$ such that $a^2 = 2b^2$.  [\textit{Hint:} Use the Fundamental Theorem of Arithmetic].
\\

Solution: 
\begin{addmargin}[1em]{0em}
\begin{proof}

\end{proof}
\end{addmargin}

\hfill \break 

\textbf{Problem statement (b)}:  Prove that $\sqrt{2}$ is irrational.  [\textit{Hint:} Use proof by contradiction.  Assume that $\sqrt{2} = \frac{a}{b}$ (with $a,b \in \mathbb{Z}$) and use part {a} to reach a contradiction].
\\

Solution: 
\begin{addmargin}[1em]{0em}
\begin{proof}

\end{proof}
\end{addmargin}

\newpage
%----------------------------------------------------------------------------------------

\end{document}