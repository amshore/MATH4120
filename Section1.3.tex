%----------------------------------------------------------------------------------------
%   PACKAGES AND OTHER DOCUMENT CONFIGURATIONS
%----------------------------------------------------------------------------------------

\documentclass{article} % paper and 12pt font size

\usepackage{scrextend, tikz, amssymb}
\usepackage{amsmath,amsfonts,amsthm} % Math packages
\setlength\parindent{0pt} % Removes all indentation from paragraphs - comment this line for an assignment with lots of text

%----------------------------------------------------------------------------------------
%   TITLE SECTION
%----------------------------------------------------------------------------------------

\newcommand{\horrule}[1]{\rule{\linewidth}{#1}} % Create horizontal rule command with 1 argument of height

\title{ 
\normalfont \normalsize 
\textsc{MATH 4120-001 --- Abstract Algebra} \\
\horrule{0.5pt} \\[0cm] % Thin top horizontal rule
\huge Section 1.3: 1d, 3, 5a, 10*, 15*, 19, 25, 30*  \\ % The assignment title
\horrule{2pt} \\[0cm] % Thick bottom horizontal rule
}
\author{Andrew Shore} % Your name
\date{\normalsize\today} % Today's date or a custom date
\begin{document}

\maketitle % Print the title

%----------------------------------------------------------------------------------------
%   PROBLEM 1
%----------------------------------------------------------------------------------------
\section*{Problem 1d}


\textbf{Problem statement}: Express 2,042,040 as a product of primes.
\\

\underline{Solution}: 
\begin{addmargin}[1em]{0em}

\end{addmargin}    

\newpage
%----------------------------------------------------------------------------------------

\section*{Problem 3}

\textbf{Problem statement}: Which of the following numbers are prime
\\

\begin{addmargin}[1em]{0em}

%part a
\textbf{(a)}: 701
\\ \hfill \break
\underline{Solution}: 
\begin{addmargin}[1em]{0em}
\end{addmargin}

%part b
\textbf{(b)}: 1009
\\ \hfill \break
\underline{Solution}: 
\begin{addmargin}[1em]{0em}
\end{addmargin}

%part c
\textbf{(c)}: 1949
\\ \hfill \break
\underline{Solution}: 
\begin{addmargin}[1em]{0em}
\end{addmargin}

%part d
\textbf{(d)}: 1951
\\ \hfill \break
\underline{Solution}: 
\begin{addmargin}[1em]{0em}
\end{addmargin}
\end{addmargin}

\newpage
%----------------------------------------------------------------------------------------

\section*{Problem 5a}


\textbf{Problem statement}: List all the positive integer divisors of $3^s5^t$, where $s, t \in \mathbb{Z}$ and $s, t > 0$.
\\

Solution: 
\begin{addmargin}[1em]{0em}
\begin{proof}

\end{proof}
\end{addmargin}

\newpage
%----------------------------------------------------------------------------------------

\section*{Problem 10*}


\textbf{Problem statement}: Let $p$ be an integer other than $0, \pm 1$.  Prove that $p$ is a prime if and only if for each $a \in \mathbb{Z}$ either $(a,p) = 1$ or $p|a$.
\\

Solution: 
\begin{addmargin}[1em]{0em}
\begin{proof}

\end{proof}
\end{addmargin}

\newpage
%----------------------------------------------------------------------------------------

\section*{Problem 15*}


\textbf{Problem statement}: If $p$ is prime and $p|a^n$, is it true that $p^n|a^n$?  Justify your answer. [\textit{Hint:} Corollary 1.6]
\\

Solution: 
\begin{addmargin}[1em]{0em}
\begin{proof}

\end{proof}
\end{addmargin}

\newpage
%----------------------------------------------------------------------------------------

\section*{Problem 19}


\textbf{Problem statement}: Suppose that $a = p_1^{r_1}p_2^{r_2} \ldots p_k^{r_k}$, where $p_1,p_2,\ldots,p_k$ are distinct positive primes and each $r_i,s_i \geq 0$.  Prove that $a|b$ if and only if $r_i \leq s_i$ for every $i$.
\\

Solution: 
\begin{addmargin}[1em]{0em}
\begin{proof}

\end{proof}
\end{addmargin}

\newpage
%----------------------------------------------------------------------------------------

\section*{Problem 25}


\textbf{Problem statement}: Let $p$ be prime and $1 \leq k < p$.  Prove that $p$ divides the binomial coefficient $\binom{p}{k}$.  [Recall that $\binom{p}{k} = \frac{p!}{k!(p-k)!}$].
\\

Solution: 
\begin{addmargin}[1em]{0em}
\begin{proof}

\end{proof}
\end{addmargin}

\newpage
%----------------------------------------------------------------------------------------

\section*{Problem 30*}


\textbf{Problem statement (a)}:  Prove that there are no nonzero integers $a,b$ such that $a^2 = 2b^2$.  [\textit{Hint:} Use the Fundamental Theorem of Arithmetic].
\\

Solution: 
\begin{addmargin}[1em]{0em}
\begin{proof}

\end{proof}
\end{addmargin}

\hfill \break 

\textbf{Problem statement (b)}:  Prove that $\sqrt{2}$ is irrational.  [\textit{Hint:} Use proof by contradiction.  Assume that $\sqrt{2} = \frac{a}{b}$ (with $a,b \in \mathbb{Z}$) and use part {a} to reach a contradiction].
\\

Solution: 
\begin{addmargin}[1em]{0em}
\begin{proof}

\end{proof}
\end{addmargin}

\newpage
%----------------------------------------------------------------------------------------

\end{document}