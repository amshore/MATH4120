%----------------------------------------------------------------------------------------
%   PACKAGES AND OTHER DOCUMENT CONFIGURATIONS
%----------------------------------------------------------------------------------------

\documentclass{article} % paper and 12pt font size

\usepackage{scrextend, tikz, amssymb, tabularx}
\usepackage{amsmath,amsfonts,amsthm} % Math packages
\setlength\parindent{0pt} % Removes all indentation from paragraphs - comment this line for an assignment with lots of text

%----------------------------------------------------------------------------------------
%   TITLE SECTION
%----------------------------------------------------------------------------------------

\newcommand{\horrule}[1]{\rule{\linewidth}{#1}} % Create horizontal rule command with 1 argument of height

\title{ 
\normalfont \normalsize 
\textsc{MATH 4120-001 --- Abstract Algebra} \\
\horrule{0.5pt} \\[0cm] % Thin top horizontal rule
\huge Section 7.1: 1, 4, 9*, 16, 21, 30 \\ % The assignment title
\horrule{2pt} \\[0cm] % Thick bottom horizontal rule
}
\author{Andrew Shore} % Your name
\date{\normalsize\today} % Today's date or a custom date
\begin{document}

\maketitle % Print the title

%----------------------------------------------------------------------------------------
%   PROBLEM 1
%----------------------------------------------------------------------------------------
\section*{Problem 1}


\textbf{Problem statement}: Find the inverse of each permutation in $S_3$
\\

\underline{Solution}: 
\begin{addmargin}[1em]{0em}
\begin{tabular}{c@\quad |@\quad c}
Permutation & Inverse \\ \hline \rule{0pt}{5ex}
$\left( \begin{matrix} 1 & 2 & 3 \\ 1 & 2 & 3 \end{matrix} \right)$ & $\left( \begin{matrix}1 & 2 & 3 \\ 1 & 2 & 3 \end{matrix} \right)$ \\ \rule{0pt}{5ex}
$\left( \begin{matrix} 1 & 2 & 3 \\ 1 & 3 & 2 \end{matrix} \right)$ & $\left( \begin{matrix} 1 & 2 & 3 \\ 1 & 3 & 2 \end{matrix} \right)$ \\ \rule{0pt}{5ex}
$\left( \begin{matrix} 1 & 2 & 3 \\ 2 & 1 & 3 \end{matrix} \right)$ & $\left( \begin{matrix} 1 & 2 & 3 \\ 2 & 1 & 3 \end{matrix} \right)$ \\ \rule{0pt}{5ex}
$\left( \begin{matrix} 1 & 2 & 3 \\ 2 & 3 & 1 \end{matrix} \right)$ & $\left( \begin{matrix} 1 & 2 & 3 \\ 3 & 1 & 2 \end{matrix} \right)$ \\ \rule{0pt}{5ex}
$\left( \begin{matrix} 1 & 2 & 3 \\ 3 & 1 & 2 \end{matrix} \right)$ & $\left( \begin{matrix} 1 & 2 & 3 \\ 2 & 3 & 1 \end{matrix} \right)$ \\ \rule{0pt}{5ex}
$\left( \begin{matrix} 1 & 2 & 3 \\ 3 & 2 & 1 \end{matrix} \right)$ & $\left( \begin{matrix} 1 & 2 & 3 \\ 3 & 2 & 1 \end{matrix} \right)$ 
\end{tabular}
\end{addmargin}    

\newpage
%----------------------------------------------------------------------------------------

\section*{Problem 4}

\textbf{Problem statement}: Determine whether the set $G$ is a group under the operation *
\\

\textbf{(a): } $G = \{2,4,6,8\}$ in $\mathbb{Z}_{10}; a*b = ab$
\begin{addmargin}[1em]{0em}
\underline{Solution}: 
\begin{addmargin}[1em]{0em}

\end{addmargin}
\end{addmargin}
\textbf{(b): } $G = \mathbb{Z}; a * b = a - b$
\begin{addmargin}[1em]{0em}
\underline{Solution}: 
\begin{addmargin}[1em]{0em}

\end{addmargin}
\end{addmargin}
\textbf{(c): } $G = \{n \in \mathbb{Z}| n$ is odd$\}; a * b = a + b$
\begin{addmargin}[1em]{0em}
\underline{Solution}: 
\begin{addmargin}[1em]{0em}

\end{addmargin}
\end{addmargin}
\textbf{(d): } $G = \{2^x| x \in \mathbb{Q}\}; a * b = ab$
\begin{addmargin}[1em]{0em}
\underline{Solution}: 
\begin{addmargin}[1em]{0em}

\end{addmargin}
\end{addmargin}

\newpage
%----------------------------------------------------------------------------------------

\section*{Problem 9*}


\textbf{Problem statement}: Write out the operation table for the group $D_3$ consisting of counterclockwise rotations $(r_0, r_1, r_2)$ and reflections about verticies $(s, t, u)$ defined as:
\[r_0:  \left( \begin{matrix} 1 & 2 & 3 \\ 1 & 2 & 3 \end{matrix} \right), 
  r_1:  \left( \begin{matrix} 1 & 2 & 3 \\ 2 & 3 & 1 \end{matrix} \right), 
  r_2:  \left( \begin{matrix} 1 & 2 & 3 \\ 3 & 1 & 2 \end{matrix} \right),
  s:      \left( \begin{matrix} 1 & 2 & 3 \\ 1 & 3 & 2 \end{matrix} \right), 
  t:       \left( \begin{matrix} 1 & 2 & 3 \\ 3 & 2 & 1 \end{matrix} \right), 
  u:      \left( \begin{matrix} 1 & 2 & 3 \\ 2 & 1 & 3 \end{matrix} \right) \]
\\

Solution: 
\begin{addmargin}[1em]{0em}
\begin{tabular}{c | c c c c c c}
* & $r_0$ & $r_1$ & $r_2$ & $s$ & $t$ & $u$ \\ \hline
$r_0$ & $r_0$ & $r_1$ & $r_2$ & $s$ & $t$ & $u$ \\
$r_1$ & $r_1$ & $r_2$ & $r_0$ & $t$ & $u$ & $s$\\
$r_2$ & $r_2$ & $r_0$ & $r_1$ & $u$ & $s$ & $t$\\
$s$    & $s$ & $u$ & $t$ & $r_0$ & $r_2$ & $r_1$\\
$t$     & $t$ & $s$ & $u$ & $r_1$ & $r_0$ & $r_2$\\
$u$    & $u$ & $t$ & $s$ & $r_2$ & $r_1$ & $r_0$
\end{tabular}
\end{addmargin}

\newpage
%----------------------------------------------------------------------------------------

\section*{Problem 16}


\textbf{Problem statement}: Let \textbf{1, i, j, k} be the following matricies with complex entries:
\[ \textbf{1} = \left( \begin{matrix} 1 & 0 \\ 0 & 1 \end{matrix} \right),
    \textbf{i} = \left( \begin{matrix} i & 0 \\ 0 & -i \end{matrix} \right),
    \textbf{j} = \left( \begin{matrix} 0 & 1 \\ -1 & 0 \end{matrix} \right),
    \textbf{k} = \left( \begin{matrix} 0 & i \\ i & 0 \end{matrix} \right) \]
\\

\textbf{(a):} Prove that :
\[ \textbf{i$^2$}  = \textbf{j$^2$} = \textbf{k$^2$} = \textbf{-1} \qquad \textbf{ij} = \textbf{-ji} = \textbf{k} \]
\[ \textbf{jk} = \textbf{-kj} = \textbf{i} \qquad \textbf{ki} = \textbf{-ik} = \textbf{j}\]
\begin{addmargin}[1em]{0em}
\underline{Solution: }
\begin{addmargin}[1em]{0em}
\begin{proof}
Suppose $H = \{\textbf{1} = \left( \begin{matrix} 1 & 0 \\ 0 & 1 \end{matrix} \right),
                          \textbf{i} = \left( \begin{matrix} i & 0 \\ 0 & -i \end{matrix} \right),
                           \textbf{j} = \left( \begin{matrix} 0 & 1 \\ -1 & 0 \end{matrix} \right),
                            \textbf{k} = \left( \begin{matrix} 0 & i \\ i & 0 \end{matrix} \right) \}$
\\First note that $H \subset \mathbb{M}_2(\mathbb{C})$.
\\So, $\textbf{i}^2 =  \left( \begin{matrix} i & 0 \\ 0 & -i \end{matrix} \right)\left( \begin{matrix} i & 0 \\ 0 & -i \end{matrix} \right)  = \left( \begin{matrix} -1 & 0 \\ 0 & -1 \end{matrix} \right) = \textbf{-1}$
\\ $\textbf{j}^2 = \left( \begin{matrix} 0 & 1 \\ -1 & 0 \end{matrix} \right)\left( \begin{matrix} 0 & 1 \\ -1 & 0 \end{matrix} \right) = \left( \begin{matrix} -1 & 0 \\ 0 & -1 \end{matrix} \right) = \textbf{-1}$
\\ $\textbf{k}^2 = \left( \begin{matrix} 0 & i \\ i & 0 \end{matrix} \right) \left( \begin{matrix} 0 & i \\ i & 0 \end{matrix} \right)  = \left( \begin{matrix} -1 & 0 \\ 0 & -1 \end{matrix} \right) = \textbf{-1}$
\\Also, $\textbf{i}\textbf{j} = \left( \begin{matrix} i & 0 \\ 0 & -i \end{matrix} \right) \left( \begin{matrix} 0 & 1 \\ -1 & 0 \end{matrix} \right) = \left( \begin{matrix} 0 & i \\ i & 0 \end{matrix} \right) = \textbf{k}$
\\ $-\textbf{j}\textbf{i} = -\left( \begin{matrix} 0 & 1 \\ -1 & 0 \end{matrix} \right) \left( \begin{matrix} i & 0 \\ 0 & -i \end{matrix} \right) = - \left( \begin{matrix} 0 & -i \\ -i & 0 \end{matrix} \right) = -(\textbf{-k}) = \textbf{k}$
\\And, $\textbf{j}\textbf{k} = \left( \begin{matrix} 0 & 1 \\ -1 & 0 \end{matrix} \right)\left( \begin{matrix} 0 & i \\ i & 0 \end{matrix} \right) = \left( \begin{matrix} i & 0 \\ 0 & -i \end{matrix} \right) = \textbf{i}$
\\ $-\textbf{k}\textbf{j} = -\left( \begin{matrix} 0 & i \\ i & 0 \end{matrix} \right)\left( \begin{matrix} 0 & 1 \\ -1 & 0 \end{matrix} \right) = -\left( \begin{matrix} -i & 0 \\ 0 & i \end{matrix} \right) = -(\textbf{-i}) = \textbf{i}$
\\Finally, $\textbf{k}\textbf{i} = \left( \begin{matrix} 0 & i \\ i & 0 \end{matrix} \right)\left( \begin{matrix} i & 0 \\ 0 & -i \end{matrix} \right) = \left( \begin{matrix} 0 & 1 \\ -1 & 0 \end{matrix} \right) = \textbf{j}$
\\ $-\textbf{i}\textbf{k} = -\left( \begin{matrix} i & 0 \\ 0 & -i \end{matrix} \right)\left( \begin{matrix} 0 & i \\ i & 0 \end{matrix} \right) = -\left( \begin{matrix} 0 & -1 \\ 1 & 0 \end{matrix} \right) = -(\textbf{-j}) = \textbf{j}$
\end{proof}
\end{addmargin}
\end{addmargin}

\newpage

\textbf{(b):} Show that the set $Q = \{\textbf{1}, \textbf{i}, \textbf{-1}, \textbf{-i}, \textbf{j}, \textbf{k}, \textbf{-j}, \textbf{-k}\}$ is a group under matrix multiplication by writing out its multiplication table.  $Q$ is called the \textbf{quaternion group}.
\begin{addmargin}[1em]{0em}
\underline{Solution: }
\begin{addmargin}[1em]{0em}
\begin{tabular}{c | c c c c c c c c}
*                & \textbf{1} & \textbf{i} & \textbf{-1} & \textbf{-i} & \textbf{j} & \textbf{k} & \textbf{-j} & \textbf{-k} \\ \hline
\textbf{1} & \textbf{1} & \textbf{i} & \textbf{-1} & \textbf{-i} & \textbf{j} & \textbf{k} & \textbf{-j} & \textbf{-k} \\
\textbf{i} & \textbf{i} & \textbf{-1} & \textbf{-i} & \textbf{1} & \textbf{k} & \textbf{-j} & \textbf{-k} & \textbf{j} \\
\textbf{-1} & \textbf{-1} & \textbf{-i} & \textbf{1} & \textbf{i} & \textbf{-j} & \textbf{-k} & \textbf{j} & \textbf{k} \\
\textbf{-i} & \textbf{-i} & \textbf{1} & \textbf{i} & \textbf{-1} & \textbf{-k} & \textbf{j} & \textbf{k} & \textbf{-j} \\
\textbf{j} & \textbf{j} & \textbf{k} & \textbf{-j} & \textbf{-k} & \textbf{-1} & \textbf{i} & \textbf{1} & \textbf{-i} \\
\textbf{k} & \textbf{k} & \textbf{-j} & \textbf{-k} & \textbf{j} & \textbf{-i} & \textbf{-1} & \textbf{i} & \textbf{1} \\
\textbf{-j} & \textbf{-j} & \textbf{-k} & \textbf{j} & \textbf{k} & \textbf{1} & \textbf{-i} & \textbf{-1} & \textbf{i} \\
\textbf{-k} & \textbf{-k} & \textbf{j} & \textbf{k} & \textbf{-j} & \textbf{i} & \textbf{1} & \textbf{-i} & \textbf{-1}

\end{tabular}
\end{addmargin}
\end{addmargin}


\newpage
%----------------------------------------------------------------------------------------

\section*{Problem 21}


\textbf{Problem statement}: Suppose $G$ is a group with operation *.  Define a new operation $\#$ on $G$ by $a \# b = b * a$.  Prove that $G$ is a group under $\#$.
\\

Solution: 
\begin{addmargin}[1em]{0em}
\begin{proof}

\end{proof}
\end{addmargin}

\newpage
%----------------------------------------------------------------------------------------

\section*{Problem 30}


\textbf{Problem statement}: Fill in the below table completely (initial entries are bolded)
\\

Solution: 
\begin{addmargin}[1em]{0em}
\begin{tabular} {c | c c c c c c}
                  & \textbf{e} & \textbf{a} & \textbf{b} & \textbf{c} & \textbf{d} & \textbf{f} \\ \hline
\textbf{e} & \textbf{e} & \textbf{a} & \textbf{b} & \textbf{c} & \textbf{d} & \textbf{f} \\
\textbf{a} & \textbf{a} & \textbf{b} & \textbf{e} & \textbf{d} &        f         &        c       \\
\textbf{b} & \textbf{b} &         e         &       a          &         f         &        c          &       d       \\
\textbf{c} & \textbf{c} & \textbf{f}  &        d          &         e         &        b          & \textbf{a} \\
\textbf{d} & \textbf{d} &          c        &       f          &          a        &         e         &       b         \\
\textbf{f} & \textbf{f} &          d        &        c         &            b      &           a       &       e        \\
\end{tabular}
\end{addmargin}

\newpage
%----------------------------------------------------------------------------------------

\end{document}