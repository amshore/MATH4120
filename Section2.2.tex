%----------------------------------------------------------------------------------------
%   PACKAGES AND OTHER DOCUMENT CONFIGURATIONS
%----------------------------------------------------------------------------------------

\documentclass{article} % paper and 12pt font size

\usepackage{scrextend, tikz, amssymb}
\usepackage{amsmath,amsfonts,amsthm} % Math packages
\setlength\parindent{0pt} % Removes all indentation from paragraphs - comment this line for an assignment with lots of text

%----------------------------------------------------------------------------------------
%   TITLE SECTION
%----------------------------------------------------------------------------------------

\newcommand{\horrule}[1]{\rule{\linewidth}{#1}} % Create horizontal rule command with 1 argument of height

\title{ 
\normalfont \normalsize 
\textsc{MATH 4120-001 --- Abstract Algebra} \\
\horrule{0.5pt} \\[0cm] % Thin top horizontal rule
\huge Section 2.2: 1d, 2, 7*, 12, 14*, 16a, 16d \\ % The assignment title
\horrule{2pt} \\[0cm] % Thick bottom horizontal rule
}
\author{Andrew Shore} % Your name
\date{\normalsize\today} % Today's date or a custom date
\begin{document}

\maketitle % Print the title

%----------------------------------------------------------------------------------------
%   PROBLEM 1
%----------------------------------------------------------------------------------------
\section*{Problem 1d}


\textbf{Problem statement}: Write out the addition and multiplication tables for $\mathbb{Z}_{12}$

\underline{Solution}: 
\begin{addmargin}[1em]{0em}
\begin{tabular}{|c|c|c|c|c|c|c|c|c|c|c|c|c|}
\hline
\textbf{$\oplus$}&\textbf{[0]}&\textbf{[1]}&\textbf{[2]}&\textbf{[3]}&\textbf{[4]}&\textbf{[5]}&\textbf{[6]}&\textbf{[7]}&\textbf{[8]}&\textbf{[9]}&\textbf{[10]}&\textbf{[11]}\\ \hline
\textbf{[0]} & [0] & [1] & [2] & [3] & [4] & [5] & [6] & [7] & [8] & [9] & [10] & [11] \\ \hline
\textbf{[1]} & [1] & [2] & [3] & [4] & [5] & [6] & [7] & [8] & [9] & [10] & [11] & [0]\\ \hline
\textbf{[2]} & [2] & [3] & [4] & [5] & [6] & [7] & [8] & [9] & [10] & [11] & [0] & [1]\\ \hline
\textbf{[3]} & [3] & [4] & [5] & [6] & [7] & [8] & [9] & [10] & [11] & [0] & [1] & [2]\\ \hline
\textbf{[4]} & [4] & [5] & [6] & [7] & [8] & [9] & [10] & [11] & [0] & [1] & [2] & [3]\\ \hline
\textbf{[5]} & [5] & [6] & [7] & [8] & [9] & [10] & [11] & [0] & [1] & [2] & [3] & [4]\\ \hline
\textbf{[6]} & [6] & [7] & [8] & [9] & [10] & [11] & [0] & [1] & [2] & [3] & [4] & [5]\\ \hline
\textbf{[7]} & [7] & [8] & [9] & [10] & [11] & [0] & [1] & [2] & [3] & [4] & [5] & [6]\\ \hline
\textbf{[8]} & [8] & [9] & [10] & [11] & [0] & [1] & [2] & [3] & [4] & [5] & [6] & [7]\\ \hline
\textbf{[9]} & [9] & [10] & [11] & [0] & [1] & [2] & [3] & [4] & [5] & [6] & [7] & [8]\\ \hline
\textbf{[10]} & [10] & [11] & [0] & [1] & [2] & [3] & [4] & [5] & [6] & [7] & [8] & [9]\\ \hline
\textbf{[11]} & [11] & [0] & [1] & [2] & [3] & [4] & [5] & [6] & [7] & [8] & [9] & [10]\\ \hline

\end{tabular}

\begin{tabular}{|c|c|c|c|c|c|c|c|c|c|c|c|c|}
\hline
\textbf{$\odot$}&\textbf{[0]}&\textbf{[1]}&\textbf{[2]}&\textbf{[3]}&\textbf{[4]}&\textbf{[5]}&\textbf{[6]}&\textbf{[7]}&\textbf{[8]}&\textbf{[9]}&\textbf{[10]}&\textbf{[11]}\\ \hline
\textbf{[0]} & [0] & [0] & [0] & [0] & [0] & [0] & [0] & [0] & [0] & [0] & [0] & [0] \\ \hline
\textbf{[1]} & [0] & [1] & [2] & [3] & [4] & [5] & [6] & [7] & [8] & [9] & [10] & [11]\\ \hline
\textbf{[2]} & [0] & [2] & [4] & [6] & [8] & [10] & [0] & [2] & [4] & [6] & [8] & [10]\\ \hline
\textbf{[3]} & [0] & [3] & [6] & [9] & [0] & [3] & [6] & [9] & [0] & [3] & [6] & [9]\\ \hline
\textbf{[4]} & [0] & [4] & [8] & [0] & [4] & [8] & [0] & [4] & [8] & [0] & [4] & [8]\\ \hline
\textbf{[5]} & [0] & [5] & [10] & [3] & [8] & [1] & [6] & [11] & [4] & [9] & [2] & [7]\\ \hline
\textbf{[6]} & [0] & [6] & [0] & [6] & [0] & [6] & [0] & [6] & [0] & [6] & [0] & [6]\\ \hline
\textbf{[7]} & [0] & [7] & [2] & [9] & [4] & [11] & [6] & [1] & [8] & [3] & [10] & [5]\\ \hline
\textbf{[8]} & [0] & [8] & [4] & [0] & [8] & [4] & [0] & [8] & [4] & [0] & [8] & [4]\\ \hline
\textbf{[9]} & [0] & [9] & [6] & [3] & [0] & [9] & [6] & [3] & [0] & [9] & [6] & [3]\\ \hline
\textbf{[10]} & [0] & [10] & [8] & [6] & [4] & [2] & [0] & [10] & [8] & [6] & [4] & [2] \\ \hline
\textbf{[11]} & [0] & [11] & [10] & [9] & [8] & [7] & [6] & [5] & [4] & [3] & [2] & [1]\\ \hline

\end{tabular}
\end{addmargin}    

\newpage
%----------------------------------------------------------------------------------------

\section*{Problem 2}

\textbf{Problem statement}: Solve the equation $x^2 \oplus x = [0]$ in $\mathbb{Z}_4$.
\\

\underline{Solution}: 
\begin{addmargin}[1em]{0em}
\begin{tabular}{cc}
\begin{tabular}{|c|c|c|c|c|}
\hline
\textbf{$\oplus$}&\textbf{[0]}&\textbf{[1]}&\textbf{[2]}&\textbf{[3]}\\ \hline
\textbf{[0]} & [0] & [1] & [2] & [3] \\ \hline
\textbf{[1]} & [1] & [2] & [3] & [0] \\ \hline
\textbf{[2]} & [2] & [3] & [0] & [1] \\ \hline
\textbf{[3]} & [3] & [0] & [1] & [2] \\ \hline

\end{tabular}

\quad

\begin{tabular}{|c|c|c|c|c|}
\hline
\textbf{$\odot$}&\textbf{[0]}&\textbf{[1]}&\textbf{[2]}&\textbf{[3]}\\ \hline
\textbf{[0]} & [0] & [0] & [0] & [0] \\ \hline
\textbf{[1]} & [0] & [1] & [2] & [3] \\ \hline
\textbf{[2]} & [0] & [2] & [0] & [2] \\ \hline
\textbf{[3]} & [0] & [3] & [2] & [1] \\ \hline
\end{tabular}
\end{tabular}
\\ \break
In order to solve this problem, first look for entries of the addition table, for $[0]$ solutions.
\\Thus, the possible solutions values to consider in the form $(x^2, x)$ are: $([0],[0])$, $([3],[1])$, $([2],[2])$, $([1],[3])$.
\\For each of the above possible solutions, it is only valid if $x^2 = x \odot x$:
\\$[0] \odot [0] = [0] = [0] \checkmark$
\\$[1] \odot [1] = [1] \neq [3]$ \textbf{X}
\\$[2] \odot [2] = [0] \neq [2]$ \textbf{X}
\\$[3] \odot [3] = [1] = [1] \checkmark$
\\Thus, $x = [0]$ and $x = [3]$ are solutions.

\begin{addmargin}[1em]{0em}
\end{addmargin}
\end{addmargin}

\newpage
%----------------------------------------------------------------------------------------

\section*{Problem 7*}


\textbf{Problem statement}: Solve the equation $x^3 \oplus x^2 \oplus x \oplus [1] = [0]$ in $\mathbb{Z}_8$.
\\

\underline{Solution}: \\
\begin{tabular}{cc}
\begin{tabular}{|c|c|c|c|c|c|c|c|c|}
\hline
\textbf{$\oplus$}&\textbf{[0]}&\textbf{[1]}&\textbf{[2]}&\textbf{[3]}&\textbf{[4]}&\textbf{[5]}&\textbf{[6]}&\textbf{[7]}\\ \hline
\textbf{[0]} & [0] & [1] & [2] & [3] & [4] & [5] & [6] & [7]\\ \hline
\textbf{[1]} & [1] & [2] & [3] & [4] & [5] & [6] & [7] & [0]\\ \hline
\textbf{[2]} & [2] & [3] & [4] & [5] & [6] & [7] & [0] & [1]\\ \hline
\textbf{[3]} & [3] & [4] & [5] & [6] & [7] & [0] & [1] & [2]\\ \hline
\textbf{[4]} & [4] & [5] & [6] & [7] & [0] & [1] & [2] & [3]\\ \hline
\textbf{[5]} & [5] & [6] & [7] & [0] & [1] & [2] & [3] & [4]\\ \hline
\textbf{[6]} & [6] & [7] & [0] & [1] & [2] & [3] & [4] & [5]\\ \hline
\textbf{[7]} & [7] & [0] & [1] & [2] & [3] & [4] & [5] & [6]\\ \hline
\end{tabular}

\quad

\begin{tabular}{|c|c|c|c|c|c|c|c|c|}
\hline
\textbf{$\odot$}&\textbf{[0]}&\textbf{[1]}&\textbf{[2]}&\textbf{[3]}&\textbf{[4]}&\textbf{[5]}&\textbf{[6]}&\textbf{[7]}\\ \hline
\textbf{[0]} & [0] & [0] & [0] & [0] & [0] & [0] & [0] & [0]\\ \hline
\textbf{[1]} & [0] & [1] & [2] & [3] & [4] & [5] & [6] & [7]\\ \hline
\textbf{[2]} & [0] & [2] & [4] & [6] & [0] & [2] & [4] & [6]\\ \hline
\textbf{[3]} & [0] & [3] & [6] & [1] & [4] & [7] & [2] & [5]\\ \hline
\textbf{[4]} & [0] & [4] & [0] & [4] & [0] & [4] & [0] & [4]\\ \hline
\textbf{[5]} & [0] & [5] & [2] & [4] & [7] & [1] & [6] & [3]\\ \hline
\textbf{[6]} & [0] & [6] & [4] & [2] & [0] & [6] & [4] & [2]\\ \hline
\textbf{[7]} & [0] & [7] & [6] & [5] & [4] & [3] & [2] & [1]\\ \hline
\end{tabular}
\end{tabular}
\begin{addmargin}[1em]{0em}
\vspace{1em}
For this problem, it is easier to check all possible values of $x$.
\\$x = [0]: \quad x^3 = [0]; \quad x^2 = [0]; \quad x = [0]$ 
\\ \hspace*{1em} thus, $x^3 \oplus x^2 \oplus x \oplus [1] = [0] \oplus [0] \oplus [0] \oplus [1] = [0] \oplus [1] = [1]$
\\$x = [1]: \quad x^3 = [1]; \quad x^2 = [1]; \quad x = [1]$ 
\\ \hspace*{1em} thus, $x^3 \oplus x^2 \oplus x \oplus [1] = [1] \oplus [1] \oplus [1] \oplus [1] = [2] \oplus [2] = [4]$
\\$x = [2]: \quad x^3 = [0]; \quad x^2 = [4]; \quad x = [2]$ 
\\ \hspace*{1em} thus, $x^3 \oplus x^2 \oplus x \oplus [1] = [0] \oplus [4] \oplus [2] \oplus [1] = [4] \oplus [3] = [7]$
\\$x = [3]: \quad x^3 = [3]; \quad x^2 = [1]; \quad x = [3]$ 
\\ \hspace*{1em} thus, $x^3 \oplus x^2 \oplus x \oplus [1] = [3] \oplus [1] \oplus [3] \oplus [1] = [4] \oplus [4] = [0] \checkmark$
\\$x = [4]: \quad x^3 = [0]; \quad x^2 = [0]; \quad x = [4]$ 
\\ \hspace*{1em} thus, $x^3 \oplus x^2 \oplus x \oplus [1] = [0] \oplus [0] \oplus [4] \oplus [1] = [0] \oplus [5] = [5]$
\\$x = [5]: \quad x^3 = [5]; \quad x^2 = [1]; \quad x = [5]$ 
\\ \hspace*{1em} thus, $x^3 \oplus x^2 \oplus x \oplus [1] = [5] \oplus [1] \oplus [5] \oplus [1] = [6] \oplus [6] = [4]$
\\$x = [6]: \quad x^3 = [0]; \quad x^2 = [4]; \quad x = [6]$ 
\\ \hspace*{1em} thus, $x^3 \oplus x^2 \oplus x \oplus [1] = [0] \oplus [4] \oplus [6] \oplus [1] = [4] \oplus [7] = [3]$
\\$x = [7]: \quad x^3 = [7]; \quad x^2 = [1]; \quad x = [7]$ 
\\ \hspace*{1em} thus, $x^3 \oplus x^2 \oplus x \oplus [1] = [7] \oplus [1] \oplus [7] \oplus [1] = [0] \oplus [0] = [0] \checkmark$
\\ Thus $x = [3]$ or $[7]$.
\end{addmargin}

\newpage
%----------------------------------------------------------------------------------------

\section*{Problem 12}


\textbf{Problem statement}: Prove or disprove: If $[a] \odot [b] = [0]$ in $\mathbb{Z}_n$, then $[a] = [0]$ or $[b] = [0]$.
\\

\underline{Solution}: 
\begin{addmargin}[1em]{0em}
This statement is false.
\\As a counterexample, let $a = 4$, $b = 2$, $n = 8$.
\\As seen in the multiplication table for $\mathbb{Z}_8$ in the previous problem, $[4] \odot [2] = [0]$ but $a \neq [0]$ and $b \neq [0]$.
\\Thus, the proposition in false.
\end{addmargin}

\newpage
%----------------------------------------------------------------------------------------

\section*{Problem 14*}


\textbf{Problem statement}: Solve the following equations.
\\

\textbf{(a)}: $x^2 + x = [0]$ in $\mathbb{Z}_5$
\\
\begin{addmargin}[1em]{0em}
\underline{Solution}: 
\begin{addmargin}[1em]{0em}
\begin{tabular}{cc}
\begin{tabular}{|c|c|c|c|c|c|}
\hline
\textbf{$\oplus$}&\textbf{[0]}&\textbf{[1]}&\textbf{[2]}&\textbf{[3]}&\textbf{[4]}\\ \hline
\textbf{[0]} & [0] & [1] & [2] & [3] & [4] \\ \hline
\textbf{[1]} & [1] & [2] & [3] & [4] & [0]\\ \hline
\textbf{[2]} & [2] & [3] & [4] & [0] & [1]\\ \hline
\textbf{[3]} & [3] & [4] & [0] & [1] & [2]\\ \hline
\textbf{[4]} & [4] & [0] & [1] & [2] & [3]\\ \hline
\end{tabular}

\quad

\begin{tabular}{|c|c|c|c|c|c|}
\hline
\textbf{$\odot$}&\textbf{[0]}&\textbf{[1]}&\textbf{[2]}&\textbf{[3]}&\textbf{[4]}\\ \hline
\textbf{[0]} & [0] & [0] & [0] & [0] & [0] \\ \hline
\textbf{[1]} & [0] & [1] & [2] & [3] & [4]\\ \hline
\textbf{[2]} & [0] & [2] & [4] & [1] & [3]\\ \hline
\textbf{[3]} & [0] & [3] & [1] & [4] & [2]\\ \hline
\textbf{[4]} & [0] & [4] & [3] & [2] & [1]\\ \hline
\end{tabular}
\end{tabular}
\\ \break First observe that $x^2 \oplus x = x\odot(x \oplus [1])$
\\Using the above tables, $x = [0]$ or $x = [4]$.
\\
\end{addmargin}
\end{addmargin}

\textbf{(b)}: $x^2 + x = [0]$ in $\mathbb{Z}_6$
\\
\begin{addmargin}[1em]{0em}
\underline{Solution}: 
\begin{addmargin}[1em]{0em}
\begin{tabular}{cc}
\begin{tabular}{|c|c|c|c|c|c|c|}
\hline
\textbf{$\oplus$}&\textbf{[0]}&\textbf{[1]}&\textbf{[2]}&\textbf{[3]}&\textbf{[4]}&\textbf{[5]}\\ \hline
\textbf{[0]} & [0] & [1] & [2] & [3] & [4] & [5]\\ \hline
\textbf{[1]} & [1] & [2] & [3] & [4] & [5] & [0]\\ \hline
\textbf{[2]} & [2] & [3] & [4] & [5] & [0] & [1]\\ \hline
\textbf{[3]} & [3] & [4] & [5] & [0] & [1] & [2]\\ \hline
\textbf{[4]} & [4] & [5] & [0] & [1] & [2] & [3]\\ \hline
\textbf{[5]} & [5] & [0] & [1] & [2] & [3] & [4]\\ \hline
\end{tabular}

\quad

\begin{tabular}{|c|c|c|c|c|c|c|}
\hline
\textbf{$\odot$}&\textbf{[0]}&\textbf{[1]}&\textbf{[2]}&\textbf{[3]}&\textbf{[4]}&\textbf{[5]}\\ \hline
\textbf{[0]} & [0] & [0] & [0] & [0] & [0] & [0]\\ \hline
\textbf{[1]} & [0] & [1] & [2] & [3] & [4] & [5]\\ \hline
\textbf{[2]} & [0] & [2] & [4] & [0] & [2] & [4]\\ \hline
\textbf{[3]} & [0] & [3] & [0] & [3] & [0] & [3]\\ \hline
\textbf{[4]} & [0] & [4] & [2] & [0] & [4] & [2]\\ \hline
\textbf{[5]} & [0] & [5] & [4] & [3] & [2] & [1]\\ \hline
\end{tabular}
\end{tabular}
\\ \break First observe that $x^2 \oplus x = x\odot(x \oplus [1])$
\\Using the above tables, $x = [0]$, $x = [2]$, $x = [3]$, $x = [5]$.
\\
\end{addmargin}
\end{addmargin}

\textbf{(c)}: If $p$ is prime, prove that the only solution of $x^2 + x = [0]$ in $\mathbb{Z}_p$ are $[0]$ and $[p-1]$.
\\
\begin{addmargin}[1em]{0em}
\underline{Solution}: 
\begin{addmargin}[1em]{0em}
\begin{proof}
Let $p$ be prime and $x$ be a solution to $x^2 \oplus x = [0]$ in $\mathbb{Z}_p$.
\\Then note that $x^2 \oplus x = x \odot (x \oplus [1])$.
\\Thus by theorem 2.8, $x = [0]$ or $x \oplus [1] = [0]$.
\\So, in the second case, $x = [0] - [1] = [p] - [1] = [p-1]$.
\\Therefore, $x = [1]$ or $x = [p-1]$.
\end{proof}
\end{addmargin}
\end{addmargin}

\newpage
%----------------------------------------------------------------------------------------

\section*{Problem 16a}


\textbf{Problem statement}: Find all $[a]$ in $\mathbb{Z}_5$ for which the equation $[a] \odot x = [1]$ has a solution.
\\

\underline{Solution}: 
\begin{addmargin}[1em]{0em}
\begin{tabular}{cc}
\begin{tabular}{|c|c|c|c|c|c|}
\hline
\textbf{$\oplus$}&\textbf{[0]}&\textbf{[1]}&\textbf{[2]}&\textbf{[3]}&\textbf{[4]}\\ \hline
\textbf{[0]} & [0] & [1] & [2] & [3] & [4] \\ \hline
\textbf{[1]} & [1] & [2] & [3] & [4] & [0]\\ \hline
\textbf{[2]} & [2] & [3] & [4] & [0] & [1]\\ \hline
\textbf{[3]} & [3] & [4] & [0] & [1] & [2]\\ \hline
\textbf{[4]} & [4] & [0] & [1] & [2] & [3]\\ \hline
\end{tabular}

\quad

\begin{tabular}{|c|c|c|c|c|c|}
\hline
\textbf{$\odot$}&\textbf{[0]}&\textbf{[1]}&\textbf{[2]}&\textbf{[3]}&\textbf{[4]}\\ \hline
\textbf{[0]} & [0] & [0] & [0] & [0] & [0] \\ \hline
\textbf{[1]} & [0] & [1] & [2] & [3] & [4]\\ \hline
\textbf{[2]} & [0] & [2] & [4] & [1] & [3]\\ \hline
\textbf{[3]} & [0] & [3] & [1] & [4] & [2]\\ \hline
\textbf{[4]} & [0] & [4] & [3] & [2] & [1]\\ \hline
\end{tabular}
\end{tabular}
\\ \break Using the above tables, $a = [1]$, $a = [2]$, $a = [3]$, $a = [4]$.
\end{addmargin}

\newpage
%----------------------------------------------------------------------------------------

\section*{Problem 16d}


\textbf{Problem statement}: Find all $[a]$ in $\mathbb{Z}_6$ for which the equation $[a] \odot x = [1]$ has a solution.
\\

\underline{Solution}: 
\begin{addmargin}[1em]{0em}
\begin{tabular}{cc}
\begin{tabular}{|c|c|c|c|c|c|c|}
\hline
\textbf{$\oplus$}&\textbf{[0]}&\textbf{[1]}&\textbf{[2]}&\textbf{[3]}&\textbf{[4]}&\textbf{[5]}\\ \hline
\textbf{[0]} & [0] & [1] & [2] & [3] & [4] & [5]\\ \hline
\textbf{[1]} & [1] & [2] & [3] & [4] & [5] & [0]\\ \hline
\textbf{[2]} & [2] & [3] & [4] & [5] & [0] & [1]\\ \hline
\textbf{[3]} & [3] & [4] & [5] & [0] & [1] & [2]\\ \hline
\textbf{[4]} & [4] & [5] & [0] & [1] & [2] & [3]\\ \hline
\textbf{[5]} & [5] & [0] & [1] & [2] & [3] & [4]\\ \hline
\end{tabular}

\quad

\begin{tabular}{|c|c|c|c|c|c|c|}
\hline
\textbf{$\odot$}&\textbf{[0]}&\textbf{[1]}&\textbf{[2]}&\textbf{[3]}&\textbf{[4]}&\textbf{[5]}\\ \hline
\textbf{[0]} & [0] & [0] & [0] & [0] & [0] & [0]\\ \hline
\textbf{[1]} & [0] & [1] & [2] & [3] & [4] & [5]\\ \hline
\textbf{[2]} & [0] & [2] & [4] & [0] & [2] & [4]\\ \hline
\textbf{[3]} & [0] & [3] & [0] & [3] & [0] & [3]\\ \hline
\textbf{[4]} & [0] & [4] & [2] & [0] & [4] & [2]\\ \hline
\textbf{[5]} & [0] & [5] & [4] & [3] & [2] & [1]\\ \hline
\end{tabular}
\end{tabular}
\\ \break Using the above table, $a = [1]$ or $a = [5]$.
\end{addmargin}

%----------------------------------------------------------------------------------------

\end{document}