%----------------------------------------------------------------------------------------
%   PACKAGES AND OTHER DOCUMENT CONFIGURATIONS
%----------------------------------------------------------------------------------------

\documentclass{article} % paper and 12pt font size

\usepackage{scrextend, tikz, amssymb}
\usepackage{amsmath,amsfonts,amsthm} % Math packages
\setlength\parindent{0pt} % Removes all indentation from paragraphs - comment this line for an assignment with lots of text

%----------------------------------------------------------------------------------------
%   TITLE SECTION
%----------------------------------------------------------------------------------------

\newcommand{\horrule}[1]{\rule{\linewidth}{#1}} % Create horizontal rule command with 1 argument of height

\title{ 
\normalfont \normalsize 
\textsc{MATH 4120-001 --- Abstract Algebra} \\
\horrule{0.5pt} \\[0cm] % Thin top horizontal rule
\huge Section 2.2: 1d, 2, 7*, 12, 14*, 16a, 16d \\ % The assignment title
\horrule{2pt} \\[0cm] % Thick bottom horizontal rule
}
\author{Andrew Shore} % Your name
\date{\normalsize\today} % Today's date or a custom date
\begin{document}

\maketitle % Print the title

%----------------------------------------------------------------------------------------
%   PROBLEM 1
%----------------------------------------------------------------------------------------
\section*{Problem 1d}


\textbf{Problem statement}: Write out the addition and multiplication tables for $\mathbb{Z}_{12}$
\\

\underline{Solution}: 
\begin{addmargin}[1em]{0em}
\begin{tabular}{|c|c|c|c|c|c|c|c|c|c|}
\hline
1&2&3&4&5 \\ \hline
\end{tabular}
\end{addmargin}    

\newpage
%----------------------------------------------------------------------------------------

\section*{Problem 2}

\textbf{Problem statement}: Solve the equation $x^2 \oplus x = [0]$ in $\mathbb{Z}_4$.
\\

\underline{Solution}: 
\begin{addmargin}[1em]{0em}
\textbf{(a)}:
\begin{addmargin}[1em]{0em}
\end{addmargin}
\end{addmargin}

\newpage
%----------------------------------------------------------------------------------------

\section*{Problem 7*}


\textbf{Problem statement}: Solve the equation $x^3 \oplus x^2 \oplus x \oplus [1] = [0]$ in $\mathbb{Z}_8$.
\\

\underline{Solution}: 
\begin{addmargin}[1em]{0em}
\begin{proof}

\end{proof}
\end{addmargin}

\newpage
%----------------------------------------------------------------------------------------

\section*{Problem 12}


\textbf{Problem statement}: Prove or disprove: If $[a] \odot [b] = [0]$ in $\mathbb{Z}_n$, then $[a] = [0]$ or $[b] = [0]$.
\\

\underline{Solution}: 
\begin{addmargin}[1em]{0em}
\begin{proof}

\end{proof}
\end{addmargin}

\newpage
%----------------------------------------------------------------------------------------

\section*{Problem 14*}


\textbf{Problem statement}: Solve the following equations.
\\

\textbf{(a)}: $x^2 + x = [0]$ in $\mathbb{Z}_5$
\\
\begin{addmargin}[1em]{0em}
\underline{Solution}: 
\begin{addmargin}[1em]{0em}
\end{addmargin}
\end{addmargin}

\textbf{(b)}: $x^2 + x = [0]$ in $\mathbb{Z}_6$
\\
\begin{addmargin}[1em]{0em}
\underline{Solution}: 
\begin{addmargin}[1em]{0em}
\end{addmargin}
\end{addmargin}

\textbf{(c)}: If $p$ is prime, prove that the only solution of $x^2 + x = [0]$ in $\mathbb{Z}_p$ are $[0]$ and $[p-1]$.
\\
\begin{addmargin}[1em]{0em}
\underline{Solution}: 
\begin{addmargin}[1em]{0em}
\end{addmargin}
\end{addmargin}

\newpage
%----------------------------------------------------------------------------------------

\section*{Problem 16a}


\textbf{Problem statement}: Find all $[a]$ in $\mathbb{Z}_5$ for which the equation $[a] \odot x = [1]$ has a solution.
\\

\underline{Solution}: 
\begin{addmargin}[1em]{0em}
\begin{proof}

\end{proof}
\end{addmargin}

\newpage
%----------------------------------------------------------------------------------------

\section*{Problem 16d}


\textbf{Problem statement}: Find all $[a]$ in $\mathbb{Z}_6$ for which the equation $[a] \odot x = [1]$ has a solution.
\\

\underline{Solution}: 
\begin{addmargin}[1em]{0em}
\begin{proof}

\end{proof}
\end{addmargin}

%----------------------------------------------------------------------------------------

\end{document}