%----------------------------------------------------------------------------------------
%   PACKAGES AND OTHER DOCUMENT CONFIGURATIONS
%----------------------------------------------------------------------------------------

\documentclass{article} % paper and 12pt font size

\usepackage{scrextend, tikz, amssymb, tabularx}
\usepackage{amsmath,amsfonts,amsthm} % Math packages
\setlength\parindent{0pt} % Removes all indentation from paragraphs - comment this line for an assignment with lots of text

%----------------------------------------------------------------------------------------
%   TITLE SECTION
%----------------------------------------------------------------------------------------

\newcommand{\horrule}[1]{\rule{\linewidth}{#1}} % Create horizontal rule command with 1 argument of height

\title{ 
\normalfont \normalsize 
\textsc{MATH 4120-001 --- Abstract Algebra} \\
\horrule{0.5pt} \\[0cm] % Thin top horizontal rule
\huge Section 7.2: 1,2,7,9b,19,20*,33 \\ % The assignment title
\horrule{2pt} \\[0cm] % Thick bottom horizontal rule
}
\author{Andrew Shore} % Your name
\date{\normalsize\today} % Today's date or a custom date
\begin{document}

\maketitle % Print the title

%----------------------------------------------------------------------------------------
%   PROBLEM 1
%----------------------------------------------------------------------------------------
\section*{Problem 1}


\textbf{Problem statement}: If $c^2 = c$ in a group, prove that $c = e$
\\

\underline{Solution}: 
\begin{addmargin}[1em]{0em}
\begin{proof}
Suppose $G$ is a group and $c \in G$ such that $c^2 = c$
\\Then $c^2 = c*c$ and $c = e*c$
\\Thus, $c*c = e*c \implies c = e$ because of cancellation in groups.
\end{proof}
\end{addmargin}    

\newpage
%----------------------------------------------------------------------------------------
\section*{Problem 2}


\textbf{Problem statement}: Let $a = \left( \begin{matrix} 1 & 2 & 3 \\ 3 & 1 & 2 \end{matrix}\right)$ and $b = \left( \begin{matrix} 1 & 2 & 3 \\ 1 & 3 & 2 \end{matrix}\right)$ in $S_3$.  Verify that $(ab)^{-1} \neq a^{-1}b^{-1}$
\\

\underline{Solution}: 
\begin{addmargin}[1em]{0em}
First note that $ab = \left( \begin{matrix} 1 & 2 & 3 \\ 2 & 1 & 3 \end{matrix} \right) \implies (ab)^{-1} = \left( \begin{matrix} 1 & 2 & 3 \\ 2 & 1 & 3\end{matrix} \right)$
\\Also, $a^{-1} = \left( \begin{matrix} 1 & 2 & 3 \\ 2 & 3 & 1 \end{matrix} \right), b^{-1} = \left( \begin{matrix} 1 & 2 & 3 \\ 1 & 3 & 2 \end{matrix}\right) \implies a^{-1}b^{-1} = \left( \begin{matrix} 1 & 2 & 3 \\ 3 & 2 & 1 \end{matrix} \right)$
\\Thus, $(ab)^{-1} \neq a^{-1}b^{-1}$
\end{addmargin}    

\newpage
%----------------------------------------------------------------------------------------
\section*{Problem 7}


\textbf{Problem statement}: Find the order of the given element
\\

\textbf{(a): } 5 in $U_8$
\begin{addmargin}[1em]{0em}
\underline{Solution}: 
\begin{addmargin}[1em]{0em}
$5^2 = 25 = 1$
\\Thus $5$ has order 1 in $U_8$
\end{addmargin}
\end{addmargin}    

\textbf{(b): } $\left(\begin{matrix} 1 & 2 & 3 & 4 & 5 & 6 & 7 \\ 2 & 3 & 7 & 5 & 1 & 4 & 6\end{matrix}\right)$ in $S_7$
\begin{addmargin}[1em]{0em}
\underline{Solution}: 
\begin{addmargin}[1em]{0em}
$\left(\begin{matrix}
1 & 2 & 3 & 4 & 5 & 6 & 7 \\
2 & 3 & 7 & 5 & 1 & 4 & 6
\end{matrix}\right)^2 = 
\left(\begin{matrix}
1 & 2 & 3 & 4 & 5 & 6 & 7 \\
3 & 7 & 6 & 1 & 2 & 5 & 4
\end{matrix}\right)$
\\$\left(\begin{matrix}
1 & 2 & 3 & 4 & 5 & 6 & 7 \\
2 & 3 & 7 & 5 & 1 & 4 & 6
\end{matrix}\right)^3 = 
\left(\begin{matrix}
1 & 2 & 3 & 4 & 5 & 6 & 7 \\
7 & 6 & 4 & 2 & 3 & 1 & 5
\end{matrix}\right)$
\\$\left(\begin{matrix}
1 & 2 & 3 & 4 & 5 & 6 & 7 \\
2 & 3 & 7 & 5 & 1 & 4 & 6
\end{matrix}\right)^4 = 
\left(\begin{matrix}
1 & 2 & 3 & 4 & 5 & 6 & 7 \\
6 & 4 & 5 & 3 & 7 & 2 & 1
\end{matrix}\right)$
\\$\left(\begin{matrix}
1 & 2 & 3 & 4 & 5 & 6 & 7 \\
2 & 3 & 7 & 5 & 1 & 4 & 6
\end{matrix}\right)^5 = 
\left(\begin{matrix}
1 & 2 & 3 & 4 & 5 & 6 & 7 \\
4 & 5 & 1 & 7 & 6 & 3 & 2
\end{matrix}\right)$
\\$\left(\begin{matrix}
1 & 2 & 3 & 4 & 5 & 6 & 7 \\
2 & 3 & 7 & 5 & 1 & 4 & 6
\end{matrix}\right)^6 = 
\left(\begin{matrix}
1 & 2 & 3 & 4 & 5 & 6 & 7 \\
5 & 1 & 2 & 6 & 4 & 7 & 3
\end{matrix}\right)$
\\$\left(\begin{matrix}
1 & 2 & 3 & 4 & 5 & 6 & 7 \\
2 & 3 & 7 & 5 & 1 & 4 & 6
\end{matrix}\right)^7 = 
\left(\begin{matrix}
1 & 2 & 3 & 4 & 5 & 6 & 7 \\
1 & 2 & 3 & 4 & 5 & 6 & 7
\end{matrix}\right)$
\\Thus $\left(\begin{matrix}
1 & 2 & 3 & 4 & 5 & 6 & 7 \\
2 & 3 & 7 & 5 & 1 & 4 & 6
\end{matrix}\right)$ has order 7 in $S^7$
\end{addmargin}
\end{addmargin} 

\textbf{(c): } $\left( \begin{matrix} 0 & -1 \\ 1 & 1\end{matrix}\right)$ in $GL(2,\mathbb{R})$
\begin{addmargin}[1em]{0em}
\underline{Solution}: 
\begin{addmargin}[1em]{0em}
$\left( \begin{matrix} 
0 & -1 \\ 
1 & 1
\end{matrix}\right)^2 = 
\left( \begin{matrix}
 -1 & -1 \\ 
 1 & 0
 \end{matrix}\right)$
\\$\left( \begin{matrix} 
0 & -1 \\ 
1 & 1
\end{matrix}\right)^3 = 
\left( \begin{matrix}
 -1 & 0 \\
 0 & -1
 \end{matrix}\right)$
\\$\left( \begin{matrix} 
0 & -1 \\ 
1 & 1
\end{matrix}\right)^4 = 
\left( \begin{matrix}
 0 & 1 \\
 -1 & -1
 \end{matrix}\right)$
\\$\left( \begin{matrix} 
0 & -1 \\ 
1 & 1
\end{matrix}\right)^5 = 
\left( \begin{matrix}
 1 & 1 \\
 -1 & 0
 \end{matrix}\right)$
\\$\left( \begin{matrix} 
0 & -1 \\ 
1 & 1
\end{matrix}\right)^6 = 
\left( \begin{matrix}
 1 & 0 \\
 0 & 1
 \end{matrix}\right)$
\\Thus, $\left(\begin{matrix}0&-1\\1&1\end{matrix}\right)$ has order 6 in $GL(2, \mathbb{R})$
\end{addmargin}
\end{addmargin} 

\textbf{(d): } $\left( \begin{matrix} -\frac{1}{2} & \frac{1}{2} \\ -\frac{3}{2} & -\frac{1}{2} \end{matrix} \right)$ in $GL(2,\mathbb{R})$
\begin{addmargin}[1em]{0em}
\underline{Solution}: 
\begin{addmargin}[1em]{0em}
$\left( \begin{matrix}
 -\frac{1}{2} & \frac{1}{2} \\
 -\frac{3}{2} & -\frac{1}{2} 
\end{matrix} \right)^2 = 
\left( \begin{matrix} 
 -\frac{1}{2} & -\frac{1}{2} \\
 \frac{3}{2} & -\frac{1}{2}
\end{matrix} \right)$
\\$\left( \begin{matrix}
 -\frac{1}{2} & \frac{1}{2} \\
 -\frac{3}{2} & -\frac{1}{2} 
\end{matrix} \right)^3 = 
\left( \begin{matrix} 
 1 & 0 \\
 0 & 1
\end{matrix} \right)$
\\Thus, $\left( \begin{matrix}
 -\frac{1}{2} & \frac{1}{2} \\
 -\frac{3}{2} & -\frac{1}{2} 
\end{matrix} \right)$ has order 3 in $GL(2, \mathbb{R})$
\end{addmargin}
\end{addmargin} 

\newpage
%----------------------------------------------------------------------------------------
\section*{Problem 9b}


\textbf{Problem statement}: List the order of each element of the group $U_{20}$
\\

\underline{Solution}: 
\begin{addmargin}[1em]{0em}
\begin{tabular}{c | c | c}
Number & Order & Proof \\ \hline
$1$ & $1$ & $1 = e$ \\
$3$ & $4$ & $3 * 3 = 9 * 3 = 7 * 3 = 1$\\
$7$ & $2$ & $7 * 7 = 1$\\
$9$ & $2$ & $9 * 9 = 1$\\
$11$ & $2$ & $11 * 11 = 1$\\
$13$ & $4$ & $13 * 13 = 9 * 13 = 17 * 13 = 1$ \\
$17$ & $4$ & $17 * 17 = 9 * 17 = 3 * 17 = 1$\\
$19$ & $2$ & $19 * 19 = 1$ \\
\end{tabular}
\end{addmargin}    

\newpage
%----------------------------------------------------------------------------------------
\section*{Problem 19}


\textbf{Problem statement}: If $a,b \in G$, prove that $|bab^{-1}| = |a|$
\\

\underline{Solution}: 
\begin{addmargin}[1em]{0em}
Suppose $a,b \in G$ and $k = |bab^{-1}|$
\\Then $(bab^{-1})^{k} = b^ka^kb^{-k} = e \implies b^ka^kb^{-k}b^k = eb^k \implies b^ka^k = eb^k \implies a^k = b \implies |a| = k$
\\Thus, $|bab^{-1}| = |a|$
\end{addmargin}    

\newpage
%----------------------------------------------------------------------------------------
\section*{Problem 20*}


\textbf{(a): }Show that $a = \left(\begin{matrix} 0 & 1 \\  -1 & -1 \end{matrix} \right)$ has order 3 in $GL(2,\mathbb{R})$ and $b = \left( \begin{matrix} 0 & -1 \\ 1 & 0\end{matrix} \right)$ has order 4.
\\

\underline{Solution}: 
\begin{addmargin}[1em]{0em}
$a = \left(\begin{matrix} 0 & 1 \\  -1 & -1 \end{matrix} \right)$
\\$a^2 = \left(\begin{matrix} -1 & -1 \\ 1 & 0 \end{matrix} \right)$
\\$a^3 = \left(\begin{matrix} 1 & 0 \\ 0 & 1 \end{matrix} \right)$
\\$b = \left( \begin{matrix} 0 & -1 \\ 1 & 0\end{matrix} \right)$
\\$b^2 = \left( \begin{matrix} -1 & 0 \\ 0 & -1\end{matrix} \right)$
\\$b^3 = \left( \begin{matrix} 0 & 1 \\ -1 & 0\end{matrix} \right)$
\\$b^4 = \left( \begin{matrix} 1 & 0 \\ 0 & 1\end{matrix} \right)$
\end{addmargin}    


\textbf{(b): }Show that $ab$ has infinite order
\\

\underline{Solution}:
\begin{addmargin}[1em]{0em}
Note that $ab = \left( \begin{matrix} 1 & 0 \\ -1 & 0\end{matrix}\right)$
\\So $(ab)^2 = \left( \begin{matrix} 1 & 0 \\ -2 & 0\end{matrix}\right)$
\\And $(ab)^3 = \left( \begin{matrix} 1 & 0 \\ -3 & 0\end{matrix}\right)$
\\Continuing the pattern, $(ab)^n = \left( \begin{matrix} 1 & 0 \\ -n & 0\end{matrix}\right) \neq I$ for $n > 0$
\\Thus, $ab$ has infinite order.
\end{addmargin}
\newpage
%----------------------------------------------------------------------------------------
\section*{Problem 33}


\textbf{Problem statement}: Assume that $a,b \in G$ and $ab = ba$.  If $|a|$ and $|b|$ are relatively prime, prove that $ab$ has order $|a||b|$
\\

\underline{Solution}: 
\begin{addmargin}[1em]{0em}
\begin{proof}
Let $G$ be a group and $a,b \in G$ such that $ab = ba$ and $(|a|,|b|) = 1$
\\Suppose that $n = |a|$ and $m = |b|$
\\Then first note that $(ab)(ab) = (ab)(ba) = ab^2a \implies (ab^2a)(ab) = (ab^2a)(ba) = a(b^2)(ba)a = ab^3a^2$
\\After $m$ iterations of this, $(ab)^m = ab^ma^{m-1} = aea^{m-1} = a^m$
\\Thus, $(ab)^{mn} = ((ab)^m)^n = (a^m)^n = (a^n)^m = e^m = e$
\\Therefore, note that $(ab)^{|a||b|} = e$
\\So, suppose $r = |ab|$.
\\Then $(ab)^r = e \implies (ab)^{rm} = e \implies a^{rm}b^{rm} = e \implies a^{rm} = e \implies n|rm \implies n|r$ (because $n$ and $m$ are coprime so $n \not| m$)
\\Also, $(ab)^r = e \implies (ab)^{rn} = e \implies a^{rn}b^{rn} = e \implies b^{rn} = e \implies m|rn \implies m|r$ (because $n$ and $m$ are coprime so $m \not| n$)
\\Thus $n|r$ and $m|r$ so $nm|r$ because $n$ and $m$ are coprime.
\\However, because $(ab)^{nm} = e, r|nm$
\\Therefore $r = nm \implies |ab| = |a||b|$
\end{proof}
\end{addmargin}    

\newpage
%----------------------------------------------------------------------------------------

\end{document}