%----------------------------------------------------------------------------------------
%   PACKAGES AND OTHER DOCUMENT CONFIGURATIONS
%----------------------------------------------------------------------------------------

\documentclass{article} % paper and 12pt font size

\usepackage{scrextend, tikz, amssymb}
\usepackage{amsmath,amsfonts,amsthm} % Math packages
\setlength\parindent{0pt} % Removes all indentation from paragraphs - comment this line for an assignment with lots of text

\reversemarginpar

%----------------------------------------------------------------------------------------
%   TITLE SECTION
%----------------------------------------------------------------------------------------

\newcommand{\horrule}[1]{\rule{\linewidth}{#1}} % Create horizontal rule command with 1 argument of height

\title{ 
\normalfont \normalsize 
\textsc{MATH 4120-001 --- Abstract Algebra} \\
\horrule{0.5pt} \\[0cm] % Thin top horizontal rule
\huge Section 2.1: 1b, 4*, 7, 12, 17*, 20, 22  \\ % The assignment title
\horrule{2pt} \\[0cm] % Thick bottom horizontal rule
}
\author{Andrew Shore} % Your name
\date{\normalsize\today} % Today's date or a custom date
\begin{document}

\maketitle % Print the title

%----------------------------------------------------------------------------------------
%   PROBLEM 1
%----------------------------------------------------------------------------------------
\section*{Problem 1b}


\textbf{Problem statement}: Show that $a^{p-1} \equiv 1 ($mod $p)$ for $a = 4$ and $p = 7$.
\\

\underline{Solution}: 
\begin{addmargin}[1em]{0em}
We need to show that $4^6 \equiv 1 ($mod $7)$.
\\Thus by definition, we must find a $k \in \mathbb{Z}$ such that $7k = 4^6 - 1 = 4096 - 1 = 4095$
\\Thus when $k = 585$, $7 * 585 = 4095$.
\\So we have shown that $4^6 \equiv 1($mod $7)$.
\end{addmargin}    

\newpage
%----------------------------------------------------------------------------------------

\section*{Problem 4*}

\textbf{Problem statement}: Virtually every item sold in a store has a 12-digit UPC barcode which is scanned at the checkout counter.  The first 11 digits of a UPC number $d_1d_2d_3 \ldots d_{11}$ identify the manufacturer and product.  The last digit $d_{12}$ is a check digit which is chosen so that 
\\ \hfill \break
\textit{$3d_1 + d_2 + 3d_3 + d_4 + 3d_5 + d_6 + 3d_7 + d_8 + 3d_9 + d_{10} + 3d_{11} + d_{12} \equiv 0 ($mod $ 10)$.}
\\ \hfill \break
If the congruence does not hold, an error has been made and the item must be scanned again, or the UPC code entered by hand.  Which of the following UPC numbers were scanned incorrectly?
\\

\textbf{(a)}: 037000356691
\\
\begin{addmargin}[1em]{0em}
\underline{Solution}: 
$3 * 0 + 1 * 3 + 3 * 7 + 1 * 0 + 3 * 0 + 1 * 0 + 3 * 3 + 1 * 5 + 3 * 6 + 1 * 6 + 3 * 9 + 1 * 1$
\\$= 0 + 3 + 21 + 0 + 0 + 0 + 9 + 5 + 18 + 6 + 27 + 1$
\\$= 90$
\\$= 9 * 10$ \checkmark
\begin{addmargin}[1em]{0em}

\end{addmargin}
\begin{addmargin}[1em]{0em}
\end{addmargin}
\end{addmargin}

\textbf{(b)}: 833732000625
\\
\begin{addmargin}[1em]{0em}
\underline{Solution}: $3 * 8 + 1 * 3 + 3 * 3 + 1 * 7 + 3 * 3 + 1 * 2 + 3 * 0 + 1 * 0 + 3 * 0 + 1 * 6 + 3 * 2 + 1 * 5$
\\$= 24 + 3 + 9 + 7 + 9 + 2 + 0 + 0 + 0 + 6 + 6 + 5$
\\$= 71$
\\$ = 7 * 10 + 1$ 
\\ \textbf{Scanned Incorrectly}
\begin{addmargin}[1em]{0em}

\end{addmargin}
\begin{addmargin}[1em]{0em}
\end{addmargin}
\end{addmargin}

\textbf{(c)}: 040293673034
\\
\begin{addmargin}[1em]{0em}
\underline{Solution}:$3 * 0 + 1 * 4 + 3 * 0 + 1 * 2 + 3 * 9 + 1 * 3 + 3 * 6 + 1 * 7 + 3 * 3 + 1 * 0 + 3 * 3 + 1 * 4$
\\$= 0 + 4 + 0 + 2 + 27 + 3 + 18 + 7 + 9 + 0 + 9 + 4$
\\$= 83$
\\$ = 8 * 10 + 3$
\\ \textbf{Scanned Incorrectly}
\begin{addmargin}[1em]{0em}

\end{addmargin}
\begin{addmargin}[1em]{0em}
\end{addmargin}
\end{addmargin}

\newpage
%----------------------------------------------------------------------------------------

\section*{Problem 7}


\textbf{Problem statement}: If $a \in \mathbb{Z}$, prove that $a^2$ is not congruent to 2 modulo 4 or to 3 modulo 4.
\\

\underline{Solution}: 
\begin{addmargin}[1em]{0em}
\begin{proof}
Let $a \in \mathbb{Z}$.
\\Then as proved in another theorem, $a \equiv 0($mod $4)$, $a \equiv 1($mod $4)$, $a \equiv 2($mod $4)$, or $a \equiv 3($mod $4)$.
\\Thus for each case, there exits $m \in \mathbb{Z}$.
\\ \underline{Case 1: $a \equiv 0($mod $4)$.}
\begin{addmargin}[1em]{0em}
So by definition, $4m = a$.
\\Thus, $a^2 = 16m^2 = 4(4m^2)$
\\So $a^2 \equiv 0($mod $4)$.
\end{addmargin}
\underline{Case 2: $a \equiv 1($mod $4)$.}
\begin{addmargin}[1em]{0em}
So by definition, $4m = a-1$.
\\Thus $(a-1)^2 = 16m^2 \implies a^2 - 2a + 1 = 16m^2$.
\\So using the case, $a = 4m + 1$, we get $a^2 = 16m^2 + 2(4m + 1) - 1 \implies a^2 = 16m^2 + 8m + 1 = 4(4m^2 + 8m) + 1 \implies a^2 - 1 = 4(4m^2 + 8m)$.
\\Thus, $a^2 \equiv 1($mod $4)$.
\end{addmargin}
\underline{Case 3: $a \equiv 2($mod $4)$.}
\begin{addmargin}[1em]{0em}
So by definition, $4m = a-2$.
\\Thus $(a-2)^2 = 16m^2 \implies a^2 - 4a + 4 = 16m^2$.
\\So using the case, $a = 4m + 2$, we get $a^2 = 16m^2 + 4(4m + 2) - 4 = 4(4m^2 + 4m) + 4 \implies a^2 = 4(4m^2 + 4m + 1)$.
\\Thus, $a^2 \equiv 0($mod $ 4)$.
\end{addmargin}
\underline{Case 4: $a \equiv 3($mod $4)$.}
\begin{addmargin}[1em]{0em}
So by definition, $4m = a-3$.
\\Thus $(a-3)^2 = 16m^2 \implies a^2 - 6a + 9 = 16m^2$.
\\So using the case, $a = 4m + 3$, we get $a^2 = 16m^2 + 6(4m + 3) - 9 = 4(4m^2 + 6m) + 9 = 4(4m^2 + 6m + 2) + 1 \implies a^2 - 1 = 4(4m^2 + 6m + 2).$
\\Thus, $a^2 \equiv 1 ($mod $4)$.
\end{addmargin}
Therefore for all values of $a$, $a^2 \equiv 0($mod $4)$ or $a^2 \equiv 0($mod $4)$.
\\Thus, $a^2$ is not congruent to 2 modulo 4 or 3 modulo 4.
\end{proof}
\end{addmargin}

\newpage
%----------------------------------------------------------------------------------------

\section*{Problem 12}


\textbf{Problem statement}: If $p \geq 5$ and $p$ is prime, prove that $[p] = [1]$ or $[p] = [5]$ in $\mathbb{Z}_6$. [\textit{Hint}: Theorem 2.3 and Corollary 2.5]
\\

\underline{Solution}: 
\begin{addmargin}[1em]{0em}
\begin{proof}
Let $p$ be prime such that $p \geq 5$.
\\Then, by Corollary 2.5, $[p]= [0]$ or $[p]= [1]$ or $[p]= [2]$ or $[p]= [3]$ or $[p]= [4]$ or $[p]= [5]$
\\ \underline{\textbf{Case 1: $[p] = [0]$}}
\begin{addmargin}[1em]{0em}
So by Theorem 2.3, $p \equiv 0($mod $6) \implies p - 0 = 6k \implies p = 6k$ for $k \in \mathbb{Z}$
\\However, this implies $6|p$ and because 6 is not prime, this leads to a contradiction.
\\Thus, $[p] \neq [0]$
\end{addmargin}
\underline{\textbf{Case 2: $[p] = [1]$}}
\begin{addmargin}[1em]{0em}
So by Theorem 2.3, $p \equiv 1($mod $6) \implies p - 1 = 6k \implies p = 6k - 1$ for $k \in \mathbb{Z}$
\\Thus, $[p]$ may equal $[1]$.
\end{addmargin}
\underline{\textbf{Case 3: $[p] = [2]$}}
\begin{addmargin}[1em]{0em}
So by Theorem 2.3, $p \equiv 2($mod $6) \implies p - 2 = 6k \implies p = 2(3k + 1)$ for $k \in \mathbb{Z}$
\\However, this implies $2|p$ and because $2 < 5$ and $p \geq 5$, this leads to a contradiction.
\\Thus, $[p] \neq [2]$
\end{addmargin}
\underline{\textbf{Case 4: $[p] = [3]$}}
\begin{addmargin}[1em]{0em}
So by Theorem 2.3, $p \equiv 3($mod $6) \implies p - 3 = 6k \implies p = 3(2k + 1)$ for $k \in \mathbb{Z}$
\\However, this implies $23|p$ and because $3 < 5$ and $p \geq 5$, this leads to a contradiction.
\\Thus, $[p] \neq [3]$
\end{addmargin}
\underline{\textbf{Case 5: $[p] = [4]$}}
\begin{addmargin}[1em]{0em}
So by Theorem 2.3, $p \equiv 4($mod $6) \implies p - 4 = 6k \implies p = 2(3k + 2)$ for $k \in \mathbb{Z}$
\\However, this implies $2|p$ and because $2 < 5$ and $p \geq 5$, this leads to a contradiction.
\\Thus, $[p] \neq [4]$
\end{addmargin}
\underline{\textbf{Case 6: $[p] = [5]$}}
\begin{addmargin}[1em]{0em}
So by Theorem 2.3, $p \equiv 5($mod $6) \implies p - 5 = 6k \implies p = 6k + 5$ for $k \in \mathbb{Z}$
\\Thus, $[p]$ may equal $[5]$.
\end{addmargin}
Because $[1]$ and $[5]$ are the only possible values for $[p]$, $[p] = [1]$ or $[p] = [5]$.
\end{proof}
\end{addmargin}

\newpage
%----------------------------------------------------------------------------------------

\section*{Problem 17*}


\textbf{Problem statement}: Prove that $10^n \equiv (-1)^n($mod $ 11)$ for every positive $n$.
\\

\underline{Solution}: 
\begin{addmargin}[1em]{0em}
\begin{proof}
To solve this problem, use induction on $n$, where $0 < n \in \mathbb{Z}$
\\ \marginpar{Base Case}
Note that when $n = 1$, $10^n = 10$ and $(-1)^n = -1$, so $10 - (-1) = 11$
\\Thus, $10^1 \equiv (-1)^1 ($mod $11)$.
\\ \marginpar{Hypothesis}
Assume $10^n \equiv (-1)^n($mod $11)$.
\\Thus, for $k \in \mathbb{Z}$, $10^n - (-1)^n = 11k$.
\\ \marginpar{Induction}
Note that $10^{n+1} = 10(10^n) = 10(11k - (-1)^n)$
\\Take this into 2 cases based off if $n$ is even or odd
\\ \underline{\textbf{Case 1: $n$ is even}}
\begin{addmargin}[1em]{0em}
Because $n$ is even, $(-1)^n = 1$
\\Thus, $10(11k + (-1)^n) = 10(11k + 1) = 11(10k) + 10 = 11(10k) + (11 - 1) = 11(10k + 1) - 1 = 11(10k + 1) + (-1)^{n+1}$
\\Because $n$ is even, so $(-1)^{n+1} = -1$.
\\So, $10^{n+1} - (-1)^{n+1} = 11(10k + 1) \implies 10^{n+1} \equiv (-1)^{n+1}($mod $11)$.
\end{addmargin}
\underline{\textbf{Case 2: $n$ is odd}}
\begin{addmargin}[1em]{0em}
Because $n$ is even, $(-1)^n = -1$
\\Thus, $10(11k + (-1)^n) = 10(11k - 1) = 11(10k) + 10 = 11(10k) - (11 - 1) = 11(10k - 1) + 1 = 11(10k - 1) + (-1)^{n+1}$
\\Because $n$ is even, so $(-1)^{n+1} = 1$.
\\So, $10^{n+1} - (-1)^{n+1} = 11(10k - 1) \implies 10^{n+1} \equiv (-1)^{n+1}($mod $11)$.
\end{addmargin}
Thus, in each case, $10^{n+1} \equiv (-1)^{n+1}($mod $11)$ and so by the principle of incuction, the original proposition is proved.
\end{proof}
\end{addmargin}

\newpage
%----------------------------------------------------------------------------------------

\section*{Problem 20}


\textbf{Problem statement (a)}: Prove or disprove: If $a^2 \equiv b^2 ($mod $n)$, then $a \equiv b($mod $ n)$ or $a \equiv -b($mod $n)$. 
\\

\underline{Solution}: 
\begin{addmargin}[1em]{0em}
This is false.
\\As a counterexample, suppose $a = 8$, $b = 2$, and $n = 15$.
\\Then $64 \equiv 4 ($mod $ 15)$.
\\However, $8 \not\equiv 2($mod $15)$ and $8 \not\equiv -2($mod $15)$.
\end{addmargin}

\hfill \break

\textbf{Problem statement (b)}: Do part (a) when $n$ is prime. 
\\

\underline{Solution}: 
\begin{addmargin}[1em]{0em}
\begin{proof}
Let $p$ be prime and $a^2 \equiv b^2($mod $p)$.
\\Then for $k \in \mathbb{Z}$, $a^2 - b^2 = (a-b)(a+b) = pk$
\\Because $p$ is prime, it's only positive prime factor is $p$.  
\\Thus because the right hand side is the product of two integers, $p|(a-b)$ or $p|(a+b)$.
\\Thus, $a \equiv b($mod $p)$ or $a \equiv -b($mod $p)$.
\end{proof}
\end{addmargin}

\newpage
%----------------------------------------------------------------------------------------

\section*{Problem 22}


\textbf{Problem statement (a)}: Give an example to show that the following statement is false: If $ab \equiv ac($mod $n)$ and $a \not\equiv 0 ($mod $n)$, then $b \equiv c($mod $ n)$. 
\\

\underline{Solution}: 
\begin{addmargin}[1em]{0em}
Let $a = 2$, $b = 3$, $c = -1$, $n = 8$.
\\Then $ab = 6$ and $ac = -2$ and $6 \equiv -2 ($mod $8)$.
\\However, $3 \not\equiv -2 ($mod $8)$
\end{addmargin}

\hfill \break

\textbf{Problem statement (b)}: Prove that the statement in part (a) is true whenever $(a,n) = 1$.
\\

\underline{Solution}: 
\begin{addmargin}[1em]{0em}
\begin{proof}
Let $a,b,c,0<n \in \mathbb{Z}$ such that $ab \equiv ac($mod $n)$, $a \not\equiv 0($mod $n)$, and $(a,n) = 1$.
\\Thus, for $k \in \mathbb{Z}$, $ab - ac = a(b-c)=nk$
\\However, $(a,n) = 1$, so thus all the prime factors of $a$ must be in $k$ and not $n$, so $k$ can be rewritten as $as$ where $s \in \mathbb{Z}$.
\\Also, because $a \not\equiv 0($mod $n)$, $a$ is nonzero and thus has at least one nonzero factor.
\\Thus, $a(b-c) = nk = nas \implies b - c = ns \implies b \equiv c($mod $s)$.
\end{proof}
\end{addmargin}

%----------------------------------------------------------------------------------------

\end{document}