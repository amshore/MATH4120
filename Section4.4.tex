%----------------------------------------------------------------------------------------
%   PACKAGES AND OTHER DOCUMENT CONFIGURATIONS
%----------------------------------------------------------------------------------------

\documentclass{article} % paper and 12pt font size

\usepackage{scrextend, tikz, amssymb}
\usepackage{amsmath,amsfonts,amsthm} % Math packages
\setlength\parindent{0pt} % Removes all indentation from paragraphs - comment this line for an assignment with lots of text

%----------------------------------------------------------------------------------------
%   TITLE SECTION
%----------------------------------------------------------------------------------------

\newcommand{\horrule}[1]{\rule{\linewidth}{#1}} % Create horizontal rule command with 1 argument of height

\title{ 
\normalfont \normalsize 
\textsc{MATH 4120-001 --- Abstract Algebra} \\
\horrule{0.5pt} \\[0cm] % Thin top horizontal rule
\huge Section 4.4: 4, 7, 9*, 12*, 24 \\ % The assignment title
\horrule{2pt} \\[0cm] % Thick bottom horizontal rule
}
\author{Andrew Shore} % Your name
\date{\normalsize\today} % Today's date or a custom date
\begin{document}

\maketitle % Print the title

%----------------------------------------------------------------------------------------
%   PROBLEM 1
%----------------------------------------------------------------------------------------
\section*{Problem 4}


\textbf{(a):}For what value of $k$ is $x-2$ a factor of $x^4 - 5x^3 + 5x^2 + 3x + k$ in $\mathbb{Q}[x]$? 
\begin{addmargin}[1em]{0em}
\underline{Solution}: 
\begin{addmargin}[1em]{0em}
When $k = 6$, $x^4 - 5x^3 + 5x^2 + 3x + 6 = (x-2)(x^3 - 4x^2 - 3x - 3)$
\end{addmargin}
\end{addmargin}    


\textbf{(b):}For what value of $k$ is $x+1$ a factor of $x^4 + 2x^3 - 3x^2 + kx + 1$ in $\mathbb{Z}_5[x]$? 
\begin{addmargin}[1em]{0em}
\underline{Solution}: 
\begin{addmargin}[1em]{0em}
When $k = 2$, $x^4 + 2x^3 - 3x^2 + 2x + 1 = (x + 1)(x^3 + x^2 + x + 1)$
\end{addmargin}
\end{addmargin}    
\newpage
%----------------------------------------------------------------------------------------

\section*{Problem 7}

\textbf{Problem statement}: Use the Factor Theorem to show that $x^7 - x$ factors in $\mathbb{Z}_7[x]$ as $x(x-1)(x-2)(x-3)(x-4)(x-5)(x-6)$, without doing any polynomial multiplication.
\\

\underline{Solution}: 
\begin{addmargin}[1em]{0em}
By the division algorithm, $f(0) = f(1) = f(2) = f(3) = f(4) = f(5) = f(6) = 0$
\\$f(0) = (0)^ 7 - (0) = 0 - 0 = 0$
\\$f(1) = (1)^7 - (1) = 1 - 1 = 0$
\\$f(2) = (2)^7 - (2) = 128 - 2 = 2 - 2 = 0$
\\$f(3) = (3)^7 - (3) = 2187 - 3 = 3 - 3 = 0$
\\$f(4) = (4)^7 - (4) = 16384 - 4 = 4 - 4 = 0$
\\$f(5) = (5)^7 - (5) = 78125 - 5 = 5 - 5 = 0$
\\$f(6) = (6)^7 - (6) = 279936 - 6 = 6 - 6 = 0$
\end{addmargin}

\newpage
%----------------------------------------------------------------------------------------

\section*{Problem 9*}


\textbf{Problem statement}: List all monic irreducilbe polynomials of degree 2 in $\mathbb{Z}_3[x]$.  Do the same in $\mathbb{Z}_5[x]$
\\

$\mathbb{Z}_3[x]$
\begin{addmargin}[1em]{0em}
$x^2 + 1$
\\$x^2 + x + 2$
\\$x^2 + 2x + 2$
\end{addmargin}

$\mathbb{Z}_5[x]$
\begin{addmargin}[1em]{0em}
$x^2 + 2$
\\$x^2 + 3$
\\$x^2 + x + 1$
\\$x^2 + x + 2$
\\$x^2 + 2x + 3$
\\$x^2 + 2x + 4$
\\$x^2 + 3x + 3$
\\$x^2 + 3x + 4$
\\$x^2 + 4x + 1$
\\$x^2 + 4x + 2$
\end{addmargin}

\newpage
%----------------------------------------------------------------------------------------

\section*{Problem 12*}


\textbf{Problem statement}: If $a \in F$ is a nonzero root of $c_nx^n + c_{n-1}x^{n-1} + ... + c_1x + c_0 \in F[x]$, show that $a^{-1}$ is a root of $c_0x^n + c_{1}x^{n-1} + ... + c_{n-1}x + c_n$
\\

Solution: 
\begin{addmargin}[1em]{0em}
\begin{proof}
Let $F$ be a field and $c_nx^n + c_{n-1}x^{n-1} + ... + c_1x + c_0 \in F[x]$ have a nonzero root $a \in F$
\\Then $f(a) = c_na^n + c_{n-1}a^{n-1} + ... + c_1a + c_0 = 0$
\\So $a^{-n}f(a) = a^{-n}(0) = 0 = (a^{-n})( c_na^n + c_{n-1}a^{n-1} + ... + c_1a + c_0) = c_na^{n-n} + c_{n-1}a^{n-1 - n} + ... + c_1a^{1-n} + c_0a^{0-n} = c_na^{0} + c_{n-1}a^{-1} + ... + c_1a^{-n + 1} + c_0a^{-n} = c_0a^{-n} + c_1a^{-n + 1} + ... + c_{n-1}a + c_n = f(a^{-1})$
\\Therefore, $a^{-1}$ is a root of $c_0x^n + c_{1}x^{n-1} + ... + c_{n-1}x + c_n$
\end{proof}
\end{addmargin}

\newpage
%----------------------------------------------------------------------------------------

\section*{Problem 24}


\textbf{Problem statement}: Let $a$ be a fixed element of $F$ and define a map $\varphi_a:F[x] \rightarrow F$ by $\varphi_a[f(x)] = f(a)$.  Prove that $\varphi_a$ is a surjective homomorphism of rings.  The map $\varphi_a$ is called a \textbf{evaluation homomorphism;} there is one for each $a \in F$.
\\

Solution: 
\begin{addmargin}[1em]{0em}
\begin{proof}
Let $F$ be a field and $a \in F$ be fixed.
\\Define a map $\varphi_a: F[x] \rightarrow F$ by $\varphi_a[f(x)] = f(a)$
\\Suppose $g(x), h(x) \in F[x]$
\\Then $\varphi_a(g(x) + h(x)) = \varphi_a(\sum_{k=0}^{\infty}{g_kx^k} + \sum_{k=0}^{\infty}{h_kx^k}) = \varphi_a(\sum_{k=0}^{\infty}{(g_k + h_k)x^k}) = \sum_{k=0}^{\infty}{(g_k + h_k)a^k} = \sum_{k=0}^{\infty}{g_ka^k} + \sum_{k=0}^{\infty}{h_ka^k} = \varphi_a(g(x)) + \varphi_a(h(x))$
\\Also, $\varphi_a(g(x)h(x)) = \varphi_a(\sum_{k=0}^{\infty}{g_kx^k} \sum_{k=0}^{\infty}{h_kx^k}) = \varphi_a(\sum_{k=0}^{\infty}{(\sum_{r=0}^{k}{g_r+h_{k-r}})x^k}) = \sum_{k=0}^{\infty}{(\sum_{r=0}^{k}{g_r+h_{k-r}})a^k} = \sum_{k=0}^{\infty}{g_ka^k} \sum_{k=0}^{\infty}{h_ka^k} = \varphi_a(g(x)) \varphi_a(h(x))$
\\Thus, $\varphi_a$ is a homomorphism of rings
\\In addition, suppose $z \in F$
\\Then when $f(x) = zx^0$, $\varphi_a(f(x)) = z$
\\Thus $\varphi_a$ is surjective
\end{proof}
\end{addmargin}

%----------------------------------------------------------------------------------------

\end{document}